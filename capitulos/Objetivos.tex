\chapter{Objetivos}
Los objetivos inicialmente previstos en la propuesta de TFG en matemáticas fueron:
\begin{itemize}
\item Realizar una recopilación de algunos teoremas de la alternativa.
\item Teorema de dualidad en programación lineal.
\item Teoremas de Karush-Kuhn-Tucker y Fritz John para programación convexa.
\item Aplicación a finanzas: teorema fundamental de valoración de activos financieros en mercados finitos.
\end{itemize}

Sin embargo, nuestro tratamiento final ha sido algo más ambicioso, pues hemos incluido todo un capítulo de aplicaciones de los teoremas de la alternativa a la teoría minimax y a la separación de convexos. Además, en lugar de considerar los teoremas de Karush-Kuhn-Tucker y Fritz John en un ambiente convexo, los hemos establecido en un contexto no lineal y diferenciable. La idea que nos ha llevado a ello ha sido aumentar el número y tipología de aplicaciones de los teoremas de la alternativa, mostrando su versatilidad en diversos campos.\\

Dentro de la parte informática se pretendía:
\begin{itemize}
	\item Realizar una breve incursión en el mundo de la digitalización 3D, conociendo a rasgos generales el proceso empleado en la obtención de un modelo.
	\item Estudio de la información geométrica del modelo obtenido.
	\item Uso de dicha información para proponer una optimización de los algoritmos empleados en el proceso.
	\item Realización de un programa para evaluar los resultados obtenidos.
\end{itemize}

Destacamos que también se han alcanzado los objetivos propuestos en este caso. Inicialmente, la propuesta de TFG no contenía ninguna especificación acerca de los algoritmos o técnicas a usar debido a la amplia cantidad de métodos existentes así como la complejidad del proceso completo de digitalizado. Así, el primer punto ha servido como fase previa para el conocimiento del tema a trabajar para poder concretar un objetivo. Tras ello, se decidió centrar el estudio en la etapa de alineado tal, como ya se ha comentado anteriormente, y orientar de ese modo, los puntos segundo y tercero en ese ámbito. Finalmente, el cuarto punto se ha enfocado como un banco de pruebas para comprobar los resultados obtenidos en los puntos anteriores y no como una aplicación para uso de terceros. Esto ha conllevado el uso de nuevas herramientas de desarrollo así como profundizar en conceptos clave de la Informática Gráfica. De esto modo, se han alcanzado estos objetivos de una manera colateral. 