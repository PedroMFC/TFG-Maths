\chapter{Objetivos}
Los objetivos inicialmente previstos en la propuesta de TFG fueron:
\begin{itemize}
\item Realizar una recopilación de algunos teoremas de la alternativa.
\item Teorema de dualidad en programación lineal.
\item Teoremas de Karush-Kuhn-Tucker y Fritz John para programación convexa.
\item Aplicación a finanzas: teorema fundamental de valoración de activos financieros en mercados finitos.
\end{itemize}

Sin embargo, nuestro tratamiento final ha sido algo más ambicioso, pues hemos incluido todo un capítulo de aplicaciones de los teoremas de la alternativa a la teoría minimax y a la separación de convexos. Además, en lugar de considerar los teoremas de Karush-Kuhn-Tucker y Fritz John en un ambiente convexo, los hemos establecido en un contexto no lineal y diferenciable. La idea que nos ha llevado a ello ha sido aumentar el número y tipología de aplicaciones de los teoremas de la alternativa, mostrando su versatilidad en diversos campos.