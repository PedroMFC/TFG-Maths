\chapter*{Lema de Simons y Teorema de la Alternativa de Gordan}
	\thispagestyle{empty}

	\paragraph{} Enunciamos ahora el siguiente lema en que destacamos la ausencia de hipótesis topológicas, lo que será importante posteriormente \cite{Simons2008}.
	
	\begin{lemaBox}
		Sea $ C $ un subconjunto no vacío y convexo de un espacio vectorial. Dadas $ f_1, ..., f_N $  funciones sobre $ C $ reales y convexas, entones existen $ \lambda_1, ..., \lambda_N \geq 0$ tales que $ \lambda_1+ ...+ \lambda_N = 1$ y
		\[
		\inf_C\left[ m\acute{a}x \{f_1, ..., f_N\}\right] = \inf_C \left[ \lambda_1 f_1+ ...+\lambda_N f_N \right]
		\] 
	\end{lemaBox}
	\begin{proof}
		Sea  \vecSpace = $ \RR^N $. Definimos $S:\vecSpace \longrightarrow P $ como \[ S(x_1, ..., x_N) := m\acute{a}x \{x_1, ..., x_N\} \]. Claramente, $ S $ es positivamente homogénea. También es subaditiva ya que dados $ x,y \in \vecSpace $ : 
		\begin{equation*}
		\begin{split}
		S(x+y) &= m\acute{a}x \{x_1 + y_1, ..., x_N +y_N \}\\ 
		&\leq m\acute{a}x \{x_1, ..., x_N\} + m\acute{a}x \{y_1, ...,y_N\} = S(x) + S(y)
		\end{split}
		\end{equation*}
		
		Por ello, $ S $ es sublineal. Tomamos el subconjunto:
		\[ 
		D = \{ (x_1, ..., x_N)\in V: \exists c \in C \quad tal \quad que \quad \forall i = 1,...N,\quad f_i(c) \leq x_i \}
		\]
		
		Veamos que $ D $ es un subconjunto convexo de \vecSpace. Sean $ x, y \in D $, por ello, existen $ c_x, c_y \in C $ tales que $ f_i (c_x) \leq x_i  $ y $ f_i (c_y) \leq y_i \quad \forall i=1,...,N $. Dado $ \lambda \in [0,1] $, llamamos $ c := (1-\lambda)c_x + \lambda c_y $ que pertenece a $ C $ por ser este convexo. Veamos que $ c $ es el elemento necesario de $ C $ para que cualquier combinación convexa de $ x $ e $ y $ esté en $ D $. Así:	
		\[
		f_i(c) = f_i((1-\lambda)c_x + \lambda c_y) \leq (1-\lambda)f(c_x) + \lambda f(c_y) \leq (1-\lambda)x_i + \lambda y_i \quad , \forall i =1,...,N 
		\]
		donde la primera desigualdad se debe a que las $ f_i $ son convexas y la segunda a que $ x,y \in D $. Por ello, $ (1-\lambda)x_i + \lambda y_i \in D , \quad \forall \lambda \in [0,1] $ por lo que $ D $ es convexo. Aplicando el Teorema de Mazur-Orlizc, existe $ L $ sobre \vecSpace tal que $ L \leq S $ e $ \inf_D L = \inf_D S $. \\
		
		Al ser $ L $ lineal, existen $ \lambda_1, .., \lambda_N \in \RR$ tales que: 
		\[
		L(x) =  \lambda_1 x_1 + ..+ \lambda_N x_N , \forall x \in V \] 
		
		Como $ L \leq S $ tenemos que $ \lambda_1 x_1 + ..+ \lambda_N x_N \leq m\acute{a}x \{x_1, ..., x_N\} $
		por lo que también se tiene que cumplir que $ \lambda_1, ..., \lambda_N \geq 0$ y $ \lambda_1+ ...+ \lambda_N = 1$. \\
		
		Finalmente:
		\[
		\inf_D L = \inf_{c\in C} \left[ \lambda_1 f(c) + ..+ \lambda_N f(c) \right] = \inf_{C} \left[ \lambda_1 f + ..+ \lambda_N f \right]
		\]
		y
		\[
		\inf_D S = \inf_{c\in C} \left[ m\acute{a}x \{f_1(c), ..., f_N(c)\} \right] = \inf_C\left[ m\acute{a}x \{f_1, ..., f_N\}\right] 
		\]
		por lo que 
		\[ \inf_{C} \left[ \lambda_1 f + ..+ \lambda_N f \right] = \inf_C\left[ m\acute{a}x \{f_1, ..., f_N\}\right]  \] 
	\end{proof}

