\section{Dualidad en programación lineal}
\newcommand{\muu}{\boldsymbol{\mathbf{\mu}}}

Gracias al lema de Farkas podemos probar resultados como el teorema de dualidad en programación lineal que enunciamos a continuación. La prueba se recoge en \cite{borwein}.
\bigskip
\begin{teoremaBox}(Teorema de dualidad en programación lineal)
	Dados $ N,M \in \RR $ tomamos la matriz $ A \in \RR^{N\times M} $ y los vectores $ \bb \in \RR^M $ y $ \cc \in \RR^N $. Consideramos el problema de optimización primal:
	
	\begin{equation*}\label{primal}\tag{$ P $}
	\begin{rcases*}
	p:=\displaystyle \inf_{\xx} \cc^T \xx\\
	\begin{split}
	\text{s.a } A^T \xx &\leq \bb \\
	\xx &\in \RR^N
	\end{split}
	\end{rcases*},
	\end{equation*}
	donde $ A^T \xx \leq \bb $ nos aporta un total de $ M $ restricciones. Sea el problema dual:
	
	\begin{equation*}\label{dual}\tag{$ D $}
	\begin{rcases*}
	d:= \displaystyle \sup_{\yy} -\bb^T \yy\\
	\begin{split}
	\text{s.a } A\yy &= -\cc \\
	\yy &\in \RR^M_+
	\end{split}
	\end{rcases*} .
	\end{equation*}
	
	Entonces, la solución del problema primal y el dual coincide, o lo que es lo mismo, $ p = d $.
\end{teoremaBox}
\begin{proof}
	En primer lugar, veamos que $ p \geq d $. Para ello, tomamos $ \xx \in \RR^N$ e $ \yy \in \RR^N $ factibles, es decir, cumplen las restricciones de sus respectivos problemas. Así:
	\[
	\begin{cases}
	A^T \xx - \bb \leq 0 \\
	\yy \geq 0
	\end{cases}
	\]
	\[
	\big\Downarrow
	\]
	\[
	(A^T \xx -\bb)^T\yy =\xx^TA\yy -\bb^T\yy\leq 0
	\]
	\[
	\big\Updownarrow
	\]
	\[
	-\bb^T\yy\leq \cc^T \xx,
	\]
	donde la última desigualdad se debe a que $ A\yy = -\cc $. Tomando supremos en la izquierda e ínfimos en la derecha llegamos a $ d \leq p $. \\
	
	Supongamos ahora que $ p \in \RR $ ya que en caso contrario habríamos terminado. Consideramos el problema homogeneizado, 
	\begin{equation*}
	\begin{rcases*}
	\begin{split} 
	A^T \xx - z\bb&\leq 0  \\
	-\cc^T\xx  + zp &>0  \\
	\xx \in \RR^N , \hspace{1mm} z &\geq 0
	\end{split}
	\end{rcases*}.
	\end{equation*} 
	Este problema no tiene solución ya que en caso contrario, si $ z > 0 $, entonces $ A^T \displaystyle \frac{\xx}{z} < \bb$ y, por la definición de $ p $, $ \cc^T \frac{\xx}{z} \geq p $ si, y solo si, $ zp \leq \cc^T \xx $, lo que no es posible, pues debe ser $ zp > \cc^T\xx $. En el caso $ z=0 $, el problema queda \\
	\begin{equation}\label{eq}
	\begin{rcases*}
	\begin{split} 
	A^T \xx &\leq 0  \\
	\cc^T\xx &<0 
	\end{split}
	\end{rcases*}.
	\end{equation}
	Sea $ \uu \in \RR^N$ un punto factible del problema primal, esto es,
	\[
	A^T\uu \leq \bb.
	\]
	Entonces, dado $ \rho > 0$,
	\begin{equation*}
	\begin{split}
	A^T(\rho\xx + \uu) &= \rho A^T\xx + A^T\uu \\
	&\leq \bb
	\end{split}
	\end{equation*}
	ya que $ A^T\xx \leq 0$, $ \rho >0 $ y $ A^T\uu \leq \bb $. Usando la definición de $ p $, 
	\begin{equation*}
	\begin{split}
	p &\leq \cc^T(\rho\xx + \uu)\\
	&=\rho\cc^T\xx + \rho\uu.
	\end{split}
	\end{equation*}
	Como $ p \in \RR $, $ \cc^T\xx < 0 $ y $ \rho >0 $ es arbitrario, llegamos a una contradicción (tómese límite cuando $ \rho \longrightarrow +\infty $). Aplicamos el lema de Farkas al problema homogeneizado, y al no tener el sistema solución observamos que no se puede dar la alternativa $ ii') $. Entonces, tenemos que existen escalares $\mu_1,\dots,\mu_M,\beta \geq 0 $ que cumplen
	\[
	\sum_{j=1}^{M}\mu_j (A_{j},-b_j) + \beta(0,-1) = (-\cc,p),
	\]
	donde para cada $ j = 1,\dots,M $, $ A_j $ denota la fila j-ésima de $ A^T $. Reescribiendo lo anterior, llamamos $ \muu = \begin{pmatrix}
	\mu_{1} \\
	\vdots \\
	\mu_{M}
	\end{pmatrix} $ y obtenemos
	
	\begin{equation*}
	\begin{rcases*}
	\begin{split} 
	\sum_{j=1}^{M}\mu_j A_{j} = A \muu &= -\cc\\
	\mu &\in \RR^M_+ \\
	-\bb^T \muu - \beta &= p
	\end{split}
	\end{rcases*}.
	\end{equation*}
	
	El vector $ \mu $ es una solución factible para el problema dual $ (D) $ con valor objetivo al menos $ p $ obteniendo entonces que $ p \leq d $, quedando así probado el resultado.
\end{proof}
\bigskip