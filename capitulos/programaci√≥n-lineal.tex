\section{Dualidad en programación lineal}
Gracias al lema de Farkas podemos probar resultados como el teorema de dualidad en programación lineal que enunciamos a continuación. La prueba se recoge en \cite{borwein}.
\bigskip
\begin{teoremaBox}(Teorema de dualidad en programación lineal)
	Dados $ N,M \in \RR $ tomamos la matriz $ A \in \RR^{N\times M} $ y los vectores $ \bb \in \RR^M $ y $ \cc \in \RR^N $. Consideramos el problema de optimización primal:
	
	\begin{equation*}\label{primal}
	\begin{rcases*}
	p:=\displaystyle \inf_{x} \cc^T x\\
	\begin{split}
	\text{s.a } A^T x &\leq \bb \\
	x &\in \RR^N
	\end{split}
	\end{rcases*} (P),
	\end{equation*}
	
	donde $ A^T x \leq \bb $ nos aporta un total de $ M $ restricciones. Sea el problema dual:
	
	\begin{equation*}\label{dual}
	\begin{rcases*}
	d:= \displaystyle \sup_{y} -\bb^T y\\
	\begin{split}
	\text{s.a } Ay &= -\cc \\
	y &\in \RR^N_+
	\end{split}
	\end{rcases*} (D).
	\end{equation*}
	
	Entonces, la solución del problema primal y el dual coincide, o lo que es lo mismo, $ p = d $.
\end{teoremaBox}
\begin{proof}
	En primer lugar, veamos que $ p \geq d $. Para ello, tomamos $ x \in \RR^N$ e $ y \in \RR^N $ factibles, es decir, cumplen las restricciones de sus respectivos problemas. Así:
	\[
	\begin{cases}
	A^T x - \bb \leq 0 \\
	y \geq 0
	\end{cases}
	\]
	\[
	\big\Downarrow
	\]
	\[
	(A^T x -\bb)^Ty =x^TAy -\bb^Ty\leq 0
	\]
	\[
	\big\Updownarrow
	\]
	\[
	-\bb^Ty\leq \cc^T x,
	\]
	donde la última desigualdad se debe a que $ Ay = -\cc $. Tomando ínfimos en la izquierda y supremos en la derecha llegamos a $ p \geq d $. \\
	
	Supongamos ahora que $ p \in \RR $. Consideramos el problema homogeneizado:
	
	\begin{equation*}
	\begin{rcases*}
	\begin{split} 
	A^T x - z\bb&\leq 0  \\
	\cc^Tx  - zp &>0  \\
	x \in \RR^N , \hspace{1mm} z &\leq 0
	\end{split}
	\end{rcases*}.
	\end{equation*}
	Este problema no tiene solución ya que en caso contrario, si $ z > 0 $ entonces $ A^T \displaystyle \frac{x}{z} < \bb$ y, por la definición de $ p $, $ \cc^T \frac{x}{z} \geq p $ si, y solo si, $ zp \leq \cc^T x $, lo que no es posible, pues debe ser $ zp > \cc^Tx $. En el caso $ z=0 $, el problema\\
	\begin{equation*}
	\begin{rcases*}
	\begin{split} 
	A^T x &\leq 0  \\
	\cc^Tx &<0  \\
	x \in \RR^N , \hspace{1mm} z &\leq 0
	\end{split}
	\end{rcases*}.
	\end{equation*}
	
	 $ p= \inf_{x} \cc^T x $ y como $ z \geq 0 $ se tiene que $ -zp \leq \cc^T x $. Aplicamos el lema de Farkas, y al no tener el sistema solución observamos que no se puede dar la alternativa $ ii') $. Para ello, consideramos cada fila de la matriz $ A^T $ como un vector de $ \RR^N $, teniendo un total de $ M $ vectores. Notamos las filas de la matriz $ A^T $ como $ A_{j\cdot} $ con $ j=1,\dots,M $. Entonces, tenemos que existen escalares $\mu_1,\dots,\mu_M,\beta \geq 0 $ que cumplen
	\[
	\sum_{j=1}^{M}\mu_j (A_{j\cdot},-b_j) + \beta(0,-1) = (-\cc,p).
	\]
	Reescribiendo lo anterior, llamamos $ \mu = \begin{pmatrix}
	\mu_{1} \\
	\vdots \\
	\mu_{M}
	\end{pmatrix} $ y obtenemos
	
	\begin{equation*}
	\begin{rcases*}
	\begin{split} 
	\sum_{j=1}^{M}\mu_j A_{j\cdot} = A \mu &= -\cc\\
	\mu &\in \RR^M_+ \\
	-\bb^T \mu - \beta &= p
	\end{split}
	\end{rcases*}.
	\end{equation*}
	
	El vector $ \mu $ es una solución factible para el problema dual $ (D) $ con valor objetivo al menos $ p $ obteniendo entonces que $ p \leq d $, quedando así probado el resultado.
\end{proof}
\bigskip