\chapter{Minimax}
\newcommand{\topSpace}{X}
\newcommand{\topSpaceY}{Y}

\paragraph{}En esta sección llegaremos a otro de los resultados clave del trabajo. Será uno de los denominados teoremas Minimax. A rasgos generales y a modo introductorio, podemos decir que un teroema Minimax es un resultado que afirma, bajo ciertas hipótesis, que:
\[
\inf_{y \in Y} \sup_{x \in X} f(x,y) = \sup_{x \in X} \inf_{y \in Y} f(x,y),
\] 
donde $ X \text{ e } Y$ son subconjuntos de un espacio vectorial y $ f: X \times Y \longrightarrow \RR $. Obviamente, esta igualdad no es cierta en general tal y como mostramos en el siguiente ejemplo. Definimos $ f:\{0,1\} \times \{0,1\} \longrightarrow \RR$ como:
\[
f(x,y) = \begin{cases}
0, & \mbox{si $ x=y $ } \\
1, & \mbox{si $ x\neq y$ }
\end{cases}.
\]
Por un lado tenemos
\[
\inf_{ y \in Y}\sup_{x \in X} f(x,y) = \min_{ y \in Y}\max_{x \in X} f(x,y) = \min_{ y \in Y}\{1\} = 1,
\]
y por otro
\[
\sup_{x \in X} \inf_{ y \in Y}f(x,y) = \max_{x \in X}\min_{ y \in Y}f(x,y) = \min_{ y \in Y}\{0\} = 0.
\]
Es claro, como muestra el ejemplo, que la desigualdad 
\[
\inf_{y \in Y} \sup_{x \in X} f(x,y) \geq \sup_{x \in X} \inf_{y \in Y} f(x,y)
\] 
siempre se da ya que
\[
\sup_{x \in X} f(x,y) \geq f(x,y) \geq \inf_{y \in Y}f(x,y) .
\]
Por lo ello, algunas veces los teoremas Minimax solo nos aportan la otra desigualdad necesaria. \\

Antes de continuar, exponemos la siguiente definición que aparecerá posteriormente en el teorema. Se trata de una propiedad más débil que la continuidad para funciones reales. 
\begin{definicion}
Sea $ \topSpace $ un espacio topológico. Decimos que $ f:\topSpace \longrightarrow \RR $ es superiormente semicontinua si para todo $ r \in \RR $ se cumple que el conjunto $ \lbrace x \in \topSpace : f(x) \geq r \rbrace $ es cerrado.
\end{definicion}

Por ejemplo, la función $ f:\RR \longrightarrow \RR $ dado por 
\[
f(x) = \begin{cases}
0, & \mbox{si $ x < 0 $ } \\
1, & \mbox{si $x \geq 0$ }
\end{cases}
\]
es superiormente semicontinua. \\

En estos momentos nos encontramos en condiciones de enunciar y demostrar nuestro teorema Minimax.
\begin{teoremaBox}\label{MinMax}
Sean $ \topSpace, \topSpaceY $ subconjuntos convexos de espacios vectoriales (no tienen que ser el mismo) tal que $ \topSpace $ está dotado de una topología que lo hace compacto.Supongamos además que $ f:  \topSpace \times \topSpaceY \longrightarrow \RR $ es:
\begin{itemize}	
\item[i)] cóncava y superiormente semicontinua en $ \topSpace $ y
\item[ii)] convexa en $ \topSpaceY $.
\end{itemize}
Entonces:
\begin{equation*}\label{eqMinMax}
\inf_{y \in Y} \max_{x \in X} f(x,y) = \max_{x \in X} \inf_{y \in Y} f(x,y).
\end{equation*}
\end{teoremaBox}
\begin{proof}
En primer lugar, podemos escribir máximo en ambos casos en vez de supremo ya que $ f $ es superiormente semicontinua en $ \topSpace $, por ello $ \inf_{y \in Y} f(x,y) $ también lo es (referenciar), y $ \topSpace $ es compacto (referenciar). Como hemos explicado anteriormente, solo necesitamos la desigualdad 
\begin{equation}\label{desAux1}
\inf_{y \in Y} \max_{x \in X} f(x,y) \leq \max_{x \in X} \inf_{y \in Y} f(x,y).
\end{equation}  Definimos $ \alpha := \inf_{y \in Y} \max_{x \in X} f(x,y) $. Primero vamos a reescribir el resultado a probar. La desigualdad (\ref{desAux1}) es equivalente a:\\
\[
\exists x_0 \in X:\text{ }\alpha \leq \inf_{y \in Y} f(x_0,y),
\]
ya que si existe un elemento en $ X $ que lo cumpla el máximo también lo cumplirá y recíprocamente. O lo que es lo mismo:
\[
\exists x_0 \in X:\text{ }y \in \topSpaceY \Longrightarrow \alpha \leq f(x_0,y).
\]
Entonces, tenemos que:
\[
\bigcap_{y\in \topSpaceY}\{x \in \topSpace: \alpha \leq f(x, y) \} \neq \emptyset,
\]
debido a que al menos $ x_0 $ está en dicha intersección. Como $ f $ es superiormente semicontinua en $ \topSpace $ estamos ante una intersección de cerrados. Usando la propiedad de intersección finita ($ \topSpace $ es compacto) obtenemos que:
\[
N\in\NN, \text{ }y_1,\dots,y_N \in \topSpaceY \Longrightarrow \bigcap_{i=1}^{N}\{x \in \topSpace: \alpha \leq f(x, y_i) \} \neq \emptyset.
\]
\[
\big\Updownarrow
\]
\[
N\in\NN, \text{ }y_1,\dots,y_N \in \topSpaceY \Longrightarrow \exists x_0\in\topSpace:\alpha \leq \min_{i=1\dots,N }f(x_0,y_i).
\]
\[
\big\Updownarrow
\]
\begin{equation}\label{desAux}
N\in\NN, \text{ }y_1,\dots,y_N \in \topSpaceY \Longrightarrow \alpha \leq \max_{x \in X} \min_{i=1\dots,N }f(x,y_i).
\end{equation}
En efecto, sean \vecN{y}$ \in \topSpaceY$ con $ N \in \NN $. Aplicamos el lema de Simons, lema (\ref{Simons}), tomando $ C := \topSpace $ y $ f_i: \topSpace \longrightarrow \RR $ definidas como $ f_i(x):=-f(x,y_i) $ para $ i=1,\dots,N $. Como $ f $ es cóncava respecto a $ X $ tenemos que las $ f_i $ son convexas para X con $  i=1,\dots,N $. De este modo, existe $ \ttt \in \Delta_N$ tal que
\[
\inf_{x \in \topSpace}\left[ \max_{i=1\dots,N } \{f_i(x)\}\right] = \inf_{x \in \topSpace} \left[ \sum_{i=1}^{N} t_i f_i(x) \right].
\] 
Si ponemos la igualdad en función de $ f $ y recordando que alcanza el supremo en $ X $:
\[
\inf_{x \in \topSpace}\left[ \max_{i=1\dots,N } \{-f(x,y_i)\}\right] = \inf_{x \in \topSpace} \left[ \sum_{i=1}^{N} t_i (-f(x,y_i)) \right]. 
\] 
\[
\big\Downarrow
\]
\[
\max_{x \in \topSpace}\left[ \min_{i=1\dots,N } \{f(x,y_i)\}\right] = \max_{x \in \topSpace} \left[ \sum_{i=1}^{N} t_i f(x,y_i) \right]. 
\] 
Al ser $ f $ convexa en $ \topSpaceY $:
\[
\max_{x \in \topSpace}\left[ \min_{i=1\dots,N } \{f(x,y_i)\}\right] \geq \max_{x \in \topSpace} \left[ f(x,\sum_{i=1}^{N} t_i y_i) \right] \geq \inf_{y \in Y} \max_{x \in \topSpace} f(x,y)= \alpha.
\]
Hemos probado entonces la desigualdad (\ref{desAux}) y, al ser la otra desigualdad sabida, podemos concluir que
\[
\max_{x \in X} \inf_{y \in Y} f(x,y) = \inf_{y \in Y} \max_{x \in X} f(x,y).
\]
\end{proof}
