\chapter{Resumen extendido y palabras clave en inglés}
The end of degree project that we present is an incursion into a field included in optimization, the theorems of the alternative, and their applications, mainly to the own optimization and finances.\\

Many of the theorems of the alternative are just reformulations of convex separation's theorems in certain contexts, so we start this memory with a study on Hahn-Banach's theorem. There are several equivalent versions of this result, collected for example in \cite{schechter1996handbook}. That is the reason why many authors talk about Hahn-Banach's theorems. We will focus on one of them, Mazur-Orlicz-König's theorem. Despite its geometrical charge, tt is mainly an algebraical result. We will provide a proof of that theorem -- the results in this work are self-contained --  and we will use it to give a convex version  of Gordan's theorem of alernative. Actually, it is an equivalent result due to S. Simons. This version is more general than the orginal Gordan's result and it will be extremmely useful in the next important results. \\

Then, we will apply the convex Gordan's theorem of the alternative in the minimax theory. A minimax inequality guarantees, under certain hypothesis, that in a two variables function we can substitute inf\hspace{0.5mm}sup for sup\hspace{0.5mm}inf. The power of minimax inequalities is clear when we use it -- in one that we will deduce from the convex Gordan's theorem -- to give classical convex separation's results. \\

Next, we will deduce from one separation's theorem the Farkas's lemma, which is one of the most known theorem of the alternative. This will allow us to prove, almost immediately, one of the key points of linear programming: the duality theorem. Considering more general optimization theorems, those in which the objective function and the inequalities constraints are differentiables, we wil establish, using Gordan's theorem of the alternative, the theorems of Fritz John and Karush-Kuhn-Tucker.\\

We will conclude with a little foray into the field of financial mathematics. After introducing some of the main concepts, like a derivate security; we will prove, as a consequence of one of the convex separation's theorem, the result kwonw as ``First Fundamental Theorem of Asset Pricing''. It can be used in the pricing of European's options in the binomial model. Finally, we have programmed using \textit{Sagemath 2.8} some examples in order to know how their price varies according to its parametres.\\

% Keywords command
\providecommand{\keywords}[1]
{
	\small	
	\textbf{\textit{Keywords: }} #1
}
\keywords{theorems of the alternative, Hahn-Banach's theorem, minimax, optimization, financial mathematics.}