\chapter{Igualdad ???}
\newcommand{\normSpace}{E}
	
Ahora, queremos obtener una igualdad que nos será de utilidad posteriormente para el Teorema de Separación. Empezamos recordando algunas nociones básicas de Análisis Funcional. El primer resultado que caracteriza la continuidad de operadores lineales en espacios normados.
	
\begin{proposicionBox}\label{caracCont}
Sean $ \normSpace_1, \normSpace_2 $ dos espacios normados y $ L:\normSpace_1 \longrightarrow \normSpace_2 $ un operador lineal. Entonces, L es continuo si, y solo si, verifica la siguiente condición:
\[
\exists \alpha > 0 : \norm{L(x)} \leq \alpha \norm{x} \quad \forall x \in \normSpace_1
\]
\end{proposicionBox}
\begin{proof}
Destacar que notamos de igual manera a la norma de $ \normSpace_1 \text{ y } \normSpace_2$ lo que no debería causar confusión.\\

En primer lugar, supongamos que $ L $ es continuo. Como es continuo en todo punto de $ \normSpace_1 $, en particular, lo es en 0. De este modo, para $ \varepsilon = 1 $, existe $ \delta >0  $  tal que, para $ y \in \normSpace_1 $ con $ \norm{y} \leq \delta$ se tiene $ \norm{L(y)} \leq 1 $. Dado $ x \in \normSpace_1 \setminus \{0\} $, podemos tomar $ y = \delta x / \norm{x} $, para obtener $ \delta \norm{L(x)} / \norm{x} = \norm{L(y)} \leq 1 $. Esto es $ \norm{L(x)} \leq (1/\delta) \norm{x} $. Esto es válido para todo $ x \in \normSpace_1 \setminus \{0\} $ y obvio para $ x = 0 $ por lo que podemos tomar $ \alpha = 1/\delta $. \\

Por otro lado si $ \exists \alpha > 0 $ tal que  $ \norm{L(x)} \leq \alpha \norm{x} $ para todo $ x \in \normSpace_1 $, para cualesquiera $ x,y \in \normSpace_1 $ se tiene que $ \norm{L(x) - L(y)} \leq \alpha \norm{x-y} $. Así, $ L $ es lipschitziana de $ \normSpace_1 \text{ a } \normSpace_2$ por lo que es continua (en realidad uniformemente continua).
\end{proof}

Consideramos el espacio dado por:
\[
\normSpace^* =  \lbrace L:\normSpace \longrightarrow \RR: L \text{ es lineal y continuo} \rbrace.
\]
Para todo $ L \in \normSpace^* $ definimos su norma como su constante de Lipschitz, es decir:
\[
\norm{L} = \min \lbrace \alpha > 0:  \norm{L(x)} \leq \alpha \norm{x} \quad \forall x \in \normSpace_1\rbrace.
\]
De este modo, podemos escribir:
\begin{equation}\label{desigNorma}
	\norm{L(x)} \leq \norm{L} \norm{x}
\end{equation}
siendo dicha desigualdad óptima. También podemos expresar su norma como el mínimo mayorante de un conjunto mayorado, que es el supremo:
\[
\norm{L} = \sup \lbrace \norm{L(x)}/\norm{x} : \forall x \in \normSpace_1 \setminus \{0\}\rbrace.
\]
Para $  x \in \normSpace_1 \setminus \{0\} $ tenemos que $ \norm{L(x)}/\norm{x} = \norm{L(x/\norm{x})}$ y es claro que $ \{ x/\norm{x} : x \in \normSpace_1 \setminus \{0\}\} $ es la esfera unidad de $ \normSpace $ que notamos como $ S_\normSpace $. Si en vez de la esfera consideramos la bola unidad, $ B_\normSpace $ el supremo no varía. Efectivamente, si $ x \in B_\normSpace  $ se tiene que $ x = \norm{x}u $ con $ u \in  S_\normSpace$, y por ello $ \norm{L(x)} = \norm{x}\norm{L(u)} \leq \norm{L(u)} $ ya que $ \norm{x} \leq 1 $. De este modo, también tenemos que:
\begin{equation}\label{normSup}
	\norm{L} =\sup_{x \in B_\normSpace} \norm{L(x)}.
\end{equation}

En este momento, estamos en disposición de enunciar y demostrar la igualdad que deseamos:
\begin{proposicionBox}
	Dado un espacio normado $ \normSpace $ y $ x \in \normSpace $, entonces se cumple que:
	\begin{equation}\label{iguNorm}
	\sup_{x^* \in B_{\normSpace ^ *}} x^*(x) = \norm{x}.
	\end{equation}
\end{proposicionBox}
\begin{proof}
Consideramos el funcional 
\begin{equation*}
\begin{split}
P:\normSpace \longrightarrow &\RR \\
x \longmapsto &\norm{x}
\end{split}
\end{equation*} 
y el conjunto $ D = \{x_0\} $ con $ x_0 \in \normSpace $ arbitrario. Es claro que $ P $ es sublineal al estar definido como la norma en $ \normSpace $ y que $ D $ es convexo. Podemos aplicar el Teorema de Mazur-Orlicz, teorema \ref{M-O}, y obtenemos que existe un funcional $ L:\normSpace \longrightarrow \RR $ lineal tal que $ L \leq P $ e $ \inf_D L = \inf_D P $. Como $ L \leq P $, entonces 
\begin{equation*}
 	\norm{L(x)} \leq \norm{P(x)} = \norm{x} \quad \forall x \in \normSpace.
\end{equation*}
Usando la proposición \ref{caracCont} concluimos que $ L $ es continua. Como $ L \in \normSpace ^* $, llamamos $ L = x^* $ y llegamos a que $ \norm{L} = \norm{x^*}  \leq 1$ ($ x^* \in  B_{\normSpace ^ *} $).  Por su parte, como  $ \inf_D L = \inf_D x^* = \inf_D P $ y $ D = \{x_0\} $ entonces, $ x^*(x_0) = \norm{x_0} $. De este modo, llegamos a que existe $ x^* \in B_{\normSpace^*} $ tal que $ x^*(x_0) = \norm{x_0} $. Si $ y^* \in B_{\normSpace^*}  $, entonces $ \norm{y^*(x_0)} \leq \norm{y^*} \norm{x_0} \leq \norm{x_0} $ y podemos asegurar que:
\[
\sup_{x^* \in B_{\normSpace ^ *}} x^*(x_0) = \norm{x_0}.
\] 
Como $ x_0 \in \normSpace $ es arbitrario concluimos que dado $ x \in \normSpace $ se cumple que:
\begin{equation*}
\sup_{x^* \in B_{\normSpace ^ *}} x^*(x) = \norm{x}.
\end{equation*}
\end{proof}