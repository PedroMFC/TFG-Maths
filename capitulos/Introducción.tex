\chapter{Introducción}
El estudio de los teoremas de la alternativa hunde sus raíces en el teorema de separación de convexos de Hahn-Banach. Es más, algunas de sus versiones más conocidas, como el teorema de la alternativa de Gordan o el lema de Farkas, son precursoras de ese resultado fundamental del análisis funcional o la optimización. Y es precisamente ahí, en un contexto de optimización, donde surgen hace casi siglo y medio. Desde entonces han aparecido en gran cantidad, vinculados a problemas de optimización convexa en muchas ocasiones y mediante el uso de técnicas de separación. Además, su aplicabilidad no se ha circunscrito exclusivamente al campo de la optimización sino que ha transcendido dicha área: análisis convexo, análisis funcinal, problemas de equilibrio $ \dots $ \\

En esta memoria se aborda el estudio de los dos resultados de la alternativa mencionados, el teorema de Gordan y el lema de Farkas, dando incluso una versión más general del primero. Para establecerlos usamos una versión muy simple del teorema de Hahn-Banach y una mejora del mismo. Además, se aplican para establecer una desigualdad minimax que deriva, en particular, en una serie de teoremas de separación de convexos. También se ilustra su aplicabilidad en establecer resultados centrales en la optimización, como es el teorema de dualidad en programación lineal o los teoremas de Fritz John y Karush-Kuhn-Tucker en un contexto diferenciable. Finalmente, dedicamos todo un capítulo de la memoria a usar los teoremas de la alternativa, en una de sus formas equivalentes, para demostrar un resultado importante en el ámbito de las matemáticas financieras, el primer teorema fundamental de valoración de activos financieros. Ello requiere un bagaje previo -- conceptos y resultados -- que también se recoge en la memoria. Dicho teorema se aplica al caso de ciertos derivados muy populares, las opciones europeas, bajo un modelo binomial y se presentan algunas simulaciones numéricas realizadas con \textit{SageMath} en su versión 2.8. \\

Centrándonos en la parte informática, el trabajo que se expone a continuación puede considerarse como una introducción al mundo del digitalizado 3D. Es una técnica que se engloba dentro del campo de la Informática Gráfica y que se aplica a una gran cantidad de sectores: diseño industrial, arte, cine, medicina, etc. Por ejemplo, en el mundo artístico se emplea para la obtención de un modelo digital de la obra con el fin de tener una copia digital de la misma u obtener más información que puede ser de utilidad, como en los procesos de restauración, sin necesidad de tener el original. También podemos poner un ejemplo de su uso en diseño industrial donde el modelo obtenido se puede utilizar para realizar simulaciones con el objetivo de detectar las partes más susceptibles a sufrir fallos. \\

La obtención de un modelo detallado y completo es un proceso complicado y ampliamente estudiado. En primer lugar, es importante el escáner que utilicemos para realizar dicho proceso. En la actualidad existen una gran cantidad de modelos que utilizan diferentes tecnologías para obtener cada una de las tomas necesarias aunque la mayoría de ellos suelen darnos como resultado una nube de puntos que define la muestra. Otras técnicas como la fotogrametría también tienen como objetivo la obtención de un modelo 3D pero a partir de fotografías. Nosotros estudiaremos mecanismos basados en nubes de puntos. En cuanto a las técnicas utilizadas para la obtención estas nubes de puntos hay dos tipos: por contacto o sin contacto. La más utilizada es la segunda de ellas que a su vez puede utilizar una tecnología láser o mediante luz estructurada. Entre otros aspectos a tener en cuenta encontramos la resolución del escáner, su precisión, las propiedades del objeto, etc. No es el objetivo de este trabajo ver las tecnologías disponibles en el digitalizado ni el propio proceso y sus consecuencias en el resultado final obtenido pero es importante conocer las alternativas disponibles. Sí hay que tener en cuenta que algunos escáneres nos pueden aportar información adicional acerca de los puntos lo que puede ser de utilidad posteriormente. Una vez tenemos las nubes de punto, podemos distinguir tres etapas hasta llegar al modelo digital:
\begin{itemize}
	\item Alinear: cuando se toman las muestras del objeto a digitalizar, es necesario mover el escáner con el que se realizan las medidas o el propio objeto. Esto conlleva que diferentes tomas van a estar en diferentes sistemas de de coordenadas. Esto se podría solucionar, por ejemplo, mediante un sistema GPS que guardara las posiciones para posteriormente realizar las transformaciones correspondientes en las muestras. El error de GPS puede ser del orden de varios metros y no funciona bien en interiores. Por otra parte hay otro tipo de mecanismos que puede merecer la pena comentar: utilizar dianas en la escena; sensores inerciales en el escáner. En cualquier caso estos mecanismos no están siempre disponibles. Por ello, esta etapa, tiene como objetivo que todas las tomas estén respecto al mismo sistema de coordenadas. Otro aspecto a destacar es si 
	estas tomas son rígidas o no, es decir, si no hay deformaciones en las mismas. Nosotros en este trabajo nos centraremos en el caso rígido. 
	
	\item Fusionar: finalizada la etapa anterior debemos hacer que todas las nubes de puntos que tenemos alineadas se transformen en una sola. Más concretamente, durante la obtención de las muestras es necesario que estas se solapen ,como veremos posteriormente, lo que hace que haya zonas con gran cantidad de puntos, muchos de ellos ``repetidos''. De este modo, debemos detectar dichos puntos para que la densidad en todo el modelo sea uniforme. También hay que tener en cuenta que es posible que haya zonas que por falta de previsión o por imposibilidad física no se hayan podido obtener una muestra. En esta etapa también se deberían arreglar dichas faltas de información. 
	\item Triangular: hasta ahora hemos trabajado con nubes de puntos pero el modelo es una superficie. Con esta etapa final se pretende obtener los triángulos que definen dicha superficie y así completar el proceso de digitalizado. Una alternativa sería obtener una función implícita que defina la superficie.
\end{itemize}

Las etapas que acabamos de mencionar no deben darse obligatoriamente en ese orden. Podríamos empezar calculando los triángulos que define cada nube de puntos y posteriormente alinear y fusionar. Sin embargo, el orden descrito es el más habitual. Durante el proceso también se suele tener en cuenta la densidad de puntos, la posible existencia de ruido, de valores atípicos o del error que tiene el propio escáner durante la toma de medidas aunque no serán aspectos a los que nosotros les demos demasiada importancia.  \\

Una idea importante a destacar es que en el digitalizado 3D no hay un algoritmo que sea capaz de reconstruir todo tipo de objetos. En este trabajo se estudiarán algunos algoritmos genéricos pero en la práctica existen métodos específicos para entornos urbanos, arquitectónicos, modelos curvilíneos, suaves o con aristas, etc. Más concretamente, nos centraremos en la primera de las etapas del proceso, el alineado. Dentro de este campo veremos dos de los algoritmos más usados: ICP y RANSAC. El primero de ellos es un algoritmo que se usa para cualquier modelo pero que veremos que tiene una gran desventaja: el tiempo de ejecución del mismo. Por ello, intentaremos disminuir todo lo posible este tiempo mediante el uso de descriptores para detectar puntos significativos del modelo y reducir de ese modo el conjunto de puntos a los que es necesario aplicar el algoritmo. Por su parte, el algoritmo RANSAC es más específico que el anterior y sirve para estimar parámetros de un modelo matemático. En nuestro caso, lo usaremos para la detección de planos que nos aportaran un número reducido de puntos clave a los que aplicar el proceso de alineado. Estos algoritmos se han programado dentro un banco de pruebas que nos permite realizar los algoritmos paso a paso con el fin de poder observar mejor el comportamiento de los mismos.\\

En definitiva, por un lado, en esta memoria se plasma tanto el carácter convexo-funcional de los teoremas de la alternativa como su aplicabilidad a campos tan diversos como la optimización, el análisis convexo y las matemáticas financieras. Por la parte informática, se ha realizado un análisis de dos procedimientos habituales para la alineación de dos nubes de puntos, proceso que se engloba dentro de la digitalización 3D. Señalamos finalmente que las referencias usadas en la elaboración de este memoria aparecen recogidas en el capítulo de Bibliografía. No obstante, los textos de \cite{borwein}, \cite{elliot1999mathematics} y \cite{Simons2008} han sido los esenciales para la parte de matemáticas financieras, mientras que \cite{QuatYan}, \cite{ICPBesl}, \cite{fischler1981random} y \cite{QT+Opengl} han sido de gran utilidad en el tema de digitalizado.
