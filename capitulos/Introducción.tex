\chapter{Introducción}
El estudio de los teoremas de la alternativa hunde sus raíces en el teorema de separación de convexos de Hahn-Banach. Es más, algunas de sus versiones más conocidas, como el teorema de la alternativa de Gordan o el lema de Farkas, son precursoras de ese resultado fundamental del análisis funcional o la optimización. Y es precisamente ahí, en un contexto de optimización, donde surgen hace casi siglo y medio. Desde entonces han aparecido en gran cantidad, vinculados a problemas de optimización convexa en muchas ocasiones y mediante el uso de técnicas de separación. Además, su aplicabilidad no se ha circunscrito exclusivamente al campo de la optimización sino que ha transcendido dicha área: análisis convexo, análisis funcinal, problemas de equilibrio $ \dots $ \\

En esta memoria se aborda el estudio de los dos resultados de la alternativa mencionados, el teorema de Gordan y el lema de Farkas, dando incluso una versión más general del primero. Para establecerlos usamos una versión muy simple del teorema de Hahn-Banach y una mejora del mismo. Además, se aplican para establecer una desigualdad minimax que deriva, en particular, en una serie de teoremas de separación de convexos. También se ilustra su aplicabilidad en establecer resultados centrales en la optimización, como es el teorema de dualidad en programación lineal o los teoremas de Fritz John y Karush-Kuhn-Tucker en un contexto diferenciable. Finalmente, dedicamos todo un capítulo de la memoria a usar los teoremas de la alternativa, en una de sus formas equivalentes, para demostrar un resultado importante en el ámbito de las matemáticas financieras, el primer teorema fundamental de valoración de activos financieros. Ello requiere un bagaje previo -- conceptos y resultados -- que también se recoge en la memoria. Dicho teorema se aplica al caso de ciertos derivados muy populares, las opciones europeas, bajo un modelo binomial y se presentan algunas simulaciones numéricas realizadas con \textit{SageMath} en su versión 2.8. \\

En definitiva, en esta memoria se plasma tanto el carácter convexo-funcional de los teoremas de la alternativa como su aplicabilidad a campos tan diversos como la optimización, el análisis convexo y las matemáticas financieras. Señalamos finalmente que las referencias usadas en la elaboración de este memoria aparecen recogidas en el capítulo de Bibliografía. No obstante, los textos de \cite{borwein}, \cite{elliot1999mathematics} y \cite{Simons2008} han sido los esenciales.