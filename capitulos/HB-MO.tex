\chapter*{Teorema de Hanh-Banach y de Mazur-Orlicz}
	\thispagestyle{empty}

	\paragraph{}El objetivo principal de esta sección es demostrar tanto una versión no tan conocida del clásico teorema de Hanh-Banach como el teorema de Mazur-Orlicz. Para ello, iremos utilizando una serie de lemas previos que nos facilitarán el proceso. Así, una vez vistos estos resultados, estaremos preparados para enunciar los conocidos como teoremas de la alternativa necesarios en el campo de la optimización convexa.
	
	\paragraph{} En primer lugar vamos a recordar la definición de funcional sublineal sobre un espacio vectorial \vecSpace . Notar que todos los espacios vectoriales que vamos a usar son reales. Del mismo modo, los subconjuntos que usaremos asumiremos que son no vacíos
	
	\begin{definicion}
		Sea \vecSpace un espacio vectorial distinto de cero. Decimos que el $P:\vecSpace \rightarrow \RR$ es sublineal si cumple las siguientes condiciones:
		\begin{itemize}
			\item $ P $ es subaditiva: $x_1, x_2 \in \vecSpace \Longrightarrow P(x_1 + x_2) \leq P(x_1) + P(x_2) $
			\item $ P $ es positivamente homogénea: $x_1 \in \vecSpace $ y $ \lambda > 0 \Longrightarrow P(\lambda x) = \lambda P(x) $
		\end{itemize}
	\end{definicion}

	\paragraph{}Por ejemplo, toda seminorma sobre \vecSpace es un funcional sublineal. También, si $ \vecSpace = \RR $ y definimos $ P(x) = m\acute{a}x \{0,x\}, \forall x \in \RR $ obtenemos un funcional sublineal sobre $\RR$.
	
	\paragraph{} El lema que exponemos a continuación, y que generalizaremos posteriormente en \ref{lema2}, nos servirá para demostrar el teorema de Hanh-Banach. 
	
	\begin{lemaBox}\label{lema1}
		Sea\vecSpace un espacio vectorial distinto de cero y $P:\vecSpace \rightarrow \RR$ un funcional sublineal. Fijamos un elemento $ y \in V $. Para todo $ x \in V $ tomamos  
		\begin{center}
			$ P_y(x) := \inf_{\lambda > 0} \left[P(x+\lambda y) - \lambda P(y)\right] $
		\end{center}
		
		Entonces, $ P_y:V \rightarrow \RR$, $ P_y $es sublineal, $ P_y \leq P $ y $ P_y (-y) \leq  -P(y)$.
	\end{lemaBox} 
	\begin{proof}
		Fijamos $ y \in V $. Sea $ x \in V $ y $ \lambda > 0$, como P es sublineal tenemos: 
		\begin{center}
			$ \lambda P(y) = P(\lambda y) =P(\lambda y +x-x) \leq P(x+\lambda y)+ P(-x)$
		\end{center}
		
		Por lo tanto, se obtiene que $ P(x+\lambda y) - \lambda P(y) \geq -P(-x) $.  Tomando el ínfimo sobre $ \lambda >0 $ llegamos a $ P_{y}(x)\geq -P(-x) > -\infty$. Por consiguiente, $ P_y:V \rightarrow \RR$. \\
		
		Probaremos ahora que $ P_y $ es sublineal. Empezamos viendo la subaditividad. Tomamos $ x_1, x_2 \in V $ y sean $ \lambda_1, \lambda_2 > 0$ arbitrarios. Entonces: 
		\begin{equation*}
		\begin{split}
		\left[ P(x_1 + \lambda_1 y) - \lambda_1 P(y) \right] &+ \left[ P(x_2 + \lambda_2 y) - \lambda_2 P(y) \right] \\
		& \geq \left[ P(x_1 + x_2 (\lambda_1+\lambda_2)y) \right] - (\lambda_1+\lambda_2) P(y) \\
		&\geq P_y (x_1 + x_2 )
		\end{split}
		\end{equation*}
		
		Tomando ínfimo sobre $ \lambda_1 $ y $ \lambda_2 $, $  P_y (x_1)  + P_y (x_2 ) \geq P_y (x_1 + x_2 ) $. Así, $ P_y $ es subaditiva. Para comprobar que es positivamente homogénea tomamos $ x \in V $ y $ \mu > 0 $. Entonces:
		\begin{equation*}
		\begin{split}
		P_y (\mu x) &= \inf_{\lambda > 0} \left[P(\mu x+\lambda y) - \lambda P(y)\right] = \mu \inf_{\lambda > 0} \left[P(u x+ (\lambda / \mu) y) - (\lambda / \mu) P(y)\right] \\
		&= \mu \inf_{\upsilon > 0} \left[P(u x+ \upsilon y) - \upsilon  P(y)\right] = \mu P_y (x)
		\end{split}
		\end{equation*}
		
		Obtenemos que $ P_y $ es positivamente homogénea y como consecuencia sublineal. \\
		
		Para demostrar que $ P_y \leq P $, sea $ x \in V $ y tomando $ \lambda = 1 $: 
		\[ P_y(x) \leq P(x+y) - P(y) \leq P(x)+ P(y) - P(y) = P(x) \Longrightarrow P_y \leq P  \]
		
		Finalmente, actuando de manera similar al caso anterior:
		\[ \qquad \qquad \qquad \qquad P_y(-y) \leq P(-y+y) - P(y) = -P(y)  \]
		
	\end{proof}
	
	\paragraph{} Ahora procedemos a probar el teorema de Hanh-Banch para funcionales sublineales, el cual es uno de los resultados más importantes del análisis funcional.
	
	\begin{teoremaBox}[Hanh-Banach]\label{H-B}
		Sea V un espacio vectorial distinto de cero y $P:V \rightarrow \RR$ un funcional sublineal. Entonces existe un funcional lineal L en V tal que $ L \leq P $.
	\end{teoremaBox}
	\begin{proof}
		Sea $ \mathcal{Q} $ el conjunto de funcionales sublineales $ Q $ en\vecSpace tales que $ Q \leq P $. Primero probaremos que todo subconjunto $ \mathcal{T} $ totalmente ordenado de $ \mathcal{Q} $ tiene una cota inferior en $ \mathcal{Q} $. Para $ T_1 ,T_2 \in \mathcal{Q} $ definimos:
		\begin{center}
			$ T_1 \leq T_2 \Longleftrightarrow T_1 (x) \leq T_2 (x) \quad \forall x \in \vecSpace $
		\end{center}
		
		obteniendo la relación de orden en  $ \mathcal{Q} $. Definimos $ Q(x):=\inf \{ T(x): T \in \mathcal{T} \} $. Si $ x \in \vecSpace $ y $ T \in \mathcal{T} $, como T es subaditiva obtenemos la siguiente relación $ 0 = T(0) = T(x-x) \leq T(x) + T(-x) \Longrightarrow T(x) \geq -T(-x) $ (1). Como $ T \in \mathcal{Q} \Longrightarrow T(x) \leq P(x) \Longrightarrow -T(x) \geq -P(x)$ (2). Usando (1), (2) y tomando ínfimo sobre $  T $  llegamos a $ Q(x) \geq -P(x) \geq - \infty $. Por lo tanto $ Q:V \rightarrow \RR$. \\
		
		Ahora probaremos que $ Q $ es subaditiva. Para ello, tomamos $ x_1, x_2\in \vecSpace $. Sean $ T_1 , T_2 \in \mathcal{T} $ arbitrarios. Si $ T_1 \geq T_2 $:
		
		\begin{center}
			$ T_1 (x_1)+  T_2 (x_2) \geq T_2(x_1)+  T_2 (x_2) \geq T_2(x_1 +x_2) \geq Q(x_1 + x_2)$
		\end{center}
		
		El caso de $ T_2 \geq T_1 $ es análogo. Concluimos que ambos casos $ T_1 (x_1)+  T_2 (x_2) \geq Q(x_1 + x_2)$. Tomando ínfimo en $ T_1 $ y $ T_2 $ obtenemos que $ Q (x_1)+  Q(x_2) \geq Q(x_1 + x_2)$. Así, $ Q $ es sublineal. Que sea positivamente homogénea es consecuencia de que $ T $ también lo es. Dado $ \mu > 0 $:
		\begin{equation*}
		\begin{split}
		Q(\mu x) &=\inf \{ T(\mu x): T \in \mathcal{T} \} \\ 
		& = \inf \{ \mu T( x): T \in \mathcal{T} \} \\ 
		&= \mu\inf \{ T( x): T \in \mathcal{T} \} \\ 
		&= \mu Q(x) 
		\end{split}
		\end{equation*}
		
		De este modo, Q es sublineal y como es claro que $ Q \leq P \Longrightarrow Q \in \mathcal{Q}$. Así, $ Q $ es el elemento minimal de $ \mathcal{T} $ en $\mathcal{Q}$.\\
		
		El lema de Zorn nos proporciona entonces un elemento minimal de $ \mathcal{Q} $ que llamaremos $ L $. Tomamos ahora $ y \in \vecSpace $. Con la notación del lema anterior, $ L_y : \vecSpace \longrightarrow \RR $ es sublineal, $ L_y \leq L $ (como consecuencia $ L_y \in \mathcal{Q} $) y $ L_y (-y) \leq L(-y) $. De hecho, como $ L $ es minimal en $ \mathcal{Q} $, $ L_y = L $ y por ello $ L (-y) \leq L(-y) $. Por otro lado, como L es subaditiva, $ L(-y) \geq -L(y) $. Combinando ambas desigualdades, $ L(-y) = -L(y) $. Tomamos $ x \in \vecSpace $ y $ \lambda < 0 $, usando la igualdad anterior llegamos a:
		\[ \qquad \quad
		L(\lambda x) = L (-(-\lambda)x) = -L(-\lambda x) = -(-\lambda)L(x) = \lambda L(x) \label{1}
		\] 
		
		obteniendo que $ L $ es homogénea. Si $ x_1, x_2 \in \vecSpace $, la subaditividad de $ L $ nos da $ L(-x_1-x_2) \leq L(-x_1) + L(-x_2) $. Usando la homogeneidad de $ L $:
		\begin{equation*}
		\begin{split} \qquad
		L(x_1+x_2) &= L(-(-x_1-x_2)) = -L(-x_1-x_2) \\ 
		& \geq -L(-x_1)-L(-x_2) = L(x_1) + L (x_2) \geq L(x_1+x_2) 
		\end{split}
		\end{equation*}
		
		Por ello, $	L(x_1+x_2) = L(x_1) + L (x_2) $ y concluimos que $ L $ es lineal.
		
	\end{proof}

	\paragraph{} El siguiente resultado importante que demostraremos será el teorema de Mazur-Orlicz. Primero, veamos un lema previo.
	
	\begin{lemaBox}\label{lema2}
		Sea\vecSpace un espacio vectorial distinto de cero y $P:\vecSpace \rightarrow \RR$ un funcional sublineal. Sea $ D $ un subconjunto no vacío y convexo de \vecSpace y $ \beta := \inf_D P \in \RR $. Para todo $ x \in V $ tomamos  
		\begin{center}
			$ Q(x) := \inf_{d \in D, \lambda > 0} \left[P(x+\lambda d) - \lambda \beta\right] $
		\end{center}
		
		Entonces, $ Q:V \rightarrow \RR$, $ Q $es sublineal, $ Q \leq P $ y $ \forall d \in D, -Q(-d) \geq \beta$.
	\end{lemaBox} 
	\begin{proof}
		Si $ x \in \vecSpace, \quad d \in D $ y $ \lambda > 0 $ entonces
		\begin{center}
			$ P(x+ \lambda d) - \lambda \beta \geq -P(-x) + \lambda P(d)-\lambda\beta \geq -P(-x) \geq -\infty$
		\end{center}
		
		La primera igualdad se deduce de la linealidad de P ya que:
		\[ \lambda P(d) = P(\lambda d) =P(\lambda d +x-x) \leq P(x+\lambda d)+ P(-x) \Longrightarrow -P(-x) \leq P(x+\lambda d)\]
		
		Y la segunda a que como $ \beta = \inf_D P \Longrightarrow \lambda P(d) \geq \lambda\beta \Longrightarrow\lambda P(d) - \lambda\beta \geq 0 $. Tomando el ínfimo sobre $ d \in D  $ y $ \lambda > 0 $ llegamos a $ Q(x)\geq -P(-x) > -\infty$. Por consiguiente, $ Q:V \rightarrow \RR$. Es relativamente fácil probar que $ Q $ es positivamente homogénea por lo que para ver que es sublineal solo queda ver la subaditividad. Para ello, tomamos $ x_1, x_2 \in V $. Sean $ d_1, d_2 \in D $ y $ \lambda_1, \lambda_2 > 0$ arbitrarios. Para simplificar la notación llamamos $ x := x_1 + x_2 $, $ \lambda := \lambda_1 + \lambda_2 $ y $ d:= (\lambda_1/\lambda)d_1 + (\lambda_2/\lambda)d_2 $. Notar que $ d \in D $ al ser este convexo. Entonces: 
		\begin{equation*}
		\begin{split}
		\left[ P(x_1 + \lambda_1 d_1) - \lambda_1 \beta \right] + \left[ P(x_2 + \lambda_2 d_2) - \lambda_2 \beta \right] &\geq P(x + \lambda_1 d_1 + \lambda_2 d_2) - \lambda \beta\\
		& = P(x +\lambda d) - \lambda \beta\\ 
		& \geq Q(x) = Q(x_1 + x_2)
		\end{split}
		\end{equation*}
		
		Tomando ínfimo sobre $ d_1, d_2, \lambda_1 $ y $ \lambda_2 $, $  Q(x_1) + Q (x_2 ) \geq Q (x_1 + x_2 ) $. Así, $ Q $ es subaditiva y como consecuencia sublineal. Fijamos $ d \in D $. Sea $ x \in \vecSpace $ arbitrario. Entonces, $ \forall \lambda > 0 $, $ Q(x) \leq P(x) + \lambda \left[P(d) - \beta \right]$. Tomando $ \lambda \longrightarrow 0 $, $ Q(x) \leq P(x)$ y como consecuencia $ Q \leq P $. Finalemente, sea $ d \in D $ arbitrario y tomando $ \lambda = 1 $:
		\[ \qquad \qquad \qquad Q(-d) \leq Q(-d+d) - \beta = -\beta \Longrightarrow  -Q(-d) \geq \beta  \]
		
	\end{proof}

	\paragraph{}Visto este lema, estamos preparados para ver el resultado que nos interesa:
	
	\begin{teoremaBox}[Mazur-Orlicz]
		Sea\vecSpace un espacio vectorial distinto de cero y $P:\vecSpace \rightarrow \RR$ un funcional sublineal.  Sea $ D $ un subconjunto no vacío y convexo de \vecSpace. Entonces existe un funcional lineal $ L $ sobre \vecSpace tal que $ L \leq P $ e $ \inf_D L = \inf_D P $
	\end{teoremaBox}
	\begin{proof}
		Sea $ \beta := \inf_D P $. En el caso de que $ \beta = -\infty $ por el teorema \ref{H-B} tenemos que $ \exists L $ sobre \vecSpace tal que es lineal y $ L \leq P$. Así:
		\begin{center}
			$ L \leq P \Longrightarrow inf_D L \leq \inf_D P = -\infty \Longrightarrow inf_D L = \inf_D P$ 
		\end{center}
		
		Supongamos entonces que $ \beta \in \RR $. Definimos el funcional auxiliar $ Q $ tal y como en el lema \ref{lema2}. Del teorema \ref{H-B} obtenemos que existe un funcional lineal $ L $ sobre \vecSpace tal que $ L \leq Q$ (como $ Q \leq P $ tenemos que $ L \leq P $). Sea $ d \in D $, entonces:
		\[
		L(d) = -L(-d) \geq -Q(-d) \geq \beta
		\]
		Tomando ínfimo sobre $ d \in D $:
		\[
		\inf_D L \geq \beta = \inf_D P
		\]
		Por otro lado, como $ L \geq P $:
		\[
		\inf_D L \leq\inf_D P
		\]
		Juntando ambas desigualdades obtenemos $ \inf_D L =\inf_D P $
	\end{proof}

