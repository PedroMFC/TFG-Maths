\chapter{Introducción}

En este trabajo fin de grado que se expone a continuación se realiza un recorrido por varios resultado importantes del Análisis Funcional o de la Optimización Convexa. Todo el desarrollo culminará con una aplicación en el mundo real de los mismos, más concretamente, en el mundo de las finanzas. Al final de este trabajo seremos capaces de saber el precio que tiene una opción de compra o venta que se explicarán en la sección correspondiente en la que también se verá que no es un problema trivial. Destacamos que en la actualidad existen una gran cantidad de métodos para realizar este proceso dependiendo, por ejemplo, del modelo matemático por el que se rige el mercado financiero. Nosotros nos centraremos en el modelo binomial para llegar al resultado deseado. Así, podríamos dividir nuestro trabajo en dos partes: la primera de ellas donde se realiza el desarrollo de todos los principales resultados matemáticos y una segunda donde se aplican a un mercado financiero.  \\

De este modo, el primer resultado importante que veremos es una versión del conocido Teorema de Hanh-Banach. Este resultado nos aporta el Teorema de Mazur-Orlicz que también es un resultado de análisis funcional y que garantiza la existencia de un funcional que satisface ciertas condiciones. Llegados a este punto, enunciaremos y demostraremos un resultado poco conocido que se conoce como Lema de Simons. Gracias a él seremos capaces de iniciar el estudio de los denominados teoremas de la alternativa entre los que destacamos el propuesto por Gordan. Además, veremos la equivalencia entre este resultado y el Lema de Simons. \\

Con el Teorema de la Alternativa de Gordan tomamos dos caminos diferentes. Por un lado, nos permite iniciar nuestro camino en el campo de la optimización convexa llegando a enunciar y probar los teoremas de Fritz John y Karush-Kuhn-Tucker. Por el otro, continuamos obteniendo los resultados necesarios para llegar a valorar opciones mediante un teorema minimax y posteriormente un teorema de separación. Veremos que este camino se puede recorrer en sentido contrario ya que el teorema de separación implica a su vez el Teorema de la Alternativa de Gordan. Es más, podemos llegar a enunciar otros teoremas de la alternativa parecidos como es el caso del Lema de Farkas. Este resultado se podría haber usado también para probar los teoremas del campo de la optimización convexa que hemos expuesto. Nosotros, sin embargo, lo usaremos para demostrar el teorema de dualidad en Programación Lineal. \\

Una vez realizado este trabajo, esteremos en  condiciones de iniciar nuestra andadura en el mundo las finanzas. Primero, haremos una introducción sobre conceptos financieros y posteriormente modelaremos matemáticamente nuestro mercado. Una vez hecho esto y gracias a unos resultados auxiliares, llegaremos al teorema de asignación de precios que nos conducirá a la fórmula culmen del proyecto para valorar opciones. Finalmente, se realizarán una serie de simulaciones para ver cómo varía el precio en función de sus parámetros.

