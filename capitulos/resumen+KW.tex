\chapter{Resumen y palabras clave}

En el ámbito de las matemáticas, el trabajo fin de grado que presentamos supone la incursión a un campo dentro de la optimización, los teoremas de la alternativa, y sus aplicaciones, principalmente a la propia optimización y finanzas. El trabajo comienza con el teorema de Mazur-Orlicz-König, versión del conocido teorema de Hahn-Banach. Posteriormente, estudiamos el teorema de la alternativa de Gordan, esencial para el resto de resultados. Su primera aplicación se da en la teoría minimax, que nos conduce a resultados clásicos sobre separación de convexos. A continuación, deduciremos otro teorema de la alternativa, el de Farkas, y lo aplicaremos a la programación lineal. Para demostrar los resultados sobre optimización, volvemos al teorema de Gordan, que nos proporcionara los teoremas de Fritz John y Karush-Kuhn-Tucker. Finalmente, nos introducimos en el mundo de las matemáticas financieras obteniendo, gracias al teorema de separación, el primer teorema de asignación de precios para valorar opciones europeas. Concluimos realizando simulaciones de su valor en diferentes casos.\\

Por su parte, dentro del campo de la informática, se ha realizado un estudio acerca del procesamiento de nubes de puntos. En general, se tienen varias nubes que corresponden a un mismo objeto pero que están tomadas desde distintos puntos de referencia. Esto hace necesario un tratamiento de los conjuntos para conseguir que se encuentren en el mismo sistema de coordenadas y tener así un modelo digital del objeto. Este problema se denomina alineado y para resolverlo se aborda el estudio de dos algoritmos diferentes: ICP y RANSAC. El primero de ellos es un algoritmo genérico que veremos que tiene una gran desventaja, el tiempo de ejecución del mismo. Por ello, intentaremos disminuir todo lo posible este tiempo mediante el uso de descriptores para detectar puntos significativos del modelo y reducir, de ese modo, el conjunto de puntos a los que es necesario aplicar el algoritmo. Por su parte, el algoritmo RANSAC es genérico y sirve para estimar parámetros de un modelo matemático. En nuestro caso, lo usamos para la detección de planos que nos aportaran un número reducido de puntos clave a los que aplicar el proceso de alineado. \\

% Keywords command
\providecommand{\pclave}[1]
{
	\small	
	\textbf{\textit{Palabras clave: }} #1
}
\pclave{teoremas de la alternativa, teorema de Hahn-Banach, minimax, optimización, matemáticas financieras, alineado, cuaternios, ICP, RANSAC.}