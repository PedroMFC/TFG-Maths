\chapter{Resumen y palabras clave}

El trabajo fin de grado que presentamos supone la incursión a un campo dentro de la optimización, los teoremas de la alternativa, y sus aplicaciones, principalmente a la propia optimización y finanzas. El trabajo comienza con el teorema de Mazur-Orlicz-König, versión del conocido teorema de Hahn-Banach. Posteriormente, estudiaremos el teorema de la alternativa de Gordan, esencial para el resto de resultados. Su primera aplicación será en la teoría minimax, que nos conducirá a resultados clásicos sobre separación de convexos. A continuación, deduciremos otro teorema de la alternativa, el de Farkas, y lo aplicaremos a la programación lineal. Para demostrar los resultados sobre optimización, volvemos al teorema de Gordan, que nos proporcionará los teoremas de Fritz John y Karush-Kuhn-Tucker. Finalmente, nos introducimos en el mundo de las matemáticas financieras obteniendo , gracias al teorema de separación, el primer teorema de asignación de precios para valorar opciones europeas. Concluimos realizando simulaciones de su valor en diferentes casos.\\

% Keywords command
\providecommand{\pclave}[1]
{
	\small	
	\textbf{\textit{Palabras clave: }} #1
}
\pclave{teoremas de la alternativa, teorema de Hahn-Banach, minimax, optimización, matemáticas financieras.}