\chapter{Conclusiones y vías futuras}
Tanto en el ámbito matemático como informático, los objetivos que nos marcamos en la propuesta inicial se han alcanzado satisfactoriamente. En el primero de ellos, incluso se ha realizado una incursión no prevista en la teoría minimax de manos de los teoremas de la alternativa. Ello ha permitido obtener una visión completa de las técnicas y aplicaciones de los teoremas de la alternativa. \\

En el caso del digitalizado 3D, se ha realizado una sólida introducción con los procedimientos básicos existentes. El estudio hecho en este trabajo se podría completar posteriormente con varias vías debido a lo amplitud del tema. Sería posible plantear otra serie de mejoras a los algoritmos como por ejemplo, el uso de \textit{voxels} para tener una mejor división espacial de la nube de puntos y acelerar el proceso del cálculo del punto más cercano. Otro posible camino a seguir sería estudiar algoritmos para la resolución de las otras dos etapas del proceso: fusión y triangulación. Finalmente, también se podría mejorar el banco de pruebas con la finalidad de que sea un \textit{software} funcional para el público en general.