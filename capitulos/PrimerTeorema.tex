\section{Primer teorema fundamental}
Una vez explicada de manera simplificada el contexto financiero debemos formalizar matemáticamente el modelo del mercado. Para ello, seguimos \cite{elliot1999mathematics}. Fijamos el conjunto de tiempos $ \mathbb{T} = \{0,1,\dots,T\}$ con $ T $ el instante en el que finaliza el modelado de nuestra actividad económica. También fijamos el espacio de probabilidad $ (\Omega, \mathcal{F}, P) $. Este espacio contiene todos los posibles estados del mercado. La información de la que disponen los inversores acerca de la estructura en cada momento $ t \in \mathbb{T}$ viene dada por una sucesión finita y creciente de sub-$ \sigma $-álgebras de $ \mathcal{F} $, $ \mathcal{F}_0 \subset \cdots \subset \mathcal{F}_T = \mathcal{F} $ con $ \mathcal{F}_0 $ trivial, es decir, contiene solamente conjuntos con medida 0 o 1. A este secuencia se le denomina filtración y se denota como $ \mathbb{F} = (\mathcal{F}_t)_{t \in \mathbb{T}} $. El sentido financiero de esta filtración es indicar la información conocida hasta el momento acerca de la evolución de los precios. Por ejemplo, si comenzamos a modelar nuestro mercado hoy, $ \mathcal{F}_0 $ nos aporta únicamente el valor actual. Sin embargo, mañana tendremos la información de los precios en ese momento junto con la obtenida hoy, por lo que es claro que $ \mathcal{F}_0 \subset \mathcal{F}_1  $. También es claro que al llegar al instante final tenemos toda la información acerca del modelo. Llamamos $ l  $ a la dimensión de nuestro mercado, es decir, el número de activos que manejamos. La evolución de los precios de los activos viene dada  por el proceso estocástico $ S = \{S^i_t:\hspace{1mm} t \in \mathbb{T},\hspace{1mm} i=0,\dots,l\} $ donde cada $ S^i_t $ es una variable aleatoria. Todos son activos con riesgos menos el marcado por $ 0 $. Asumimos que el proceso $ S_t^i $ es $ \mathcal{F}_t $-medible para $ i=0,\dots,l $, es decir, se adapta a la filtración $ \mathbb{F} $. Eso significa que para cada $ t \in \mathbb{T} $ se conocen los precios de cada activo hasta ese instante como se ha explicado anteriormente. En general, primero se escoge $ S $ y se toma $ \mathbb{F} $ como la filtración que genera. La tupla $ (\Omega, \mathcal{F}, P, \mathbb{F}, S) $ es la que modela nuestro mercado de valores. \\

El poder adquisitivo de una misma cantidad decrece con el tiempo, consideramos el factor de descuento (o de actualización) en el instante $ t $ dado por $ \beta_t < 1 $ y notamos al valor actual como $ \bar{S}_t = \beta_t S_t $. Sin pérdida de generalidad se puede asumir que $ S^0_0 = 1 $ por lo que podemos expresar todas las unidades en función a dicho valor. En ese caso, el factor $ \beta_t = 1/S^0_t $ es la cantidad de dinero que necesitamos para invertir en bonos en el instante 0 para tener una unidad en el instante $ t $. Podríamos considerar que tomar $ S^0_0 = 1 $  es una simple normalización de los precios. El factor de descuento se puede ver como $ \beta_t = (1+r)^{-t} $. Para entender esta fórmula podemos ver la situación contraria: calcular una cantidad con cierto intereses. De este modo, $ r $ representa el interés considerado o cómo aumentará el valor del dinero en el futuro. Así, el valor de una cantidad $ S $ en el futuro es $ (1+r)S $ considerando solo un paso o $ (1+r)^t S $ si consideramos varios. Por eso, al actualizar pasamos dicho término dividiendo y llegamos a la fórmula de $ \beta_t $ expuesta. Observemos que en ocasiones, como en el caso continuo, el papel de $ (1+r)^t $ es una función exponencial.\\

La cartera de inversión o portafolio para el instante $ t \in \mathbb{T} $ viene dada por la variable aleatoria $ l+1 $ dimensional $ \theta_t = (\theta_t^0,\dots,\theta_t^l) $. Cada $ \theta_t^i $ indica el número de activos del tipo $ i $ que tiene un inversor en el instante $ t $. El valor de la cartera viene determinado por $ V_t(\theta) $ donde
\[
V_0(\theta) = \langle \theta_1 , S_0\rangle, \hspace{2.5mm} V_t(\theta) = \langle \theta_t , S_t\rangle = \sum_{i=0}^{l} \theta_t^i S_t^i \text{ para } \hspace{0.5mm} t \in \mathbb{T}, t \geq 1.
\]
Cada inversor selecciona la cartera de inversión del instante $ t $ una vez se conocen los precios del momento anterior $ t-1 $. La estrategia de inversión $ \theta = \{\theta_t: \hspace{0.5mm}t=1,\dots,T\} $ es el conjunto de todas las carteras de inversión \textit{predecibles}. Decimos que una cartera $ \theta_{t} $ es predecible si depende solamente de los precios de los activos hasta el instante $ t-1 $. Cuando cada cartera de inversión $ \theta_{t+1} $ se puede financiar completamente con las ganancias o pérdidas actuales decimos que la estrategia es \textit{autofinanciada}, es decir, 
\begin{equation*}\label{selfinance}
\langle \theta_{t+1}, S_t \rangle = \langle \theta_{t}, S_t \rangle,
\end{equation*} Notamos $ \Delta X_t = X_t - X_{t-1} $ para cualquier función sobre $ \mathbb{T} $. Si una cartera es autofinanciada, escribimos la ganancia de una cartera de inversión entre los instantes $ (t-1,t] $ como 
\[
\Delta V_t(\theta) = \langle\theta_t, S_t \rangle - \langle\theta_{t-1}, S_{t-1}\rangle = \langle\theta_t, S_t \rangle - \langle\theta_{t}, S_{t-1}\rangle = \langle\theta_{t}, \Delta S_{t}\rangle.
\]
De este modo, la ganancia asociada a la estrategia $ \theta $ hasta el instante $ t $ viene dada por
\[
G_0(\theta) = 0, \hspace{2.5mm} G_t(\theta) = \langle\theta_{1}, \Delta S_{1}\rangle + \cdots + \langle\theta_{t}, \Delta S_{t}\rangle \text{ para } \hspace{0.5mm} t \in \mathbb{T}, t \geq 1.
\]
Vemos entonces que una estrategia es autofinanciada si, y solo si
\begin{equation}\label{selfinifonlyif}
V_t(\theta) = V_0(\theta) + G_t(\theta), \hspace{1mm} \forall t \in \mathbb{T}.
\end{equation}

Destacamos que todas la definiciones anteriores se pueden definir también en base a los valores actualizados $ \bar{S}_t $. \\

También es necesario imponer o aclarar una serie de suposiciones sobre nuestro modelo financiero:
\begin{enumerate}
	\item Ninguna de las transacciones conlleva un coste extra.
	\item Positividad: todos los precios son positivos, es decir,
	\[
	S_t^i > 0 \text{ para } t \in \mathbb{T},\hspace{1mm} i=0,\dots,l.
	\]
	\item Divisibilidad, liquidez y \textit{short selling} (venta en corto): el número de activos que posee un inversor puede ser cualquier valor real. Así,
	\[
	\theta_t^i \in \RR \text{ para } t \in \mathbb{T},\hspace{1mm} i=0,\dots,l.
	\]
	La divisibilidad hace referencia a que $ \theta_t^i $ puede ser una fracción. Claramente, no podemos tener, por ejemplo, media acción. Sin embargo, cuando estamos trabajando con un gran número de acciones podemos considerar que tienen cifras decimales para trabajar con números menores. El hecho de que tome valores en $ \RR $ también significa que podemos comprar o vender el número de acciones deseado, es decir, no imponemos ninguna restricción. Este hecho se conoce como liquidez. Claramente, en mercados reales sí existe dicha restricción en el volumen de las transacciones pero nosotros estamos trabajando con un modelo idealizado. Finalmente, cuando dichos valores son positivos decimos que el inversor tiene una posición larga o \textit{long position}. Si por el contrario estos son negativos, tiene una posición corta o \textit{short position}, por ejemplo, cuando vende algún tipo de bono. Estas acciones se suelen llevar a cabo, por ejemplo, en las denominadas ventas en corto o \textit{short selling} que se usan cuando se prevé que un determinado activo va a bajar de valor. Así, suponemos que tenemos una acción cuyo precio actual es 100 y suponemos que va a disminuir. Antes de que baje más, la vendemos, es decir, tomamos una posición \textit{short}. Pasado un periodo de tiempo, la acción bajan a 70 por lo que tomamos una posición \textit{long} y las compramos de nuevo. De este modo, tenemos las misma cantidad de activos que al principio pero hemos obtenido un beneficio de 30. En estas operaciones también corremos el riesgo de perder dinero ya que es posible que el valor de los activos crezca en vez de disminuir. 
	
	\item Solvencia: el valor de todas las carteras de inversión debe ser siempre no negativo,
	\[
	V_t (\theta) \geq 0\text{ para todo } t \in \mathbb{T}.
	\]
	En este caso, decimos que la cartera es \textit{admisible}. A la clase de estrategias autofinanciadas y admisibles la denotaremos como $ \Theta_a $.
	\item El espacio $ \Omega $ es finito, es decir, las variables aleatorias $ S_t^i $ no pueden tomar infinitos valores. Así, tenemos que $ \Omega = \{ \omega_1,\dots,\omega_n\} $.
\end{enumerate}

Ya tenemos el modelo financiero sobre el que vamos a trabajar. Exponemos ahora unos resultados que nos servirán posteriormente para probar el teorema principal del capítulo.
\bigskip
%% Parecido Prop 4.1 del otro libro Mathematics for finance
%% LEMMA 2.2.1
\begin{lemaBox}\label{2.2.1}
	Dada $ V_0 $ una función $ \mathcal{F}_0 $-medible y para $ l \in \NN  $ sean los procesos reales y predecibles $ \theta^1,\dots,\theta^l $, el único proceso predecible $ \theta^0 $ que convierte a $ \theta = (\theta^0,\theta^1,\dots,\theta^l) $ en una estrategia autofinanciada con valor inicial \nolinebreak$ V_0 (\theta)=$\nolinebreak$ V_0$  viene dado por
	\[
	\theta^0_t = V_0 + \sum_{i=1}^{t-1}(\theta^1_i\Delta\bar{S}^1_i+\cdots+\theta^l_i\Delta\bar{S}^l_i) - (\theta^1_t\bar{S}^1_{t-1}+\cdots+\theta^l_t\bar{S}^l_{t-1}).
	\]
\end{lemaBox}
\begin{proof}
	Claramente $ \theta^0 $ es predecible. Para ver que la estrategia es autofinanciada, por \eqref{selfinifonlyif} solo necesitamos ver que $ \theta^0_t $ es la única solución predecible de la ecuación 
	\begin{equation*}
	\begin{split}
	\bar{V}_t(\theta) &= \theta^0_t + \theta^1_t\bar{S}^1_t+\cdots+\theta^l_t\bar{S}^l_t \\
	&= V_0 + \sum_{i=1}^{t}(\theta^1_i\Delta\bar{S}^1_i+\cdots+\theta^l_i\Delta\bar{S}^l_i).
	\end{split}
	\end{equation*}
\end{proof}
\bigskip
Introducimos ahora el concepto de \textit{viabilidad}, esencial para lo que sigue. Decimos que un mercado es viable si para toda estrategia admisible y autofinanciada no contiene ninguna oportunidad de arbitraje. La ausencia de arbitraje significa que si el valor inicial de una cartera de inversión es $ V_0(\theta) = 0 $ entonces $ V_T(\theta) = 0 $ con probabilidad 1 para toda $ \theta \in\Theta_a $. La viabilidad del mercado impone la siguiente restricción.

\bigskip
%% LEMMA 3.2.1
\begin{lemaBox}\label{3.2.1}
	Si el modelo de mercado es viable, las ganancias actualizadas a cualquier proceso predecible $ \hat{\theta} \in \RR ^l $ no pueden pertenecer a 
	\[
	C = \{Y \in \RR^n: Y_i \geq 0\text{ para } i=1,\dots,n \text{ y } \exists i \text{ tal que } Y_i > 0\}.
	\]
\end{lemaBox}
\begin{proof}
	En primer lugar vemos que $ C $ es el octante positivo de $ \RR^n $ sin el origen, que claramente es un cono y es convexo. La ausencia de arbitraje significa que para toda estrategia admisible $ \theta \in \Theta_a $ tal que $ V_0(\theta) = 0 $ entonces
	\[
	\bar{V}_t(\theta) = \bar{G}_t (\theta) \notin C.
	\]
	Por el lema \ref{2.2.1}, dados los procesos predecibles $ \hat{\theta} = (\theta^1, \dots,\theta^l) $, existe un único proceso real $ \theta^0 $ tal que $ \theta = (\theta^0, \theta^1,\dots, \theta^l) $ es autofinanciada y $ V_0(\theta) = 0 $. Las ganancias con los valores actualizados viene dada por
	\[
	\bar{G}_t(\hat{\theta}) = \sum_{j=1}^{t} \langle \theta_j, \Delta \bar{S}_j \rangle =   \sum_{j=1}^{t} \left( \sum_{i=1}^{l} \theta_j^i \Delta \bar{S}_j^i  \right).
	\]
	Supongamos que $ \bar{G}_t(\hat{\theta}) \in C $; si $ \beta_T $ denota el factor de descuento en el instante $ T $,
	\[
	V_T(\theta) = \beta_T^{-1} \bar{V}_t(\theta) = \beta_T^{-1}(V_0 (\theta) + \bar{G}_t(\theta)) = \beta_T^{-1}\bar{G}_t(\hat{\theta}).
	\]
	Vemos entonces que $ V_T(\theta) $ es no negativa y estrictamente positiva con probabilidad no nula, lo que contradice la viabilidad al existir arbitraje.
\end{proof}
\bigskip
Nuestro objetivo es caracterizar la viabilidad de un mercado en términos de los incrementos de $ \bar{S} $. Para ello son necesarias las martingalas.
\bigskip
\begin{definicion}
	Un proceso $ \mathbb{F} $-adaptado $ M = (M_t)_{t\in \mathbb{T}} $ es una $ ( \mathbb{F},P)$-mar\-tingala si $ E(|M_t|) < \infty $ para todo $ t \in \mathbb{T} $ y 
	\[
	E(M_{t+1}|\mathcal{F}_t) = M_t \textit{ para todo } t \in \mathbb{T}\setminus{T}.
	\]
	Si $ M = (M_t) $ es una martingala y $ \phi = (\phi_t)_{t\in \mathbb{T}} $ es un proceso predecible en $ (\Omega, \mathcal{F}, P, \mathbb{F}, S) $ entonces al proceso $ X = \phi \cdot M $ dado por
	\[
	X_0=0, \hspace{2mm}X_t = \phi_1\Delta M_1+\cdots+ \phi_t\Delta M_t \hspace{1.5mm} t \geq1
	\]
	se le denomina martingala transformada de $ M $ por $ \phi $.
\end{definicion}

Notamos que $ M $ es una martingala si, y solo, si
\[
E(\Delta M_{t+1} |\mathcal{F}_t ) = 0 \text{ para todo } t\in \mathbb{T}\setminus\{T\}.
\]
También es importante destacar que, por la linealidad de la esperanza, cualquier combinación lineal de martingalas es una martingala. \\

Que los precios en el mercado sigan una martingala no debe de ser extraño. Recordemos que para $ t \in \mathbb{T} $, $ \mathcal{F}_t $ indica la cantidad de información acerca de los precios de los activos hasta dicho momento. Por lo tanto, la esperanza condicionada solo nos indica que estamos calculando el valor esperado a partir de lo que conocemos hasta ahora. Además, para que el mercado sea ``justo'', dicho valor en el futuro debería ser, en media, el que tenemos ahora. \\

El siguiente resultado es meramente técnico y nos servirá posteriormente. 

\bigskip
%% THEOREM 2.3.5
\begin{teoremaBox}\label{2.3.5}
	Un proceso real $ M $ es una martingala si, y solo si, 
	\[
	E((\phi \cdot M)_t) = E(\sum_{i=1}^{t}\phi_i\Delta M_i) = 0, \hspace{1mm} \forall t \in \mathbb{T}\setminus\{0\}
	\]
	para todo proceso $ \phi $ predecible y acotado.
\end{teoremaBox}
\begin{proof}
	Si $ M $ es un martingala, también lo es la tranfomada $ X = \phi \cdot M $ y $ X_0 =0 $ ya que
	\[
	E(\Delta X_{t+1}|\mathcal{F}_t) = E(\phi_{t+1}\Delta M_{t+1}|\mathcal{F}_t) =  \phi_{t+1} E(\Delta M_{t+1}|\mathcal{F}_t) = 0.
	\] 
	Por ello, $ E((\phi \cdot M)_t) = 0 $ para todo $ t \geq 1 $ en $ \mathbb{T} $.
	Demostremos ahora la otra implicación. Para $ s > 0 $, sea $ A \in \mathcal{F}_s $ y definimos el proceso predecible $ \phi $ como $ \phi_{s+1} = 1_A $ y $ \phi_t = 0 $ para el resto de $ t\in \mathbb{T} $. Entonces, para $ t > s $ se tiene que
	\[
	0 = E((\phi \cdot M)_t) = E(1_A(M_{s+1}-M_s)).
	\]
	Como es cierto para todo $ A \in \mathcal{F}_s $, se cumple que $ E(\Delta M_{s+1} | \mathcal{F}_s) = 0 $ por lo que $ M $ es una martingala.
\end{proof}
\bigskip
Nos encontramos ahora un contexto general donde no asumimos que el modelo sea finito o que $ \mathbb{F} $ sea generada por $ S $. Supongamos que el proceso de los precios actualizados $ \bar{S} $ es una martingala bajo una probabilidad $ Q $, esto es:
\[
E_Q (\Delta \bar{S}^i_t | \mathcal{F}_{t-1}) = 0, \text{ para } t \in \mathbb{T}\setminus \{0\} \text{ e } i = 0,\dots,l,
\]
donde $ E_Q $ significa la esperanza respecto de la probabilidad (medida) $ Q $. Sea $ \theta \in $\nolinebreak$ \Theta_a $ una estrategia admisible tal que los procesos de los precios actualizados son integrables respecto a $ Q $. Por \eqref{selfinifonlyif} tenemos que:
\begin{equation*}
\begin{split}
\bar{V}_t(\theta) &= V_0(\theta) + \bar{G}_t(\theta) \\
&= \langle \theta_{1}, S_0 \rangle + \sum_{u=1}^{t} \langle \theta_{u}, \Delta \bar{S}_u \rangle \\
&= \sum_{i=1}^{l}(\theta_1^i S_0^i + \sum_{u=1}^{t} \theta_{u}^i \Delta \bar{S}_u^i).
\end{split}
\end{equation*}
Vemos entonces que $ \bar{V}_t(\theta) $ es una constante más una suma finita de martingalas transformadas, por lo que también es una martingala con valor inicial $ V_0 (\theta) $. Entonces, tenemos que: \[ E(\bar{V}_t (\theta)) = E(V_0 (\theta)) = V_0(\theta) .\] 
Esta situación imposibilita la existencia de arbitraje. Si sabemos de antemano que los procesos de los precios actualizados son integrables respecto a $ Q $, supongamos que $ V_0 (\theta) = 0$ y $ V_T (\theta) \geq 0$ casi seguramente (respecto a $ Q $). Como $ E_Q = (\bar{V}_t ( \theta)) = 0 $ se sigue que $ V_T(\theta) = 0$ casi seguramente (respecto a $ Q $). Esto sigue siendo verdadero casi seguramente para $ P $, demostrando que $ P $ y $ Q $ tienen los mismos conjuntos vacíos. Llegamos entonces a la siguiente definición:
\bigskip
\begin{definicion}
	Una probabilidad $ Q $ que sea equivalente a $ P $ como media $ (Q\sim P) $ es una medida martingala equivalente (EMM) para $ S $ si el proceso de los precios actualizados $ \bar{S} $ es una martingala bajo $ Q $ para la filtración $ \mathbb{F} $. Es decir, para cada $ i =0,\dots, d $ tenemos que $ \bar{S}^i $ es una $ (\mathbb{F},Q) $-martingala.
\end{definicion}
\bigskip
Todo lo presentado anteriormente, nos aporta el siguiente resultado que acabamos de demostrar:
\bigskip
\begin{proposicionBox}\label{martThenViab}
	Si existe una medida martingala equivalente para $ S $, entonces el modelo de mercado discreto es viable, es decir, no contiene ninguna oportunidad de arbitraje.	
\end{proposicionBox}
\bigskip
El recíproco de esta afirmación es cierto, tal y como exponemos en el siguiente teorema que es el objetivo final de este capítulo y en el que usaremos el teorema de la alternativa de Gordan en forma equivalente del teorema de separación establecido en el corolario \ref{coroSep}. Recibe el nombre de ``primer teorema fundamental de asignación de precios''.
\bigskip
\begin{teoremaBox}(Primer reorema fundamental de asignación de precios)\label{VIABLEiofEMM}
	Un modelo de mercado discreto es viable si, y solo si, existe una medida de martingala equivalente para $ S $.
\end{teoremaBox}
\begin{proof}
	Ya sabemos por la proposición \ref{martThenViab} que la existencia de una medida de martingala equivalente garantiza la viabilidad del modelo por lo que solo tenemos que probar la otra implicación. \\
	
	Suponemos entonces que el modelo es viable. Necesitamos construir una medida $ Q \sim P $ en la que los precios son martingalas relativas a la filtración $ \mathbb{F} $. Sea $ C $ es el cono convexo de todas las variables aleatorias reales $ \phi $ en $ (\Omega, \mathcal{F}) $ tales que $ \phi(\omega) \geq 0 $ casi seguramente y $ \phi(\omega_i) > 0 $ para al menos un $ \omega_i = \Omega = \{\omega_i,\dots, \omega_n \} $. Asumimos $ p_i = P(\{\omega_i\}) > 0 $. Por el lema \ref{3.2.1}, hemos visto que para un mercado viable debemos tener $ \bar{G}_t(\hat{\theta}) \notin C$ para todos los procesos predecibles $ \hat{\theta} \in  \RR^l $. Por otro lado, el conjunto definido por tales ganancias 
	\[
	L = \{\bar{G}_t(\hat{\theta}):\hspace{0.5mm} \hat{\theta}=(\theta^1,\dots,\theta^l),\text{ con } \theta^i \text{ predecible para } i=1,\dots,l \},
	\]
	es un subespacio vectorial del espacio de todas las funciones reales y  $ \mathcal{F} $-medibles en $ \Omega $. Como $ L $ y $ C $ son disjuntos, podemos separar $ L $ y el subconjunto compacto de $ C $ definido como $ K =\{X\in C: \hspace{0.5mm} E_P(X) = 1\} $ gracias al corolario \ref{coroSep}. Sea $ \xx^0 \in \RR^n $ el elemento que proporciona dicho resultado. Tomamos $ \xi_i = (0,\dots,\frac{1}{p_i},\dots,0) $ para $ i \leq n $. Vemos que $ E_P(\xi_i) = \frac{p_i}{p_i} = 1$ por lo que $ \xi_i \in K $ y $ \langle \xx^0, \xi_i \rangle = \frac{x^0_i}{p_i} > 0$. Por ello, $ x^0_i > 0$ para todo $ i=1,\dots,n $.
	
	Definimos ahora el funcional lineal $ g(\xx) = \frac{\langle \xx^0, \xx\rangle}{\alpha} $ donde $ \alpha = \sum_{i=1}^{n} x^0_i$. Sea $ p^* \in \RR^n$ un vector con $ p^*_i = \frac{x_i^0}{\alpha} $ por lo que $ \sum_{i=1}^{n} p^*_i = 1 $. Usamos el vector $ p^* $ para inducir una probabilidad $ P^* $ en $ \Omega = \{\omega_1,\dots,\omega_n\} $ haciendo $ P^*(\{\omega_i\}) = p^*_{i} > 0 $. Veamos que $ P^* $ es la martingala equivalente deseada. En efecto, sea $ E^*(\cdot) $ que denota la esperanza relativa a $ P^* $. Nuevamente, por el corolario \ref{coroSep}, tenemos que $ g(\xx) = \frac{1}{\alpha}\langle \xx^0,\xx\rangle = 0$ para todo $ \xx\in L $. En particular, esta situación se da para  para $ \bar{G}_T (\hat{\theta}) $ con $ \hat{\theta}=(\theta^1,\dots,\theta^l) $ un vector de procesos predecibles por lo que.
	\[ E^*(\bar{G}_T (\hat{\theta})) = 0.\]
	Hemos conseguido una estrategia auto-financiada $ \theta $ con $ V_0(\theta) = 0 $. Como $ \bar{V}_T (\theta) = V_0(\theta) + \bar{G}_T (\theta)$, implica que $ E^*(\bar{V}_T(\theta)) = 0 $ para ese $ \theta $. Por el lema \ref{2.2.1} podemos generar tal $ \theta $ para cada proceso predecible de $ n $  dimensiones. En particular, lo podemos hacer para $ (0,\dots, \theta^i,\dots,0) $ donde $ i\leq n $. Por lo tanto
	\[
	E^*(\sum_{t=1}^{T}\theta_t^i\Delta \bar{S}^i_t) = 0
	\]
	se da para cada proceso predecible y acotado $ (\theta_i)_{i=1,\dots,T} $. El teorema \ref{2.3.5} implica que cada $ S^i $ es una martingala bajo $ P^* $. Por ello, $ P^* \sim P $.
\end{proof}
\bigskip