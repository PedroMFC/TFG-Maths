\documentclass[a4paper,11pt]{book}
%\documentclass[a4paper,twoside,11pt,titlepage]{book}
\usepackage{listings}
\usepackage[utf8]{inputenc}
% \usepackage[style=list, number=none]{glossary} %
%\usepackage{titlesec}
%\usepackage{pailatino}
\usepackage[spanish]{babel}
\usepackage{amsmath,amssymb,amsthm,mathtools}
\usepackage{mdframed}
\usepackage{lipsum}
\usepackage{dcolumn}
\usepackage{titlesec}
\usepackage{pgfplots} %% PINTAR
\newcolumntype{.}{D{.}{\esperiod}{-1}}
\makeatletter
\makeatother

%%%%%%%%%% LO QUE YO HE INTRODUCIDO %%%%%%%%%%%%
 % Recuadrar teorema/lema y asignarlo a un capítulo
\newmdtheoremenv{teoremaBox}{Teorema}[chapter]
\newmdtheoremenv{lemaBox}{Lema}[chapter]
\newmdtheoremenv{proposicionBox}{Proposición}[chapter]

\newtheorem{observacion}{Observación}[chapter]
 % Usar recuadro negro el terminar demostración
\renewcommand\qedsymbol{$ \blacksquare $}
%\renewcommand\proof{\textit{Demostración:}  \qedsymbol}

 % EVITAR QUE PONGA "CHAPTER *" AL INICIO DEL CAPÍTULO
\titleformat{\chapter}[display]{\normalfont\bfseries}{}{0pt}{\Huge}

\def\spanishoperators{adj traza vect dom dist sop vol sgn  Hess Jac rango grado diag img longitud Maximizar Minimizar Optimizar sec cotan cosec}
\newcommand{\refPar}[1]{(#1)}
\newcommand{\w}{\displaystyle}
\newcommand{\norm}[1]{\left\Vert#1\right\Vert}
\newcommand{\abs}[1]{\left\vert#1\right\vert}
\newcommand{\pre}[1]{\left\langle#1\right\rangle}
\newcommand{\set}[1]{\left\{#1\right\}}
\newcommand{\RR}{\mathbb{R}}
\newcommand{\NN}{\mathbb{N}}
\newcommand{\ZZ}{\mathbb{Z}}
\newcommand\restr[2]{{% ejemplo: \restr{f}{A}
		\left.\kern-\nulldelimiterspace #1 \vphantom{\big|} \right|_{#2} 
}}

\newcommand{\vecN}[2][N]{$\{{#2}_{1},...,{#2}_{#1} \}$ }
\newcommand{\vecSpace}{ $V$ }

%%%%%%%%%%%%%%%%%%%%%%%%%%%%%%%%%%%%%%%%%%%%%%%%%%%%%%%%%%%%%%

%\usepackage[chapter]{algorithm}
\RequirePackage{verbatim}
%\RequirePackage[Glenn]{fncychap}
\usepackage{fancyhdr}
\usepackage{graphicx}
\usepackage{afterpage}

\usepackage{longtable}

\usepackage[pdfborder={000}]{hyperref} %referencia

% ********************************************************************
% Re-usable information
% ********************************************************************
\newcommand{\myTitle}{Título del proyecto\xspace}
\newcommand{\myDegree}{Grado en ...\xspace}
\newcommand{\myName}{Nombre Apllido1 Apellido2 (alumno)\xspace}
\newcommand{\myProf}{Nombre Apllido1 Apellido2 (tutor1)\xspace}
\newcommand{\myOtherProf}{Nombre Apllido1 Apellido2 (tutor2)\xspace}
%\newcommand{\mySupervisor}{Put name here\xspace}
\newcommand{\myFaculty}{Escuela Técnica Superior de Ingenierías Informática y de
Telecomunicación\xspace}
\newcommand{\myFacultyShort}{E.T.S. de Ingenierías Informática y de
Telecomunicación\xspace}
\newcommand{\myDepartment}{Departamento de ...\xspace}
\newcommand{\myUni}{\protect{Universidad de Granada}\xspace}
\newcommand{\myLocation}{Granada\xspace}
\newcommand{\myTime}{\today\xspace}
\newcommand{\myVersion}{Version 0.1\xspace}


\hypersetup{
pdfauthor = {\myName (email (en) ugr (punto) es)},
pdftitle = {\myTitle},
pdfsubject = {},
pdfkeywords = {palabra_clave1, palabra_clave2, palabra_clave3, ...},
pdfcreator = {LaTeX con el paquete ....},
pdfproducer = {pdflatex}
}

%\hyphenation{}


%\usepackage{doxygen/doxygen}
%\usepackage{pdfpages}
\usepackage{url}
\usepackage{colortbl,longtable}
\usepackage[stable]{footmisc}
%\usepackage{index}

%\makeindex
%\usepackage[style=long, cols=2,border=plain,toc=true,number=none]{glossary}
% \makeglossary

% Definición de comandos que me son tiles:
%\renewcommand{\indexname}{Índice alfabético}
%\renewcommand{\glossaryname}{Glosario}

\pagestyle{fancy}
\fancyhf{}
\fancyhead[LO]{\leftmark}
\fancyhead[RE]{\rightmark}
\fancyhead[RO,LE]{\textbf{\thepage}}
\renewcommand{\chaptermark}[1]{\markboth{\textbf{#1}}{}}
\renewcommand{\sectionmark}[1]{\markright{\textbf{\thesection. #1}}}

\setlength{\headheight}{1.5\headheight}

\newcommand{\HRule}{\rule{\linewidth}{0.5mm}}


%Definimos los tipos teorema, ejemplo y definición podremos usar estos tipos
%simplemente poniendo \begin{teorema} \end{teorema} ...
\newtheorem{teorema}{Teorema}[chapter]
\newtheorem{ejemplo}{Ejemplo}[chapter]
\newtheorem{definicion}{Definición}[chapter]

\definecolor{gray97}{gray}{.97}
\definecolor{gray75}{gray}{.75}
\definecolor{gray45}{gray}{.45}
\definecolor{gray30}{gray}{.94}

\lstset{ frame=Ltb,
     framerule=0.5pt,
     aboveskip=0.5cm,
     framextopmargin=3pt,
     framexbottommargin=3pt,
     framexleftmargin=0.1cm,
     framesep=0pt,
     rulesep=.4pt,
     backgroundcolor=\color{gray97},
     rulesepcolor=\color{black},
     %
     stringstyle=\ttfamily,
     showstringspaces = false,
     basicstyle=\scriptsize\ttfamily,
     commentstyle=\color{gray45},
     keywordstyle=\bfseries,
     %
     numbers=left,
     numbersep=6pt,
     numberstyle=\tiny,
     numberfirstline = false,
     breaklines=true,
   }
 
% minimizar fragmentado de listados
\lstnewenvironment{listing}[1][]
   {\lstset{#1}\pagebreak[0]}{\pagebreak[0]}

\lstdefinestyle{CodigoC}
   {
	basicstyle=\scriptsize,
	frame=single,
	language=C,
	numbers=left
   }
\lstdefinestyle{CodigoC++}
   {
	basicstyle=\small,
	frame=single,
	backgroundcolor=\color{gray30},
	language=C++,
	numbers=left
   }

 
\lstdefinestyle{Consola}
   {basicstyle=\scriptsize\bf\ttfamily,
    backgroundcolor=\color{gray30},
    frame=single,
    numbers=none
   }


\newcommand{\bigrule}{\titlerule[0.5mm]}


%Para conseguir que en las páginas en blanco no ponga cabecerass
\makeatletter
\def\clearpage{%
  \ifvmode
    \ifnum \@dbltopnum =\m@ne
      \ifdim \pagetotal <\topskip
        \hbox{}
      \fi
    \fi
  \fi
  \newpage
  \thispagestyle{empty}
  \write\m@ne{}
  \vbox{}
  \penalty -\@Mi
}
\makeatother

\usepackage{pdfpages}
\begin{document}
%\frontmatter
%\tableofcontents
%\listoffigures
%\listoftables
%
%\mainmatter
%\setlength{\parskip}{5pt}

%\begin{titlepage}
	
	\newlength{\centeroffset}
	\setlength{\centeroffset}{-0.5\oddsidemargin}
	\addtolength{\centeroffset}{0.5\evensidemargin}
	\thispagestyle{empty}
	
	\noindent\hspace*{\centeroffset}\begin{minipage}{\textwidth}
		
		\centering
		\includegraphics[width=0.9\textwidth]{imagenes/logo_ugr.jpg}\\[1.4cm]
		
		\textsc{ \Large TRABAJO FIN DE GRADO\\[0.2cm]}
		\textsc{ INGENIERÍA INFORMÁTICA Y MATEMÁTICAS}\\[1cm]
		% Upper part of the page
		% 
		% Title
		{\Large\bfseries Teoremas de la alternativa, optimización convexa, valoración de activos financieros \\
		}
		{\Large\bfseries y \\
		}
		%\noindent\rule[-1ex]{\textwidth}{2pt}\\[3.5ex]
		{\Large\bfseries procesamiento de nubes de puntos generadas por escáner láser.\\
		}
		\noindent\rule[-1ex]{\textwidth}{2pt}\\[3.5ex]
	\end{minipage}
	
	\vspace{1cm}
	\noindent\hspace*{\centeroffset}\begin{minipage}{\textwidth}
		\centering
		
		\textbf{Autor}\\ {Pedro Manuel Flores Crespo}\\[2.5ex]
		\textbf{Directores}\\
		{Manuel Ruiz Galán \\
			Juan Carlos Torres Cantero }\\[2cm]
		
		%\begin{figure}
		\begin{minipage}[btr]{0.25\textwidth}
			\centering
			\includegraphics[width=\textwidth]{imagenes/ciencias_logo.png}
		\end{minipage}
		\hspace{1cm}
		\begin{minipage}[btl]{0.3\textwidth}
			\centering
			\includegraphics[width=\textwidth]{imagenes/etsiit_logo.png}
			%\textsc{Escuela Técnica Superior de Ingenierías Informática y de Telecomunicación}\\
		\end{minipage}\\
		%\end{figure}
		\vspace{1cm} 
		\textsc{---}\\
		Granada, junio de 2020
	\end{minipage}
	%\addtolength{\textwidth}{\centeroffset}
	%\vspace{\stretch{2}}
\end{titlepage}




\chapter{Teorema de Hanh-Banach y de Mazur-Orlicz}
	\thispagestyle{empty}
	\paragraph{}El objetivo principal de esta sección es demostrar tanto una versión no tan conocida del clásico teorema de Hanh-Banach como el teorema de Mazur-Orlicz. Para ello, iremos utilizando una serie de lemas previos que nos facilitarán el proceso. Así, una vez vistos estos resultados, habremos construido la base para el lema de Simons 
	
	\paragraph{} En primer lugar vamos a recordar la definición de funcional sublineal sobre un espacio vectorial \vecSpace . Notar que todos los espacios vectoriales que vamos a usar son reales. Del mismo modo, los conjuntos que usaremos asumiremos que son no vacíos
	
	\begin{definicion}
		Sea \vecSpace un espacio vectorial distinto de cero. Decimos que el $P:\vecSpace \rightarrow \RR$ es sublineal si cumple las siguientes condiciones:
		\begin{itemize}
			\item $ P $ es subaditiva: $x_1, x_2 \in \vecSpace \Longrightarrow P(x_1 + x_2) \leq P(x_1) + P(x_2) $.
			\item $ P $ es positivamente homogénea: $x_1 \in \vecSpace $ y $ \lambda > 0 \Longrightarrow P(\lambda x) = \lambda P(x) $.
		\end{itemize}
	\end{definicion}

	\paragraph{}Por ejemplo, toda seminorma sobre \vecSpace es un funcional sublineal. También, si $ \vecSpace = \RR $ y definimos $ P(x) = \max \{0,x\}, \forall x \in \RR $ obtenemos un funcional sublineal sobre $\RR$.
	
	\paragraph{} El lema que exponemos a continuación, y que generalizaremos posteriormente en el lema \ref{lema2}, nos servirá para demostrar el teorema de Hanh-Banach. 
	
	\begin{lemaBox}\label{lema1}
		Sea\vecSpace un espacio vectorial distinto de cero y $P:\vecSpace \rightarrow \RR$ un funcional sublineal. Fijamos un elemento $ y \in V $. Para todo $ x \in V $ tomamos  
		\begin{center}
			$ P_y(x) := \inf_{\lambda > 0} \left[P(x+\lambda y) - \lambda P(y)\right] $
		\end{center}
		
		Entonces, $ P_y:V \rightarrow \RR$, $ P_y $es sublineal, $ P_y \leq P $ y $ P_y (-y) \leq  -P(y)$.
	\end{lemaBox} 
	\begin{proof}
		Fijamos $ y \in V $. Sea $ x \in V $ y $ \lambda > 0$, como P es sublineal tenemos: 
		\begin{center}
			$ \lambda P(y) = P(\lambda y) =P(\lambda y +x-x) \leq P(x+\lambda y)+ P(-x)$
		\end{center}
		
		Por lo tanto, se obtiene que $ P(x+\lambda y) - \lambda P(y) \geq -P(-x) $.  Tomando el ínfimo sobre $ \lambda >0 $ llegamos a $ P_{y}(x)\geq -P(-x) > -\infty$. Por consiguiente, $ P_y:V \rightarrow \RR$. \\
		
		Probaremos ahora que $ P_y $ es sublineal. Empezamos viendo la subaditividad. Tomamos $ x_1, x_2 \in V $ y sean $ \lambda_1, \lambda_2 > 0$ arbitrarios. Entonces: 
		\begin{equation*}
		\begin{split}
		\left[ P(x_1 + \lambda_1 y) - \lambda_1 P(y) \right] &+ \left[ P(x_2 + \lambda_2 y) - \lambda_2 P(y) \right] \\
		& \geq \left[ P(x_1 + x_2 (\lambda_1+\lambda_2)y) \right] - (\lambda_1+\lambda_2) P(y) \\
		&\geq P_y (x_1 + x_2 )
		\end{split}
		\end{equation*}
		
		Tomando ínfimo sobre $ \lambda_1 $ y $ \lambda_2 $, $  P_y (x_1)  + P_y (x_2 ) \geq P_y (x_1 + x_2 ) $. Así, $ P_y $ es subaditiva. Para comprobar que es positivamente homogénea tomamos $ x \in V $ y $ \mu > 0 $. Entonces:
		\begin{equation*}
		\begin{split}
		P_y (\mu x) &= \inf_{\lambda > 0} \left[P(\mu x+\lambda y) - \lambda P(y)\right] = \mu \inf_{\lambda > 0} \left[P(u x+ (\lambda / \mu) y) - (\lambda / \mu) P(y)\right] \\
		&= \mu \inf_{\upsilon > 0} \left[P(u x+ \upsilon y) - \upsilon  P(y)\right] = \mu P_y (x)
		\end{split}
		\end{equation*}
		
		Obtenemos que $ P_y $ es positivamente homogénea y como consecuencia sublineal. \\
		
		Para demostrar que $ P_y \leq P $, sea $ x \in V $ y tomando $ \lambda = 1 $: 
		\[ P_y(x) \leq P(x+y) - P(y) \leq P(x)+ P(y) - P(y) = P(x) \Longrightarrow P_y \leq P  \]
		
		Finalmente, actuando de manera similar al caso anterior:
		\[ \qquad \qquad \qquad \qquad P_y(-y) \leq P(-y+y) - P(y) = -P(y)  \]
		
	\end{proof}
	\paragraph{} Ahora procedemos a probar el teorema de Hanh-Banch para funcionales sublineales, el cual es uno de los resultados más importantes del análisis funcional.	
	\begin{teoremaBox}[Hanh-Banach]\label{H-B}
		Sea V un espacio vectorial distinto de cero y $P:V \rightarrow \RR$ un funcional sublineal. Entonces existe un funcional lineal L en V tal que $ L \leq P $.
	\end{teoremaBox}
	\begin{proof}
		Sea $ \mathcal{Q} $ el conjunto de funcionales sublineales $ Q $ en\vecSpace tales que $ Q \leq P $. Primero probaremos que todo subconjunto $ \mathcal{T} $ totalmente ordenado de $ \mathcal{Q} $ tiene una cota inferior en $ \mathcal{Q} $. Para $ T_1 ,T_2 \in \mathcal{Q} $ tenemos la relación de orden usual, es decir:
		\begin{center}
			$ T_1 \leq T_2 \Longleftrightarrow T_1 (x) \leq T_2 (x) \quad \forall x \in \vecSpace $
		\end{center}
		
		Definimos $ Q(x):=\inf \{ T(x): T \in \mathcal{T} \} $. Si $ x \in \vecSpace $ y $ T \in \mathcal{T} $, como T es subaditiva obtenemos la siguiente relación $ 0 = T(0) = T(x-x) \leq T(x) + T(-x) \Longrightarrow T(x) \geq -T(-x) $ (1). Como $ T \in \mathcal{Q} \Longrightarrow T(x) \leq P(x) \Longrightarrow -T(x) \geq -P(x)$ (2). Usando (1), (2) y tomando ínfimo sobre $  T $  llegamos a $ Q(x) \geq -P(x) \geq - \infty $. Por lo tanto $ Q:V \rightarrow \RR$. \\
		
		Ahora probaremos que $ Q $ es subaditiva. Para ello, tomamos $ x_1, x_2\in \vecSpace $. Sean $ T_1 , T_2 \in \mathcal{T} $ arbitrarios. Si $ T_1 \geq T_2 $ (el caso de $ T_2 \geq T_1 $ es análogo.):
		
		\begin{center}
			$ T_1 (x_1)+  T_2 (x_2) \geq T_2(x_1)+  T_2 (x_2) \geq T_2(x_1 +x_2) \geq Q(x_1 + x_2)$
		\end{center}
		
		Concluimos que ambos casos $ T_1 (x_1)+  T_2 (x_2) \geq Q(x_1 + x_2)$. Tomando ínfimo en $ T_1 $ y $ T_2 $ obtenemos que $ Q (x_1)+  Q(x_2) \geq Q(x_1 + x_2)$. Así, $ Q $ es sublineal. Que sea positivamente homogénea es consecuencia de que $ T $ también lo es. Dado $ \mu > 0 $:
		\begin{equation*}
		\begin{split}
		Q(\mu x) &=\inf \{ T(\mu x): T \in \mathcal{T} \} \\ 
		& = \inf \{ \mu T( x): T \in \mathcal{T} \} \\ 
		&= \mu\inf \{ T( x): T \in \mathcal{T} \} \\ 
		&= \mu Q(x) 
		\end{split}
		\end{equation*}
		
		De este modo, Q es sublineal y como es claro que $ Q \leq P \Longrightarrow Q \in \mathcal{Q}$. Así, es directo que $ Q $ es el elemento minimal de $ \mathcal{T} $ en $\mathcal{Q}$.\\
		
		El lema de Zorn nos proporciona entonces un elemento minimal de $ \mathcal{Q} $ que llamaremos $ L $. Tomamos ahora $ y \in \vecSpace $. Con la notación del lema anterior, $ L_y : \vecSpace \longrightarrow \RR $ es sublineal, $ L_y \leq L $ (como consecuencia $ L_y \in \mathcal{Q} $) y $ L_y (-y) \leq L(-y) $. De hecho, como $ L $ es minimal en $ \mathcal{Q} $, $ L_y = L $ y por ello $ L (-y) \leq L(-y) $. Por otro lado, como L es subaditiva, $ L(-y) \geq -L(y) $. Combinando ambas desigualdades, $ L(-y) = -L(y) $. Tomamos $ x \in \vecSpace $ y $ \lambda < 0 $, usando la igualdad anterior llegamos a:
		\[ \qquad \quad
		L(\lambda x) = L (-(-\lambda)x) = -L(-\lambda x) = -(-\lambda)L(x) = \lambda L(x) \label{1}
		\] 
		
		obteniendo que $ L $ es homogénea. Si $ x_1, x_2 \in \vecSpace $, la subaditividad de $ L $ nos da $ L(-x_1-x_2) \leq L(-x_1) + L(-x_2) $. Usando la homogeneidad de $ L $:
		\begin{equation*}
		\begin{split} \qquad
		L(x_1+x_2) &= L(-(-x_1-x_2)) = -L(-x_1-x_2) \\ 
		& \geq -L(-x_1)-L(-x_2) = L(x_1) + L (x_2) \geq L(x_1+x_2) 
		\end{split}
		\end{equation*}
		
		Por ello, $	L(x_1+x_2) = L(x_1) + L (x_2) $ y concluimos que $ L $ es lineal.
		
	\end{proof}

	\paragraph{} El siguiente resultado importante que demostraremos será el teorema de Mazur-Orlicz. Primero, veamos un lema previo.
	
	\begin{lemaBox}\label{lema2}
		Sea\vecSpace un espacio vectorial distinto de cero y $P:\vecSpace \rightarrow \RR$ un funcional sublineal. Sea $ D $ un subconjunto no vacío y convexo de \vecSpace y $ \beta := \inf_D P \in \RR $. Para todo $ x \in V $ tomamos  
		\begin{center}
			$ Q(x) := \inf_{d \in D, \lambda > 0} \left[P(x+\lambda d) - \lambda \beta\right] $
		\end{center}
		
		Entonces, $ Q:V \rightarrow \RR$, $ Q $es sublineal, $ Q \leq P $ y $ \forall d \in D, -Q(-d) \geq \beta$.
	\end{lemaBox} 
	\begin{proof}
		Si $ x \in \vecSpace, \quad d \in D $ y $ \lambda > 0 $ entonces
		\begin{center}
			$ P(x+ \lambda d) - \lambda \beta \geq -P(-x) + \lambda P(d)-\lambda\beta \geq -P(-x) \geq -\infty$
		\end{center}
		
		La primera igualdad se deduce de la linealidad de P ya que:
		\[ \lambda P(d) = P(\lambda d) =P(\lambda d +x-x) \leq P(x+\lambda d)+ P(-x) \Longrightarrow -P(-x) \leq P(x+\lambda d)\]
		
		Y la segunda a que como $ \beta = \inf_D P \Longrightarrow \lambda P(d) \geq \lambda\beta \Longrightarrow\lambda P(d) - \lambda\beta \geq 0 $. Tomando el ínfimo sobre $ d \in D  $ y $ \lambda > 0 $ llegamos a $ Q(x)\geq -P(-x) > -\infty$. Por consiguiente, $ Q:V \rightarrow \RR$. Es relativamente fácil probar que $ Q $ es positivamente homogénea por lo que para ver que es sublineal solo queda ver la subaditividad. Para ello, tomamos $ x_1, x_2 \in V $. Sean $ d_1, d_2 \in D $ y $ \lambda_1, \lambda_2 > 0$ arbitrarios. Para simplificar la notación llamamos $ x := x_1 + x_2 $, $ \lambda := \lambda_1 + \lambda_2 $ y $ d:= (\lambda_1/\lambda)d_1 + (\lambda_2/\lambda)d_2 $. Notar que $ d \in D $ al ser este convexo. Entonces: 
		\begin{equation*}
		\begin{split}
		\left[ P(x_1 + \lambda_1 d_1) - \lambda_1 \beta \right] + \left[ P(x_2 + \lambda_2 d_2) - \lambda_2 \beta \right] &\geq P(x + \lambda_1 d_1 + \lambda_2 d_2) - \lambda \beta\\
		& = P(x +\lambda d) - \lambda \beta\\ 
		& \geq Q(x) = Q(x_1 + x_2)
		\end{split}
		\end{equation*}
		
		Tomando ínfimo sobre $ d_1, d_2, \lambda_1 $ y $ \lambda_2 $, $  Q(x_1) + Q (x_2 ) \geq Q (x_1 + x_2 ) $. Así, $ Q $ es subaditiva y como consecuencia sublineal. Fijamos $ d \in D $. Sea $ x \in \vecSpace $ arbitrario. Entonces, $ \forall \lambda > 0 $, $ Q(x) \leq P(x) + \lambda \left[P(d) - \beta \right]$. Tomando $ \lambda \longrightarrow 0 $, $ Q(x) \leq P(x)$ y como consecuencia $ Q \leq P $. Finalemente, sea $ d \in D $ arbitrario y tomando $ \lambda = 1 $:
		\[ \qquad \qquad \qquad Q(-d) \leq Q(-d+d) - \beta = -\beta \Longrightarrow  -Q(-d) \geq \beta  \]
		
	\end{proof}

	\paragraph{}Visto este lema, estamos preparados para ver el resultado que nos interesa:
	
	\begin{teoremaBox}[Mazur-Orlicz]
		Sea\vecSpace un espacio vectorial distinto de cero y $P:\vecSpace \rightarrow \RR$ un funcional sublineal.  Sea $ D $ un subconjunto no vacío y convexo de \vecSpace. Entonces existe un funcional lineal $ L $ sobre \vecSpace tal que $ L \leq P $ e $ \inf_D L = \inf_D P $
	\end{teoremaBox}
	\begin{proof}
		Sea $ \beta := \inf_D P $. En el caso de que $ \beta = -\infty $ por el teorema de Hanh-Banach tenemos que $ \exists L $ sobre \vecSpace tal que es lineal y $ L \leq P$. Así:
		\begin{center}
			$ L \leq P \Longrightarrow inf_D L \leq \inf_D P = -\infty \Longrightarrow inf_D L = \inf_D P$ 
		\end{center}
		
		Supongamos entonces que $ \beta \in \RR $. Definimos el funcional auxiliar $ Q $ como en el lema \ref{lema2}. Del teorema de Hanh-Banach obtenemos que existe un funcional lineal $ L $ sobre \vecSpace tal que $ L \leq Q$ (como $ Q \leq P $ tenemos que $ L \leq P $). Sea $ d \in D $, entonces:
		\[
		L(d) = -L(-d) \geq -Q(-d) \geq \beta
		\]
		Tomando ínfimo sobre $ d \in D $:
		\[
		\inf_D L \geq \beta = \inf_D P
		\]
		Por otro lado, como $ L \geq P $:
		\[
		\inf_D L \leq\inf_D P
		\]
		Juntando ambas desigualdades obtenemos $ \inf_D L =\inf_D P $
	\end{proof}



\chapter{Igualdad ???}
\newcommand{\normSpace}{E}
	
Ahora, queremos obtener una igualdad que nos será de utilidad posteriormente para el Teorema de Separación. Empezamos recordando algunas nociones básicas de Análisis Funcional. El primer resultado que caracteriza la continuidad de operadores lineales en espacios normados.
	
\begin{proposicionBox}\label{caracCont}
Sean $ \normSpace_1, \normSpace_2 $ dos espacios normados y $ L:\normSpace_1 \longrightarrow \normSpace_2 $ un operador lineal. Entonces, L es continuo si, y solo si, verifica la siguiente condición:
\[
\exists \alpha > 0 : \norm{L(x)} \leq \alpha \norm{x} \quad \forall x \in \normSpace_1
\]
\end{proposicionBox}
\begin{proof}
Destacar que notamos de igual manera a la norma de $ \normSpace_1 \text{ y } \normSpace_2$ lo que no debería causar confusión.\\

En primer lugar, supongamos que $ L $ es continuo. Como es continuo en todo punto de $ \normSpace_1 $, en particular, lo es en 0. De este modo, para $ \varepsilon = 1 $, existe $ \delta >0  $  tal que, para $ y \in \normSpace_1 $ con $ \norm{y} \leq \delta$ se tiene $ \norm{L(y)} \leq 1 $. Dado $ x \in \normSpace_1 \setminus \{0\} $, podemos tomar $ y = \delta x / \norm{x} $, para obtener $ \delta \norm{L(x)} / \norm{x} = \norm{L(y)} \leq 1 $. Esto es $ \norm{L(x)} \leq (1/\delta) \norm{x} $. Esto es válido para todo $ x \in \normSpace_1 \setminus \{0\} $ y obvio para $ x = 0 $ por lo que podemos tomar $ \alpha = 1/\delta $. \\

Por otro lado si $ \exists \alpha > 0 $ tal que  $ \norm{L(x)} \leq \alpha \norm{x} $ para todo $ x \in \normSpace_1 $, para cualesquiera $ x,y \in \normSpace_1 $ se tiene que $ \norm{L(x) - L(y)} \leq \alpha \norm{x-y} $. Así, $ L $ es lipschitziana de $ \normSpace_1 \text{ a } \normSpace_2$ por lo que es continua (en realidad uniformemente continua).
\end{proof}

Consideramos el espacio dado por:
\[
\normSpace^* =  \lbrace L:\normSpace \longrightarrow \RR: L \text{ es lineal y continuo} \rbrace.
\]
Para todo $ L \in \normSpace^* $ definimos su norma como su constante de Lipschitz, es decir:
\[
\norm{L} = \min \lbrace \alpha > 0:  \norm{L(x)} \leq \alpha \norm{x} \quad \forall x \in \normSpace_1\rbrace.
\]
De este modo, podemos escribir:
\begin{equation}\label{desigNorma}
	\norm{L(x)} \leq \norm{L} \norm{x}
\end{equation}
siendo dicha desigualdad óptima. También podemos expresar su norma como el mínimo mayorante de un conjunto mayorado, que es el supremo:
\[
\norm{L} = \sup \lbrace \norm{L(x)}/\norm{x} : \forall x \in \normSpace_1 \setminus \{0\}\rbrace.
\]
Para $  x \in \normSpace_1 \setminus \{0\} $ tenemos que $ \norm{L(x)}/\norm{x} = \norm{L(x/\norm{x})}$ y es claro que $ \{ x/\norm{x} : x \in \normSpace_1 \setminus \{0\}\} $ es la esfera unidad de $ \normSpace $ que notamos como $ S_\normSpace $. Si en vez de la esfera consideramos la bola unidad, $ B_\normSpace $ el supremo no varía. Efectivamente, si $ x \in B_\normSpace  $ se tiene que $ x = \norm{x}u $ con $ u \in  S_\normSpace$, y por ello $ \norm{L(x)} = \norm{x}\norm{L(u)} \leq \norm{L(u)} $ ya que $ \norm{x} \leq 1 $. De este modo, también tenemos que:
\begin{equation}\label{normSup}
	\norm{L} =\sup_{x \in B_\normSpace} \norm{L(x)}.
\end{equation}

En este momento, estamos en disposición de enunciar y demostrar la igualdad que deseamos:
\begin{proposicionBox}
	Dado un espacio normado $ \normSpace $ y $ x \in \normSpace $, entonces se cumple que:
	\begin{equation}\label{iguNorm}
	\sup_{x^* \in B_{\normSpace ^ *}} x^*(x) = \norm{x}.
	\end{equation}
\end{proposicionBox}
\begin{proof}
Consideramos el funcional 
\begin{equation*}
\begin{split}
P:\normSpace \longrightarrow &\RR \\
x \longmapsto &\norm{x}
\end{split}
\end{equation*} 
y el conjunto $ D = \{x_0\} $ con $ x_0 \in \normSpace $ arbitrario. Es claro que $ P $ es sublineal al estar definido como la norma en $ \normSpace $ y que $ D $ es convexo. Podemos aplicar el Teorema de Mazur-Orlicz, teorema \ref{M-O}, y obtenemos que existe un funcional $ L:\normSpace \longrightarrow \RR $ lineal tal que $ L \leq P $ e $ \inf_D L = \inf_D P $. Como $ L \leq P $, entonces 
\begin{equation*}
 	\norm{L(x)} \leq \norm{P(x)} = \norm{x} \quad \forall x \in \normSpace.
\end{equation*}
Usando la proposición \ref{caracCont} concluimos que $ L $ es continua. Como $ L \in \normSpace ^* $, llamamos $ L = x^* $ y llegamos a que $ \norm{L} = \norm{x^*}  \leq 1$ ($ x^* \in  B_{\normSpace ^ *} $).  Por su parte, como  $ \inf_D L = \inf_D x^* = \inf_D P $ y $ D = \{x_0\} $ entonces, $ x^*(x_0) = \norm{x_0} $. De este modo, llegamos a que existe $ x^* \in B_{\normSpace^*} $ tal que $ x^*(x_0) = \norm{x_0} $. Si $ y^* \in B_{\normSpace^*}  $, entonces $ \norm{y^*(x_0)} \leq \norm{y^*} \norm{x_0} \leq \norm{x_0} $ y podemos asegurar que:
\[
\sup_{x^* \in B_{\normSpace ^ *}} x^*(x_0) = \norm{x_0}.
\] 
Como $ x_0 \in \normSpace $ es arbitrario concluimos que dado $ x \in \normSpace $ se cumple que:
\begin{equation*}
\sup_{x^* \in B_{\normSpace ^ *}} x^*(x) = \norm{x}.
\end{equation*}
\end{proof}

\chapter{Lema de Simons y Teorema de la Alternativa de Gordan}
	\thispagestyle{empty}
	\newcommand{\ttt}{\textbf{\emph{t}}}
	\newcommand{\sss}{\textbf{\emph{s}}}
	
	\paragraph{}Este capítulo se centra en lema de Simons, el cual será de gran ayuda para demostrar el Teorema de la Alternativa de Gordan.

	\paragraph{}Antes de comenzar, necesitamos hacer la siguiente definición.
	
	\begin{definicion}
		Dado $ N \in \NN $ con $ N > 1 $ llamamos símplex unitario de $ \RR^N $ al conjunto convexo y compacto de $ \RR^N $ dado por:
		\begin{equation*}
			\Delta_N := \left\lbrace \ttt \in \RR^N: \sum_{i=1}^{N}{t_i} = 1 \hspace{0.5em} , \hspace{0.5em} t_1,...,t_N \geq 0 \right\rbrace .
		\end{equation*}
	\end{definicion}

	Por ejemplo, si $ N = 2 $, tenemos que $ \Delta_2 = \mathrm{co} \{(1,0),(0,1)\}$, es decir es la envolvente convexa de los vectores de la base usual de $ \RR^2 $. De hecho en general, $ \Delta_N = co\{e_1,...e_N\} $. \\
	
	Antes de continuar veamos que efectivamente $ \Delta_N $ es convexo y compacto:
	
	\begin{itemize}
		\item  Convexo: tenemos que comprobar que dados $ \ttt, \sss \in \Delta_N  \Longrightarrow \lambda\ttt + (1-\lambda)\sss \in \Delta_N, \lambda \in \lbrack 0,1 \rbrack$. En efecto, las coordenadas de $ \lambda\ttt + (1-\lambda)\sss \in \RR^N $ verifican:
		\begin{itemize}
			\item [i) ] $ \lambda t_i + (1-\lambda)s_i \geq 0 $ para todo $ i = 1,..., N $ ya que $ t_i, s_i \geq 0 $.
			\item [ii) ] $ \sum_{i=1}^{N} (\lambda t_i + (1-\lambda)s_i) =   \lambda \sum_{i=1}^{N} t_i + (1-\lambda)\sum_{i=1}^{N} s_i = \lambda + (1 - \lambda) = 1$ ya que $ \sum_{i=1}^{N} t_i = \sum_{i=1}^{N} s_i = 1  $.
		\end{itemize}
	
		Por lo tanto, $ \lambda\ttt + (1-\lambda)\sss \in \Delta_N \text{ para todo }\lambda \in \lbrack 0,1 \rbrack $.
		
		\item Compacto: al encontrarnos en $ \RR^N $ y aplicando el conocido Teorema de Heine-Borel basta ver que $ \Delta_N $ es cerrado y acotado. Claramente es acotado por lo que nos centraremos en la de cerrado. Sea $ \lbrace \ttt_n \rbrace_{ n \in \NN} $ una sucesión de $ \Delta_N $ y sea $ \ttt_0 \in \RR^N $ tal que $ \lbrace \ttt_n \rbrace_{ n \in \NN} \longrightarrow \ttt_0 $. Tenemos que comprobar que $ \ttt_0 \in \Delta_N $. 
		\begin{itemize}
		\item[i) ] Como todas las coordenadas de $ \ttt_n $ son no negativas podemos asegurar que las de $ \ttt_0 $ también lo son.
		\item[ii) ] la función $ f: \RR^N \longrightarrow \RR $ definida como $ f(\ttt) =  \sum_{i=1}^{N} t_i $ es continua. Claramente $ f(\ttt) = 1 $ para todo $ \ttt \in \Delta_N $ y por ello $ \lbrace f(\ttt_n) \rbrace_{ n \in \NN} \longrightarrow 1$. Por continuidad de $ f $ y unicidad de límite tenemos que $ f(\ttt_0) = 1 $ pero eso implica que la suma de sus componentes vale 1
		\end{itemize} 
	
		Así, hemos demostrado que $ \ttt_0 \in \Delta_N $ y por lo tanto $ \Delta_N $ es compacto. 
	\end{itemize}

	\paragraph{} Enunciamos este lema previo que usaremos posteriormente el la demostración del de Simons. 
	\begin{lemaBox}\label{lema2.1}
		Sea $ N \in \NN $, con $ N \geq 1 $ y $ L: \RR^N \longrightarrow \RR $ definida por $ S(x):=\max\{x_1,...,x_N\}$. Entonces, $ S $ es sublineal. Además, si $ L:\RR^N \longrightarrow \RR $ es un funcional lineal tal que $ L \leq S $ entonces $ L $ es de la forma $ L(x) = t_1 x_1 + ... + t_N x_N $ con $ (t_1,...,t_N) \in \Delta_N$. De hecho, el recíproco también es cierto, es decir, si $ L =  (t_1,...,t_N) \in \Delta_N $ entonces $ L \leq S $.
		
		%%%%%% NOTA %%%%%%%%
		%% Las aplicaciones lineales coninciden con el dual del espacio. Es decir, podemos identificar la aplicación con sus coeficientes obeniendo de ese modo un punto del espacio.
		
	\end{lemaBox}
	\begin{proof}
		Claramente, $ S $ es positivamente homogénea. También es subaditiva ya que dados $ x,y \in \vecSpace $ : 
		\begin{equation*}
		\begin{split}
		S(x+y) &= \max \{x_1 + y_1, ..., x_N +y_N \}\\ 
		&\leq \max\{x_1, ..., x_N\} + \max \{y_1, ...,y_N\} = S(x) + S(y)
		\end{split}
		\end{equation*}
		
		Por ello, $ S $ es sublineal. Para terminar veamos que 
		\begin{equation*}
			\left\lbrace L:\RR^N \longrightarrow \RR : L \text{ lineal y }L\leq S \right\rbrace = \Delta_N
		\end{equation*}
		
		\begin{itemize}
			\item[$ \supseteq $ )] Sea $ \mathbf{t} \in \Delta_N$ definimos $ L:\RR^N \longrightarrow \RR $ como $ L(x):= \langle \mathbf{t},x \rangle $. Es evidente que $ L $ es lineal al serlo el producto escalar. Dado $ x \in \RR $:
			\begin{equation*}
				L(x) = \sum_{i=1}^{N}{t_i x_i} \leq \sum_{i=1}^{N}{t_i S(x)} = S(x)\sum_{i=1}^{N}{t_i} = S(x)
			\end{equation*}
			donde la primera desigualdad se debe a que $ x_i \leq S(x)$ para todo $ x_i$ con $ i=1,...,N $ y a que $ t_i \geq 0 $ ya que $ \mathbf{t} \in \Delta_N$. Esto también justifica la que última igualdad ya que $ \sum_{i=1}^{N}{t_i} = 1 $.
			
			\item[$ \subseteq $ )] Sea $ L =  (t_1,...,t_N) \in \RR^N $ lineal tal que $ \forall x \in \RR^N $ cumple $ \sum_{i=1}^{N}{t_i x_i} \leq \max\{x_1,...,x_N\}$. Así, si tomamos $ e_i \in \RR^N $ donde $ e_i $ representa el i-ésimo elemento de la base usual de $ \RR^N $ con $ i=1,...,N $. Entonces:   		
			\begin{equation*}
				L(-e_i) = -t_i \leq 0 \Longrightarrow t_i \geq 0 \quad \forall i=1,...,N
			\end{equation*}
			Si ahora llamamos $ e = \sum_{i=1}^{N}{e_i} $. Obtenemos:
			\begin{equation*}
				\begin{rcases*}
				L(e) = \sum_{i=1}^{N}{t_i} \leq \max \{1, ...,1\} = 1 \\
				L(-e) = -\sum_{i=1}^{N}{t_i} \leq \max \{-1, ...,-1\} = -1
				\end{rcases*} \Longrightarrow \sum_{i=1}^{N}{t_i} = 1
			\end{equation*}
			Concluimos entonces que $ L =  (t_1,...,t_N) \in \Delta_N $.
		\end{itemize}
		
	\end{proof}

	\paragraph{} Enunciamos ahora lema de Simons\cite{Simons2008} en que destacamos la ausencia de hipótesis topológicas, lo que será importante posteriormente.
	
	\begin{lemaBox}[Lema de Simons]\label{Simons}
		Sea $ C $ un subconjunto no vacío y convexo de un espacio vectorial. Dadas $ f_1, ..., f_N $ ($ N \geq 1 $)  funciones sobre $ C $ reales y convexas, entones existen $ (t_1, ..., t_N) \in \Delta_N $ que cumplen
		\[
		\inf_C\left[ \max \{f_1, ..., f_N\}\right] = \inf_C \left[ t_1 f_1+ ...+t_N f_N \right]
		\] 
	\end{lemaBox}
	\begin{proof}
		Sea  \vecSpace = $ \RR^N $ con $ N \in \NN $. Definimos $S:\vecSpace \longrightarrow P $ como \[ S(x_1, ..., x_N) := \max \{x_1, ..., x_N\} \]. Por el lema \ref{lema2.1} $ S $ es sublineal. Tomamos el subconjunto:
		\[ 
		D = \{ (x_1, ..., x_N)\in V: \exists c \in C \quad \text{ tal que } \quad \forall i = 1,...N,\quad f_i(c) \leq x_i \}
		\]
		
		$ D $ es un subconjunto convexo de \vecSpace. Sean $ x, y \in D $, por ello, existen $ c_x, c_y \in C $ tales que $ f_i (c_x) \leq x_i  $ y $ f_i (c_y) \leq y_i \quad \forall i=1,...,N $. Dado $ \lambda \in [0,1] $, llamamos $ c := (1-\lambda)c_x + \lambda c_y $ que pertenece a $ C $ por ser este convexo. Veamos que $ c $ es el elemento necesario de $ C $ para que cualquier combinación convexa de $ x $ e $ y $ esté en $ D $. Así, para todo $ i =1,...,N  $:	
		\[
		f_i(c) = f_i((1-\lambda)c_x + \lambda c_y) \leq (1-\lambda)f(c_x) + \lambda f(c_y) \leq (1-\lambda)x_i + \lambda y_i 
		\]
		donde la primera desigualdad se debe a que las $ f_i $ son convexas y la segunda a que $ x,y \in D $. Por ello, $ (1-\lambda)x_i + \lambda y_i \in D , \quad \forall \lambda \in [0,1] $ por lo que $ D $ es convexo. Aplicando el Teorema de Mazur-Orlizc, existe $ L $ funcinal lineal sobre \vecSpace tal que $ L \leq S $ e $ \inf_D L = \inf_D S $. \\
		
		Nuevamente, por el lema \ref{lema2.1} tenemos que $ L = (t_1,...,t_N) \in \Delta_N$. \\
		
		Finalmente:
		\[
		\inf_D L = \inf_{c\in C} \left[ t_1 f(c) + ..+ t_N f(c) \right] = \inf_{C} \left[ t_1 f + ..+ t_N f \right]
		\]
		y
		\[
		\inf_D S = \inf_{c\in C} \left[ \max \{f_1(c), ..., f_N(c)\} \right] = \inf_C\left[ \max \{f_1, ..., f_N\}\right] 
		\]
		por lo que 
		\[ \inf_{C} \left[ t_1 f + ..+ t_N f \right] = \inf_C\left[ \max \{f_1, ..., f_N\}\right]  \] 
	\end{proof}

	\paragraph{}Enunciamos ahora el Teorema de la alternativa de Gordan en su versión convexa.
	
	\begin{teoremaBox}[Teorema de la Alternativa de Gordan-versión convexa]\label{Gordan}
		Sea $ C $ un subconjunto convexo de un espacio vectorial y sean $ f_1,...,f_N : C \longrightarrow \RR $ funciones convexas ($ N \geq 1 $). Entonces una, y solo una, de la siguientes afirmaciones se cumple:
		\begin{itemize}
			\item[i)] $ \exists \mathbf{t} \in \Delta_N $ tal que $ 0 \leq \inf_{C}  \sum_{i=1}^{N}{t_i f_i}$.
			\item[ii)] $ \exists c \in C $ que cumple $ \max \lbrace f_1(c), ..., f_N(c) \rbrace < 0 $.
		\end{itemize}
	\end{teoremaBox}
	\begin{proof}
		Si aplicamos el lema de Simons, lema \ref{Simons} a las funciones $ f_1,...,f_N $ obtenemos:
		
		\begin{equation*}
			\exists t \in \Delta_N \text{ : } \inf_{ C}\left[ \max \lbrace f_1, ..., f_N \rbrace \right] = \inf_{C} \sum_{i=1}^{N}t_i f_i
		\end{equation*}
		
		Supongamos en primer lugar que $ \alpha := \inf_{C}\left[ \max \lbrace f_1, ..., f_N \rbrace \right] \in \RR $. Planteamos la siguiente alternativa, cuyos casos son excluyentes:
		\begin{itemize}
			\item[a)] $ 0 \leq \alpha $: implica $ i) $ ya que $ \alpha = \inf_{C} \sum_{i=1}^{N}t_i f_i $ 
			\item[b)] $ \alpha < 0 $: este caso, por su parte, implica $ ii) $ ya que:
			
			\begin{equation*}
				\alpha = \inf_{C}\left[ \max \lbrace f_1, ..., f_N \rbrace \right] < 0 \Longleftrightarrow \exists c \in C  : \max \lbrace f_1(c), ..., f_N(c) \rbrace < 0 
			\end{equation*}  
		\end{itemize}
	
		Para finalizar, veamos cuando $ \alpha =-\infty $. En este caso, estamos en la misma situación que en b) por lo que solo se puede dar $ ii) $.
	\end{proof}

	\paragraph{} Destacamos las siguientes observaciones:
	
	\begin{observacion}
		Esta versión convexa del teorema implica la versión clásica del mismo.
	\end{observacion}

	Dados $ \{x_1,...x_N\}$ con $ x_i \in \RR^M , (M \geq 1 )$ $i=1,...,N$, la versión clásica del teorema nos aporta las siguientes alternativas:

	\begin{itemize}
		\item[i*)] $ \exists \mathbf{t} \in \Delta_N $ tal que $ 0 = \sum_{i=1}^{N}{t_i x_i}$.
		\item[ii*)] $ \exists y \in \RR^M $ tal que cumple $ \max_{i=1,...,N} \langle y, x_i \rangle < 0 $.
	\end{itemize}

	Para ello, basta aplicar la versión convexa del teorema a $ C := \RR^M $ y a las funciones $ f_1,...,f_N : C \longrightarrow \RR $ definidas por $ f_i(c):=\langle c,x_i \rangle , \forall i=1,...,N  $. Notar que las funciones $ f_1,...,f_N $ son lineales por lo que en particular son convexas. En este caso, la alternativa $ ii) $ implica $ ii*) $ ya que:
	
	\begin{equation*}
		\exists c \in C = \RR^M \text{ : } \max_{i=1,...,N} {\langle c,x_i \rangle}  =  \max_{i=1,...,N}f_i (c) < 0 
	\end{equation*}
	
	Por su parte, la alternativa $ i) $ nos da:
	\begin{equation*}
		\exists \mathbf{t} \in \Delta_N \text{ : } 0 \leq \inf_{c \in \RR^M}  \sum_{i=1}^{N}{t_i f_i(c) } = \inf_{c \in \RR^M} \sum_{i=1}^{N}{t_i\langle c,x_i \rangle} = \inf_{c \in \RR^M} \langle c, \sum_{i=1}^{N}{t_i x_i} \rangle 
	\end{equation*}
	
	Hemos obtenido por ello que : $0  \leq \inf_{c \in \RR^M} \langle c, \sum_{i=1}^{N}{t_i x_i} \rangle  $ lo que nos lleva a $ 0 \leq \langle c, \sum_{i=1}^{N}{t_i x_i} \rangle  $ para todo $ c \in \RR^M $. Usando la linealidad por la izquierda del producto escalar:
	\[
	0 \leq \langle -c, \sum_{i=1}^{N}{t_i x_i} \rangle \Longleftrightarrow 	0 \leq -\langle c, \sum_{i=1}^{N}{t_i x_i} \rangle 
	\Longleftrightarrow  \langle c, \sum_{i=1}^{N}{t_i x_i}\rangle \leq 0, \quad \forall c \in \RR^M
	\]
	
	Juntando ambas desigualdades obtenemos que $ 0 =  \langle c, \sum_{i=1}^{N}{t_i x_i}\rangle, \quad \forall c \in \RR^M $. Como la igualdad anterior se cumple para todo elemento de $ \RR^M $ entonces podemos deducir que $ \sum_{i=1}^{N}{t_i x_i} = 0 $	ya que $ \sum_{i=1}^{N}{t_i x_i} \in (\RR^M)^{\perp} = \{0\} $. Así pues, tenemos que
	\begin{equation*}
		\text{Se cumple }i) \Longleftrightarrow \exists t \in \Delta_N \text{ : }  0 = \sum_{i=1}^{N}{t_i x_i}
	\end{equation*} 	
	
	Es claro entonces que obtenemos $ i*) $.
	
	\begin{observacion}
		El lema de Simons(lema \ref{Simons}) y el Teorema de la Alternativa de Gordan (teorema \ref{Gordan}) son equivalentes.
	\end{observacion}

	\paragraph{} Ya hemos visto que el Lema de Simons implica el Teorema de la Alternativa de Gordan. Veamos que el recíproco también es cierto.  \\
	
	Llamamos $ \alpha := \inf_{C}\left[ \max \lbrace f_1, ..., f_N \rbrace \right] $. Si $ \alpha = -\infty $. Por el lema \ref{lema2.1} sabemos que $ \forall \ttt \in \Delta_N $ se cumple que $ \sum_{i=1}^{N} t_i f_i(c) \leq \max \lbrace f_1 (c), ... , f_N (c) \rbrace$ para todo $ c \in C$. Tomando ínfimos en C:
	\[
	\inf_C\left[ \sum_{i=1}^{N} t_i f_i \right] \leq \inf_C \left[ \max \lbrace f_1, ..., f_N \rbrace \right] = -\infty \Longrightarrow \inf_C\left[ \sum_{i=1}^{N} t_i f_i \right] = -\infty 
	\]
	
	y por ello $ \forall \ttt \in \Delta_N $ (en particular para uno) se cumple que
	
	\[
	\inf_C\left[ \sum_{i=1}^{N} t_i f_i \right] = \inf_C \left[ \max \lbrace f_1, ..., f_N \rbrace \right] \]

	Supongamos ahora que $ \alpha \in \RR $. Sean las funciones $ q_1, ..., g_N: C \longrightarrow \RR $ definidas como $ g_i = f_i - \alpha $ con $ i=1,...,N$. Veamos que las funciones $ g_1, ..., g_N $ son convexas como consecuencia de que $ f_1, ..., f_N $ lo son. Sean $ c_1,c_2 \in C $ y $ \lambda \in \left[0,1\right] $:
	\begin{equation*}
	\begin{split}
	g_i(\lambda c_1 + (1-\lambda) c_2) &= f_i(\lambda c_1 + (1-\lambda) c_2) - \alpha \\
	&\leq \lambda f_i(c_1) + (1-\lambda)f_i(c_2) - \alpha \\
	&= \lambda f_i(c_1) + (1-\lambda)f_i(c_2) - \lambda \alpha + (1-\lambda)\alpha \\
	&= \lambda( f_i(c_1) - \alpha ) + (1-\lambda) (f_i (c_2) - \alpha) \\
	&= \lambda g_i(c_1) + (1-\lambda) g_i (c_2)
	\end{split}
	\end{equation*}
	
	Obtenemos que $ g_i $ es convexa para todo $ i = 1, ..., N $. Si usamos el Teorema de la Alternativa de Gordan obtenemos que solo se pueden dar una y solo de las siguientes posibilidades:
	
	\begin{itemize}
		\item[i)] $ \exists \mathbf{t} \in \Delta_N $ tal que $ 0 \leq \inf_{C}  \sum_{i=1}^{N}{t_i g_i}$.
		\item[ii)] $ \exists c \in C $ que cumple $ \max \lbrace g_1 (c), ..., g_N(c) \rbrace < 0 $.
	\end{itemize}
	
	Razonemos que no se puede dar $ ii) $. Si fuese así, tendríamos que $ \exists c \in C $ tal que $ \max \lbrace g_1(c), ..., g_N(c) \rbrace  =  \max \lbrace f_1(c) - \alpha, ..., f_N(c) - \alpha \rbrace < 0 $. En particular, existiría un índice $ j \in {1,...,N} $ que cumpliría $ f_j(c) - \alpha < 0 \Longrightarrow f_j(c) < \alpha = inf_{C}\left[ \max \lbrace f_1, ..., f_N \rbrace \right] $. Esto es imposible por la propia definición de ínfimo. Por ello, afirmamos que $ \exists \mathbf{t} \in \Delta_N $ tal que $ 0 \leq \inf_{C}  \sum_{i=1}^{N}{t_i g_i}$. Desarrollando el sumatorio:
	\begin{equation*}
	\begin{split}
	0 &< \inf_{C}  \sum_{i=1}^{N}{t_i g_i} = \inf_{C}  \sum_{i=1}^{N}{t_i(f_i - \alpha)} = \inf_{C} \left[ \sum_{i=1}^{N}{t_i(f_i)} - \sum_{i=1}^{N}{t_i\alpha} \right]\\
	&= \inf_{C} \left[ \sum_{i=1}^{N}{t_i(f_i)} -\alpha \sum_{i=1}^{N}{t_i} \right] = \inf_{C} \left[ \sum_{i=1}^{N}{t_i(f_i)} - \alpha \right] = \inf_{C} \left[ \sum_{i=1}^{N}{t_i(f_i)}\right] - \alpha
	\end{split}
	\end{equation*}
	
	Por lo tanto:
	\[
	0 < \inf_{C} \left[ \sum_{i=1}^{N}{t_i(f_i)}\right] - \alpha \Longleftrightarrow \inf_{C}\left[ \max \lbrace f_1, ..., f_N \rbrace \right] = \alpha  < \inf_{C} \left[ \sum_{i=1}^{N}{t_i(f_i)}\right]
	\]
	
	El lema \ref{lema2.1} nos aporta la otra desigualdad y llegamos nuevamente a que $ \exists \ttt \in \Delta_N $ que cumple:
		\[
	\inf_C\left[ \sum_{i=1}^{N} t_i f_i \right] = \inf_C \left[ \max \lbrace f_1, ..., f_N \rbrace \right] \]
	

\section{Optimización con restricciones: Fritz John y Karush-Kuhn-Tucker}
		\newcommand{\barx}{\bar{x} }
		\newcommand{\barxx}{\bar{\mathbf{x}}}
		\newcommand{\lambdaa}{\boldsymbol{\mathbf{\lambda}}}
		\newcommand{\dd}{\textbf{\emph{d}}}

	
Concluimos este capítulo con un apartado dedicado a la optimización con restricciones. A pesar de tratarse de resultados no lineales, el	teorema de Gordan clásico basta para establecer los resultados fundamentales. Al igual que en el apartado anterior, seguimos el texto \cite{borwein}. En primer lugar, empezamos recordando la definición de derivada direccional. 
	\begin{definicion}
			Sea $ D \subset \RR^M $ con $ M \in \NN $ y sea la función $ g: D \longrightarrow  \RR$, definimos la derivada direccional de g en $\xx \in D $ en la dirección del vector $ \dd\in \RR^M $ como
			\[
			g'(\xx;\dd) = \lim_{t\rightarrow0}\frac{g(\xx+t\dd) - g(\xx)}{t}
			\]
			siempre y cuando el límite exista. Diremos que g es diferenciable en el sentido de Gâteaux en $ \xx $ si $ g'(\xx;\cdot):\RR^M \longrightarrow \RR $ es lineal y en ese caso escribimos $ \nabla g(\xx) = g'(\xx;\cdot) $, es decir, $ g'(\xx;\dd) = \langle \nabla g(\xx), \dd\rangle $ con $ \dd \in \RR^M $.
	\end{definicion}
	
	Dentro del contexto de este trabajo, cuando decimos que una función es diferenciable nos referimos a que lo es en el sentido de Gâteaux. A estas funciones también las llamaremos Gâteaux diferenciables. Destacamos que este concepto de diferenciabilidad es más débil que el de Fréchet. De hecho, si una función $ g $ es diferenciable en $ \xx_0 \in D$ en el sentido de Fréchet, y notamos su derivada como $ Dg(\xx_0) $ entonces $ g $ también es diferenciable en el sentido de Gâteaux en $ \xx_0 $ y además $ Dg(\xx_0) = \nabla g(\xx_0) $. El recíproco no es cierto tal y como mostramos en el siguiente ejemplo. Sea $ f:\RR^2 \longrightarrow \RR $ definida como:
	\[
	f(x,y) = \begin{cases}
	\displaystyle \frac{xy^3}{x^2+y^2}, & \mbox{si $ (x,y) \neq (0,0) $ } \\
	0, & \mbox{si $ (x,y) = (0,0) $ }
	\end{cases}.
	\]
	Sabemos que si una función es Fréchet diferenciable $ x_0 $, entonces es continua en $ x_0 $. Como $ f $ no es continua en $ (0,0) $, podemos afirmar que no es Fréchet diferenciable en dicho punto. Sin embargo, sí es Gâteaux diferenciable en $ (0,0) $. Para $ \dd =  (d_1, d_2) \in \RR^2 $,
	
\begin{equation*}
\begin{split}
f'((0,0); (d_1, d_2)) &= \lim_{t\rightarrow0}\frac{f((0,0)+t(d_1, d_2)) - f(0,0)}{t} \\
&=  \lim_{t\rightarrow0}\frac{\frac{td_1(td_2)^3}{(td_1)^2+(td_2)^2} - 0}{t} \\
&= \lim_{t\rightarrow0}\frac{t^4d_1d_2^3}{t^3d_1^2+t^3d_2^2} \\
&= \lim_{t\rightarrow0}\frac{td_1d_2^3}{d_1^2+d_2^2} \\
& = 0.
\end{split}
\end{equation*}
Como el límite existe y es lineal, la función es diferenciable en el sentido Gâteaux en $ (0,0) $. \\

A continuación, demostremos cómo se calcula la derivada de la función máximo de un número finito de funciones diferenciables, lo que nos será útil en posteriores resultados.
\bigskip
	\begin{proposicionBox}\label{dirDeriv}
		Sean $ D \subset \RR^M$ ($ M \in \NN $) , $ \barxx $ un punto del interior de $ D $ y sean $ g_1, ..., g_N : D \longrightarrow \RR $  funciones continuas y diferenciables en $ \barxx $ donde $ N \in \NN $. Definimos $ g:D \longrightarrow \RR $ como \[ g(\xx):=\max_{i=1,...,N}\{g_i(\xx)\} \] y el conjunto de índices $ K = \lbrace  i : g_i(\barxx) =  g(\barxx) \rbrace $. Entonces, para toda dirección $ \dd \in \RR^M $ la derivada direccional de $ g $ existe en todo $ \RR^M $ y viene dada por la siguiente expresión:
		\begin{equation}
			g'(\barxx;\dd) = \max_{i \in K} \langle \nabla g_i(\barxx), \dd \rangle, \quad \dd\in\RR^M.
		\end{equation}
	\end{proposicionBox}
	\begin{proof}
		Podemos suponer sin pérdida de generalidad que $ K = \{1, ..., N \} $, por facilitar la notación. Para cada $ i \in K $ tenemos la siguiente desigualdad:
		\begin{equation*}
			\liminf_{t\rightarrow0}\frac{g(\barxx+t\dd) - g(\barxx)}{t} \geq \liminf_{t\rightarrow0}\frac{g_i(\barxx+t\dd) - g_i(\barxx)}{t} = \langle \nabla g_i(\barxx), \dd \rangle.
		\end{equation*}	
		La primera desigualdad se deduce de la definición de $ g $ ya que es el máximo de las $ g_i $ e $ i \in K $ para $ i=1,...,N$ y la segunda igualdad de que todas las $ g_i $ son diferenciables en $ \barxx $ y por tanto existe el límite de la definición de derivada direccional y coincide con el límite inferior. Por lo tanto:
		\[
		\liminf_{t\rightarrow0}\frac{g(\barxx+t\dd) - g(\barxx)}{t} \geq \max_{i=1,...,N}\langle \nabla g_i(\barxx), \dd \rangle.
		\]
		Por otro lado, afirmamos que 
		\begin{equation*}
			\limsup_{t\rightarrow0}\frac{g(\barxx+t\dd) - g(\barxx)}{t} \leq \max_{i=1,...,N}\langle \nabla g_i(\barxx), \dd \rangle.
		\end{equation*}
		De lo contrario, existirían una sucesión $ \{t_n\}\rightarrow 0 $ y $ \varepsilon > 0 $ que cumplirían:
		\[
		\frac{g(\barxx+t_n \dd) - g(\barxx)}{t_n} \geq \max_{i=1,...,N}\langle \nabla g_i(\barxx), \dd \rangle + \varepsilon \quad \forall n \in \NN.
		\]
		Tomamos ahora una sucesión parcial $ \{t_{\sigma(n)}\}_{n \in \NN} $ con $ \sigma:\NN \longrightarrow \NN $ estrictamente creciente y $ j \in K $ un índice fijo tal que para todo $ k \in \{\sigma(n): \hspace{1mm} n\in \NN\} $ se cumple que $ g(\barxx + t_k \dd) = g_j (\barxx + t_k \dd)$. Tomando límite obtenemos que:
		\begin{equation*}
		\begin{split}
		\limsup_{t\rightarrow0}\frac{g(\barxx+t\dd) - g(\barxx)}{t} &= 	\limsup_{t\rightarrow0}\frac{g_j(\barxx+t\dd) - g_j(\barxx)}{t} \\
		&= \langle \nabla g_j(\barxx), \dd \rangle \\ &\geq \max_{i=1,...,N}\langle \nabla g_i(\barxx), \dd \rangle + \varepsilon,
		\end{split}
		\end{equation*}
		lo cual, es imposible. Finalmente, hemos obtenido que:
		\[
		\limsup_{t\rightarrow0}\frac{g(\barxx+t\dd) - g(\barxx)}{t} \leq \max_{i=1,...,N}\langle \nabla g_i(\barxx), \dd \rangle \leq 	\liminf_{t\rightarrow0}\frac{g(\barxx+t\dd) - g(\barxx)}{t}.
		\]
		Como el límite inferior es siempre menor o igual que el superior concluimos que ambos coinciden y por lo tanto existe el límite y además:
		\[
		\lim_{t\rightarrow0}\frac{g(\barxx+t\dd) - g(\barxx)}{t} = g'(\barxx;\dd) = \max_{i=1,...,N}\langle \nabla g_i(\barxx), \dd \rangle.
		\]
	\end{proof}
\bigskip
	Nuestro objetivo ahora es encontrar soluciones a problemas del siguiente tipo:
	
		\begin{equation}\label{probMin}
		\begin{cases}
		\displaystyle\inf_{\xx\in D} f(\xx)\\
		\begin{split}
		\text{s.a } g_1(\xx) &\leq 0 \\
		&\vdots \\
		g_N(\xx) &\leq 0
		\end{split}
		
		\end{cases} 
		\end{equation}
		donde $ D \subset \RR^M$, $ f  $ es la función objetivo y las restricciones $ g_i $ con $ i =1,\dots, N $ son funciones reales definidas en $ D $ y continuas. Si un punto satisface todas las restricciones diremos que es \textit{factible}  y como consecuencia llamamos \textit{región factible} al conjunto de todos los puntos factibles. Para un punto factible $ \barxx $ definimos el \textit{conjunto activo} como $ I(\barxx) = \{i: g_i(\barxx) = 0\}$. El papel que juega este conjunto de índices no es trivial, apareciendo en los teoremas de Fritz John y Karush-Kuhn-Tucker como los únicos necesarios. Este trabajo ya se ha realizado en la proposición \ref{dirDeriv}. Para este problema y asumiendo que $ \barxx \in D $, llamamos \textit{vector de multiplicadores de Lagrange para $ \barxx $} a $ \lambdaa \in (\RR^N)^+ $ si $ \barxx $ es un punto crítico de:
		\[
		L(\xx;\lambdaa) = f(\xx) + \sum_{i=1}^{N} \lambda_i g_i(\xx),
		\]
		es decir, se cumple que $ f,g_1,\dots,g_N $ son diferenciables y:
		\[
		\nabla f(\barxx) + \sum_{i=1}^{N} \lambda_i \nabla g_i(\barxx) = 0.
		\]
		Además, $ \lambda_i = 0 $ si $ i \notin I(\barxx) $, por tanto
		\[
		\nabla f(\barxx) + \sum_{i \in I(\barxx)} \lambda_i \nabla g_i(\barxx) = 0.
		\] 
		También, se dice que un punto $ \barxx \in D $ es un \textit{mínimo global} del problema \ref{probMin} si es un punto factible y además $ f(\barxx) \leq f(\xx) $ para todo $ \xx \in D $ factible. Por otro lado, diremos que $ \barxx \in D $ es un \textit{mínimo local} del problema \ref{probMin} si es un punto factible y existe $ U $ entorno de $ \barxx $ tal que $ f(\barxx) \leq f(\xx) $ para todo $ \xx \in D\cap U $ factible. Los resultados que exponemos a continuación se centran en mínimos locales.
		\bigskip
		\begin{teoremaBox}[Teorema de Fritz John]\label{FritzJohn}
			Supongamos que el problema (\ref{probMin}) tiene un mínimo local en $ \barxx \in D $. Si las funciones $ f, g_i $ con $ i \in I(\barxx) $ son diferenciables en $ \barxx $ entonces existen $ \lambda_0, \lambda_i \in \RR^+ $ para $ i \in I(\barxx) $, no todas cero, que satisfacen:
			\[
			\lambda_0 \nabla f(\barxx) + \sum_{i \in I(\barxx)} \lambda_i \nabla g_i(\barxx) = 0.
			\]
		\end{teoremaBox}
		\begin{proof}
			Consideramos la función
			\[
			g(\xx) = \max \{ f(\xx) - f(\barxx),\text{ } g_i(\xx) : i \in I(\barxx)\}.
			\]
			definida para todo punto $ \xx \in D$ y perteneciente a la región factible del problema \eqref{probMin}. Como $ \barxx $ es un mínimo local de dicho problema también lo es de $ g $. Supongamos que no fuera así, es decir, existe $ U $ entorno de $ \barxx $ existe un punto $ \xx_0 \in D \cap U $  factible tal que $ g(\xx_0) < g(\barxx) $. Como $ \barxx $ es un mínimo local del problema se tiene que $ f(\xx_0)-f(\barxx) > 0 $. En ese caso, como $ \xx $ es factible (y por ello cumple las restricciones), se tiene que 
			\begin{equation*}
			\begin{split}
			g(\xx_0) & = \max \{ f(\xx_0) - f(\barxx),\text{ } g_i(\xx_0) : i \in I(\barxx)\} \\ 
			&=  f(\xx_0) - f(\barxx)\\ &
			> 0. 
			\end{split}
			\end{equation*}
			Esto es imposible ya que se tendría que $ 0 < g(\xx_0) < g(\xx) = 0 $. Por la proposición \ref{dirDeriv} tenemos que para toda dirección $ \dd \in \RR^M $ se cumple:
			\[
			g'(\barxx;\dd) = \max \{ \langle \nabla f(\barxx), \dd\rangle , \langle \nabla g_i(\barxx),\dd \rangle : i \in I(\barxx)\} \geq 0,
			\]
			ya que si $ g'(\barxx;\dd) < 0 $, para todo $ t > 0 $ suficientemente pequeño tendríamos que $ g(\barxx + t\dd) < g(\barxx) $ lo que contradice que $ g $ alcanza un mínimo local en $ \barxx $. \\
			
			Por lo tanto, el sistema 
			\begin{equation*}
			\begin{cases}
			\langle \nabla f(\barxx), \dd\rangle  < 0 \\
			\langle \nabla g_i(\barxx),\dd \rangle < 0\text{ con } i \in I(\barxx)
			\end{cases}
			\end{equation*}
			no tiene solución (para ninguna dirección) ya que al menos uno es no negativo. Si aplicamos el Teorema de la Alternativa de Gordan en su versión clásica, teorema \ref{GordanClasic}, vemos que solo se puede dar la alternativa i*) y en ese caso obtenemos que:
			\[
			 \exists \ttt = (t_0, ..., t_{R})\in \Delta_{R+1}  \text{ tal que }0 = t_0 \nabla f(\barxx) + \sum_{i \in I(\barxx)}  t_i \nabla g_i (\barxx),
			 \]
			con $ R $ el cardinal del conjunto $ I(\barxx) $. La demostración concluye llamando $ \lambda_0 = t_0 $ y $ \lambda_i = t_i $ con $ i \in I(\barxx) $.
		\end{proof}
	\bigskip
		\paragraph{}El teorema de Fritz John nos aporta una gran desventaja y es que es posible que $ \lambda_0 = 0 $ por lo que la función objetivo es independiente de las restricciones y no influye en la información que se da. Por ello, necesitamos imponer algunas condiciones de regularidad extra. En esta situación diremos que se cumple la \textit{condición de Mangasarian-Fromovitz} si existe una dirección $ \dd_0 \in \RR^M $ que satisface que $ \langle \nabla g_i(\barxx),\dd_0 \rangle < 0 $ para todo índice $ i \in I(\barxx)$. Enunciamos ahora otro teorema clásico que soluciona el problema que acabamos de comentar.
		
		\bigskip
		\begin{teoremaBox}[Karush-Kuhn-Tucker]
		Supongamos que el problema (\ref{probMin}) tiene un mínimo local en $ \barxx \in D $. Si las funciones $ f, g_i $ con $ i \in I(\barxx) $ son diferenciables en $ \barxx $ y se cumple la condición de Mangasarian-Fromovitz entonces existe un vector de multiplicadores de Lagrange para $ \barxx $.
		\end{teoremaBox} 
	
		\begin{proof}
		Del teorema \ref{FritzJohn} de Fritz John obtenemos que:
		\[
		\exists \ttt = (t_0, ..., t_{R})\in \Delta_{R+1}  \text{ tal que }0 = t_0 \nabla f(\barxx) + \sum_{i \in I(\barxx)}  t_i \nabla g_i (\barxx)
		\]		
		con $ R $ el cardinal de $ I(\barxx) $. Si multiplicamos escalarmente la igualdad por $ \dd_0 $ (dirección del espacio vectorial que nos aporta la condición de  Mangasarian-Fromovitz), obtenemos:
		\[
		0 = t_0 \langle \nabla f(\barxx), \dd_0 \rangle + \sum_{i \in I(\barxx)}  t_i \langle \nabla g_i (\barxx), \dd_0 \rangle.
		\]
		Entonces $ t_0 \neq 0$. Razonemos por reducción al absurdo. Si no fuese así, tendríamos que
		\[
		0 = \sum_{i \in I(\barxx)}  t_i \langle \nabla g_i (\barxx), \dd_0 \rangle.
		\]
		Al tener $ \ttt \in \Delta_{R+1} $ se cumple que todas sus componentes son no negativas y $ \sum_{i \in I(\barxx)}  t_i = 1 $ (estamos suponiendo que $ t_0 = 0 $ por lo que no influye en la suma de la definición de $ \Delta_{R+1} $) por lo que algún término es distinto de 0. Tenemos garantizado que $  \langle g_i (\barxx), \dd_0 \rangle < 0 \quad \forall i \in I(\barxx) $. Así tendríamos que:
			\[
		0 = \sum_{i \in I(\barxx)}  t_i \langle \nabla g_i (\barxx), \dd_0 \rangle < 0,
		\]
		lo cual es imposible. Por ello, concluimos que $ t_0 \neq 0 $. La demostración finaliza tomando $ \lambda_i = t_i / t_0 $ para $ i \in I(\barxx) $.
		\end{proof}
		\bigskip
		
La condición de Mangasarian-Fromovitz no es prescindible en el teorema de Karush-Kuhn-Tucker, tal y como pone de manifiesto este sencillo ejemplo.

\begin{equation*}
\begin{cases}
f(x,y) = 2x\\
\begin{split}
\text{s.a } g_1(x,y) &= 2y-5x^3\\
g_2(x,y) &= -y
\end{split}
\end{cases}.
\end{equation*}
Es claro que el mínimo se alcanza en $ \barxx = (0,0) $ y tenemos que $ \nabla f(0,0) = (2,0) $, $ \nabla g_1(0,0) = (0,2) $, $ \nabla g_2(0,0) = (0,-1) $ e $ I(\barxx) = \{1,2\} $. La condición de Mangasarian-Fromovitz no se cumple. Si existiese una dirección $ d = (d_1, d_2) \in \RR^2 $ que la cumpliese se deberían satisfacer las siguientes condiciones simultáneamente

\begin{equation*}
\begin{cases}
\begin{split}
\langle \nabla g_1(0,0),(d_1, d_2) \rangle < 0 \Longleftrightarrow  \langle (0,2),(d_1, d_2) \rangle < 0 \Longleftrightarrow 2d_2 &< 0\\
\langle \nabla g_2(0,0),(d_1, d_2) \rangle < 0 \Longleftrightarrow \langle (0,-1),(d_1, d_2) \rangle < 0 \Longleftrightarrow d_2 &> 0
\end{split}
\end{cases} 
\end{equation*}
lo que es imposible. Además, si $ \lambda, \mu \geq 0$ fuesen multiplicadores de Lagrange para $ (0,0) $, entonces 
\[
\nabla f(0,0) + \lambda \nabla g_1(0,0) + \mu \nabla g_2(0,0) = (0,0)
\]
\[
\big\Updownarrow
\]
\[
(2,2\lambda - \mu) = (0,0),
\]
que no se puede dar. \\

Finalmente, notamos que los teoremas de Fritz John y de Karush-Kuhn-Tucker que acabamos de demostrar también se pueden considerar consecuencia del lema de Farkas. Destacar también que en el caso convexo e imponiendo ciertas condiciones de regularidad podemos probar resultados análogos sin hipótesis de diferenciabilidad \cite{borwein}.

\chapter{Minimax}
\newcommand{\topSpace}{X}
\newcommand{\topSpaceY}{Y}

\paragraph{}En esta sección llegaremos a otro de los resultados clave del trabajo. Será uno de los denominados teoremas Minimax. A rasgos generales y a modo introductorio, podemos decir que un teroema Minimax es un resultado que afirma, bajo ciertas hipótesis, que:
\[
\inf_{y \in Y} \sup_{x \in X} f(x,y) = \sup_{x \in X} \inf_{y \in Y} f(x,y),
\] 
donde $ X \text{ e } Y$ son subconjuntos de un espacio vectorial y $ f: X \times Y \longrightarrow \RR $. Obviamente, esta igualdad no es cierta en general tal y como mostramos en el siguiente ejemplo. Definimos $ f:\{0,1\} \times \{0,1\} \longrightarrow \RR$ como:
\[
f(x,y) = \begin{cases}
0, & \mbox{si $ x=y $ } \\
1, & \mbox{si $ x\neq y$ }
\end{cases}.
\]
Por un lado tenemos
\[
\inf_{ y \in Y}\sup_{x \in X} f(x,y) = \min_{ y \in Y}\max_{x \in X} f(x,y) = \min_{ y \in Y}\{1\} = 1,
\]
y por otro
\[
\sup_{x \in X} \inf_{ y \in Y}f(x,y) = \max_{x \in X}\min_{ y \in Y}f(x,y) = \min_{ y \in Y}\{0\} = 0.
\]
Es claro, como muestra el ejemplo, que la desigualdad 
\[
\inf_{y \in Y} \sup_{x \in X} f(x,y) \geq \sup_{x \in X} \inf_{y \in Y} f(x,y)
\] 
siempre se da ya que
\[
\sup_{x \in X} f(x,y) \geq f(x,y) \geq \inf_{y \in Y}f(x,y) .
\]
Por lo ello, algunas veces los teoremas Minimax solo nos aportan la otra desigualdad necesaria. \\

Antes de continuar, exponemos la siguiente definición que aparecerá posteriormente en el teorema. Se trata de una propiedad más débil que la continuidad para funciones reales. 
\begin{definicion}
Sea $ \topSpace $ un espacio topológico. Decimos que $ f:\topSpace \longrightarrow \RR $ es superiormente semicontinua si para todo $ r \in \RR $ se cumple que el conjunto $ \lbrace x \in \topSpace : f(x) \geq r \rbrace $ es cerrado.
\end{definicion}

Por ejemplo, la función $ f:\RR \longrightarrow \RR $ dado por 
\[
f(x) = \begin{cases}
0, & \mbox{si $ x < 0 $ } \\
1, & \mbox{si $x \geq 0$ }
\end{cases}
\]
es superiormente semicontinua. \\

En estos momentos nos encontramos en condiciones de enunciar y demostrar nuestro teorema Minimax.
\begin{teoremaBox}
Sean $ \topSpace, \topSpaceY $ subconjuntos convexos de espacios vectoriales (no tienen que ser el mismo) tal que $ \topSpace $ está dotado de una topología que lo hace compacto.Supongamos además que $ f:  \topSpace \times \topSpaceY \longrightarrow \RR $ es:
\begin{itemize}	
\item[i)] cóncava y superiormente semicontinua en $ \topSpace $ y
\item[ii)] convexa en $ \topSpaceY $.
\end{itemize}
Entonces:
\begin{equation}\label{eqMinMax}
\inf_{y \in Y} \max_{x \in X} f(x,y) = \max_{x \in X} \inf_{y \in Y} f(x,y).
\end{equation}
\end{teoremaBox}
\begin{proof}
En primer lugar, podemos escribir máximo en ambos casos en vez de supremo ya que $ f $ es superiormente semicontinua en $ \topSpace $, por ello $ \inf_{y \in Y} f(x,y) $ también lo es (referenciar) y $ \topSpace $ es compacto (referenciar). Como hemos explicado anteriormente, solo necesitamos la desigualdad 
\begin{equation}\label{desAux}
\inf_{y \in Y} \max_{x \in X} f(x,y) \leq \max_{x \in X} \inf_{y \in Y} f(x,y).
\end{equation}  Definimos $ \alpha := \inf_{y \in Y} \max_{x \in X} f(x,y) $. Primero vamos a reescribir el resultado a probar. La desigualdad (\ref{desAux}) es equivalente a:\\
\[
\exists x_0 \in X:\text{ }\alpha \leq \inf_{y \in Y} f(x_0,y),
\]
ya que si existe un elemento en $ X $ que lo cumpla el máximo también lo cumplirá y recíprocamente. O lo que es lo mismo:
\[
\exists x_0 \in X:\text{ }y \in \topSpaceY \Longrightarrow \alpha \leq f(x_0,y).
\]
Entonces, tenemos que:
\[
\bigcap_{y\in \topSpaceY}\{x \in \topSpace: \alpha \leq f(x, y) \} \neq \emptyset,
\]
debido a que al menos $ x_0 $ está en dicha intersección. Como $ f $ es superiormente semicontinua en $ \topSpace $ estamos ante una intersección de cerrados. Usando la propiedad de intersección finita ($ \topSpace $ es compacto) obtenemos que:
\[
N\in\NN, \text{ }y_1,\dots,y_N \in \topSpaceY \Longrightarrow \bigcap_{i=1}^{N}\{x \in \topSpace: \alpha \leq f(x, y_i) \} \neq \emptyset.
\]
\[
\big\Updownarrow
\]
\[
N\in\NN, \text{ }y_1,\dots,y_N \in \topSpaceY \Longrightarrow \exists x_0\in\topSpace:\alpha \leq \min_{i=1\dots,N }f(x_0,y_i).
\]
\[
\big\Updownarrow
\]
\[
N\in\NN, \text{ }y_1,\dots,y_N \in \topSpaceY \Longrightarrow \alpha \leq \max_{x \in X} \min_{i=1\dots,N }f(x,y_i).
\]
En efecto, sean \vecN{y}$ \in \topSpaceY$ con $ N \in \NN $. Aplicamos el lema de Simons, lema (\ref{Simons}), tomando $ C := \topSpace $ y $ f_i: \topSpace \longrightarrow \RR $ definidas como $ f_i(x):=-f(x,y_i) $ para $ i=1,\dots,N $. Como $ f $ es cóncava respecto a $ X $ tenemos que las $ f_i $ son convexas para X con $  i=1,\dots,N $. De este modo, existe $ \ttt \in \Delta_N$ tal que
\[
\inf_{x \in \topSpace}\left[ \max_{i=1\dots,N } \{f_i(x)\}\right] = \inf_{x \in \topSpace} \left[ \sum_{i=1}^{N} t_i f_i(x) \right].
\] 
Si ponemos la igualdad en función de $ f $ y recordando que alcanza el supremo en $ X $:
\[
\inf_{x \in \topSpace}\left[ \max_{i=1\dots,N } \{-f(x,y_i)\}\right] = \inf_{x \in \topSpace} \left[ \sum_{i=1}^{N} t_i (-f(x,y_i)) \right]. 
\] 
\[
\big\Downarrow
\]
\[
\max_{x \in \topSpace}\left[ \min_{i=1\dots,N } \{f(x,y_i)\}\right] = \max_{x \in \topSpace} \left[ \sum_{i=1}^{N} t_i f(x,y_i) \right]. 
\] 
Al ser $ f $ convexa en $ \topSpaceY $:
\[
\max_{x \in \topSpace}\left[ \min_{i=1\dots,N } \{f(x,y_i)\}\right] \geq \max_{x \in \topSpace} \left[ f(x,\sum_{i=1}^{N} t_i y_i) \right] \geq \inf_{y \in Y} \max_{x \in \topSpace} f(x,y)= \alpha.
\]
Hemos probado entonces la desigualdad (\ref{desAux}), al ser la otra desigualdad sabida, podemos concluir que
\[
\max_{x \in X} \inf_{y \in Y} f(x,y) = \inf_{y \in Y} \max_{x \in X} f(x,y).
\]
\end{proof}


\chapter{Teorema de Separación}

En esta sección introducimos algunos resultados sobre separación. En general, estos nos aportan herramientas para poder concluir cuándo dos subconjuntos convexos pueden ser separados mediante un hiperplano. En la siguiente imagen vemos un ejemplo sobre la situación en la que nos encontramos:

\begin{figure}[h!]
\begin{center}
\begin{tikzpicture}[thick,fill opacity=0.5]
\filldraw[fill=red][rotate = 30] (0:4cm) ellipse (8mm and 5 mm);
\filldraw[fill=green] (1cm:1cm) circle (12mm);
\draw (-1,3) -- (5,0);
%\node at (0.9cm,0.6cm) {\large A};
%\node at (3.5cm,2.1cm) {\large B};
\end{tikzpicture}
\end{center}
\caption{Situación de teoremas de separación}
\end{figure}

Expones un resultado sencillo en el que solo involucramos un conjunto

\begin{teoremaBox}\label{sep1}
Dado $ N \in \NN $ y sea $ C \subset \RR^N $ convexo y definimos
\[
\delta := \inf\{ \norm{c}: c \in C \}.
\]
Entonces, existe $ x_0 \in \RR^N $ tal que si $ c \in C $ se cumple que $ \delta \leq \langle x_0,c\rangle $.
\end{teoremaBox}
\begin{proof}
En primer lugar, vamos a reescribir la tesis del teorema. Queremos ver que:
\[
\exists x_0 \in \RR^N : c \in C \Longrightarrow \delta \leq \langle x_0,c\rangle.
\]
\[
\big\Updownarrow
\]
\[
\exists \alpha > 0, \exists x_0 \in \alpha B_{\RR^N} : c \in C \Longrightarrow \delta \leq \langle x_0,c\rangle.
\]
\[
\big\Updownarrow
\]
\[
\exists \alpha > 0, \exists x_0 \in \alpha B_{\RR^N} : \delta \leq \inf_{c\in C}\langle x_0,c\rangle.
\]
\[
\big\Updownarrow
\]
\[
\exists \alpha > 0 : \delta \leq \max_{x\in \alpha B_{\RR^N}}\inf_{c\in C}\langle x,c\rangle.
\]
Llamamos $ \topSpace := \alpha B_{\RR^N}$ que es compacto y conexo, $ \topSpaceY:= C$ convexo y $ f $ función a la real y continua con valores en $ \topSpace \times \topSpaceY $ definida como $ f(x,y):=\langle x,y \rangle $ (al ser $ f $ continua, en particular es superiormente semicontinua). Aplicamos el teorema Minimax, teorema (\ref{MinMax}), y obtenemos que probar la última desigualdad es equivalente a probar que
\[
\exists \alpha > 0 : \delta \leq \inf_{c\in C}\max_{x\in \alpha B_{\RR^N}}\langle x_0,c\rangle.
\]
Tenemos que $ \max_{x\in \alpha B_{\RR^N}}\langle x,c\rangle \leq \alpha \norm{c} $ por la desigualdad de Cauchy–Schwarz. Así, debemos demostrar que
\[
\exists \alpha > 0 : \delta \leq \inf_{c\in C} \alpha \norm{c} \leq \alpha  \inf_{c\in C}\norm{c} = \alpha \delta.
\]
Pero esta desigualdad es cierta tomando, por ejemplo, $ \alpha = 1 $.
\end{proof}

Ahora, vamos a hacer una generalización de este resultado.
\begin{teoremaBox}\label{separacion1}
Sean $ A,B $ subconjuntos convexos de $ \RR^N $ para $ N \in \NN $ tal que $ A $ es cerrado, $ B $ es compacto y $ A \cap B = \emptyset$. Entonces existe $ x_0 \in \RR^N $ tal que
\[
\sup_{a \in A} \langle x_0,a\rangle < \inf_{b\in B} \langle x_0,b\rangle.
\]
\end{teoremaBox}
\begin{proof}
En primer lugar, veamos que $ \dist(A,B) > 0 $ donde la distancia viene dada por $\dist(A,B) = \inf\{ \norm{a-b} : a\in A, b\in B\}$. Para ello, razonemos por reducción al absurdo. Suponemos $ \dist(A,B) = 0 $, entonces existe una sucesión $ \{u_n\}_{n\in\NN} \subset A-B $ tal que $ \{u_n\}_{n\in\NN} \longrightarrow 0 $. Para $ n\in\NN $ tenemos que $ u_n = a_n - b_n $ con $ a_n \in A $ y $ b_n \in B $. De este modo obtenemos las sucesiones $ \{a_n\}_{n\in\NN} \subset A $ y $ \{b_n\}_{n\in\NN} \subset B $. Como $ B $ es compacto, existe una sucesión parcial convergente, es decir, existe $ \sigma: \NN \longrightarrow \NN $ estrictamente creciente tal que $ \{b_{\sigma(n)}\}_{n\in\NN} \longrightarrow b$ con $ b \in B $. Así,
\[
\norm{a_{\sigma(n)}-b} = \norm{a_{\sigma(n)} - b_{\sigma(n)} + b_{\sigma(n)} - b} \leq  \norm{a_{\sigma(n)} - b_{\sigma(n)}} + \norm{b_{\sigma(n)} - b}.
\]
Entonces tenemos que $ \norm{a_{\sigma(n)}-b} \longrightarrow 0 $ ya que $ \{b_{\sigma(n)}\}_{n\in\NN} \longrightarrow b$ y como $ \norm{a_n - b_n}\longrightarrow 0$ también se cumple que $ \norm{a_{\sigma(n)} - b_{\sigma(n)}}\longrightarrow 0$. Llegamos a que $ \{a_{\sigma(n)}\}_{n\in\NN} \longrightarrow b$. Al ser $ A $ cerrado se debe cumplir que $ b \in A $ lo cual es imposible ya que $ A \cap B = \emptyset$. \\

Ahora, aplicamos el teorema (\ref{sep1}) a $ C:= B-A $. Notar que $ C $ es convexo por serlo $ A $ y $ B $. Obtenemos entonces que existe $ x_0 \in \RR^N $ tal que si $ c \in C $ se cumple que:
\[
\delta = \inf_{c \in C} \norm{c} \leq \langle x_0, c\rangle.
\]
Por la definición de $ C $, se tiene que:
\[
\delta = \inf\{ \norm{b-a} : a\in A, b\in B\} = \dist(A,B) > 0.
\] 
Del mismo modo, $ c = b-a $ para todo $ c \in C $ con $ a \in A $ y $ b \in B $. Así.
\[
\exists x_0 \in \RR^N: a\in A, b\in B \Longrightarrow 0 <\delta \leq \langle x_0, b-a\rangle = \langle x_0, b\rangle - \langle x_0,a\rangle.
\]
\[
\big\Updownarrow
\]
\[
\exists x_0 \in \RR^N: a\in A, b\in B \Longrightarrow \langle x_0, a\rangle + \delta \leq \langle x_0, b\rangle ,\text{ con } \delta > 0.
\]
\[
\big\Updownarrow
\]
\[
\exists x_0 \in \RR^N \Longrightarrow \sup_{a\in A}\langle x_0, a\rangle + \delta \leq \inf_{b \in B}\langle x_0, a\rangle, \text{ con } \delta > 0.
\]
\[
\big\Updownarrow
\]
\[
\exists x_0 \in \RR^N : \sup_{a\in A}\langle x_0, a\rangle < \inf_{b \in B}\langle x_0, a\rangle.
\]
Por ello, queda probado el teorema.
\end{proof}

El teorema de separación anterior podemos escribirlo no solo en $ \RR^N $ sino que se puede generalizar a cualquier espacio normado finito dimensional. A continuación, exponemos otro teorema de separación válido solo en espacios normados reales pero que con unas hipótesis más débiles podemos obtener una tesis parecida.

\begin{teoremaBox}
Sea $ N \in \NN $ y $ A \text{ y } B$ dos subconjuntos convexos y disjuntos de $ \RR^N $. Entonces existe $ x_0 \in \RR^N \setminus \{0\} $ tal que
\[
\sup_{a \in A} \langle x_0,a\rangle \leq \inf_{b\in B} \langle x_0,b\rangle.
\]
\end{teoremaBox}
\begin{proof}
Al igual que en el teorema anterior, podemos reducir la prueba al caso en que $ C $ es un subconjunto de $ \RR^N $ convexo de forma que $ 0 \notin C $, demostrando que:
\[
\exists x_0 \in \RR^N \setminus \{0\}: \quad \sup_{c \in C} \langle x_0, c \rangle \leq 0,
\]
equivalentemente, 
\[
\exists x_0 \in S_{\RR^N} \setminus \{0\}: \quad \sup_{c \in C} \langle x_0, c \rangle \leq 0.
\]
Pero esta afirmación no es más que:
\[
\bigcap_{c \in C} \{x \in S_{\RR^N}: \langle x, c \rangle \leq 0 \} \neq \emptyset.
\]
Usando que $ S_{\RR^N} $ es compacta y por ello tenemos la propiedad ed intersección finita, equivale a que:
\[
\emptyset \neq C_0 \subset C \text{ finito} \Longrightarrow \bigcap_{c \in C_0} \{x \in S_{\RR^N}: \langle x, c \rangle \leq 0 \} \neq \emptyset.
\]
\[
\big\Updownarrow
\]
\[
M \in \NN, \text{ } \{c_1,\dots, c_M\} \subset C\Longrightarrow \exists x_0 \in S_{\RR^N}: \quad \max_{i=1,\dots,M} \langle x_0, c_i \rangle \leq 0.
\]
Esta condición se cumple, ya que si $ M \geq 1 $ y $ \{c_1,\dots, c_M\} \subset C $, entonces, por ser $ C $ convexo se cumple que $ \mathrm{co}\{c_1,\dots, c_M \} \subset C $ y como $ 0 \notin C $, entonces $  0 \notin \mathrm{co}\{c_1,\dots, c_M \} $. Por tanto, la versión clásica del Teorema de Gordan, teorema (\ref{GordanClasic}):
\[
\exists z_0 \in \RR^N \setminus \{0\}: \quad \max_{i=1,\dots,M} \langle z_0, c_i \rangle \leq 0.
\]
En particular, $ z_0 \neq 0 $ y tomando $ x_0 = \frac{z_0}{\norm{z_0}} $ queda probado el enunciado.
\end{proof}

Destacamos ahora lo siguiente:
\begin{observacion}
El Teorema de separación (\ref{separacion1}) implica el Teorema de Gordan, teorema (\ref{GordanClasic}).
\end{observacion}

En efecto, dados $ \{x_1,\dots, x_M \} \subset \RR^N $ con $ M, N \in \NN  $ del enunciado de la alternativa de Gordan planteamos las siguientes alternativas excluyentes:
\begin{enumerate}
\item Si $ 0 \in \mathrm{co}\{x_1,\dots, x_M \} $ entonces es claro que se cumple la alternativa $ i*) $.
\item Si $ 0 \notin \mathrm{co}\{x_1,\dots, x_M \} $ entonces $ \{0\} \cap \mathrm{co}\{x_1,\dots, x_M \} = \emptyset $. Al ser ambos conjuntos son compactos usamos el teorema de separación y obtenemos que:
\[
\exists y \in \RR^M: \quad \sup_{x \in \mathrm{co}\{x_1,\dots, x_M \}} \langle y, x \rangle < 0.
\]
\[
\big\Updownarrow
\]
\[
\exists y \in \RR^M: \quad \max_{i = 1,\dots, M} \langle y, x_i \rangle < 0.
\]
Esta última equivalencia es válida ya que como
\[ \{x_1,\dots, x_M \} \subset \mathrm{co}\{x_1,\dots, x_M \}  \Longrightarrow \max_{i = 1,\dots, M} \langle y, x_i \rangle \leq \sup_{x \in \mathrm{co}\{x_1,\dots, x_M \}} \langle y, x \rangle.
\]
\end{enumerate}

\chapter{Lema de Farkas}

En la sección anterior hemos vimos como uno de los teoremas de separación implica el único teorema de la alternativa que hemos visto hasta el momento. Ahora, vamos a deducir de manera parecida otro teorema de la alternativa. Antes exponemos la siguiente definición.

\begin{definicion}
	Dados $ \{x_1,\dots, x_M \} \subset \RR^N  $ con $ M, N \in \NN $, llamamos cono generado por $ \{x_1,\dots, x_M \} $ al conjunto convexo y cerrado de $ \RR^N $ dado por:
	\begin{equation*}
	\mathrm{cone}\{x_1,\dots, x_M \} := \left\lbrace \sum_{j=1}^{N}{\mu_j x_j } : \text{ } \mu_1,...,\mu_N \geq 0 \right\rbrace .
	\end{equation*}
\end{definicion}

Veamos que efectivamente $ \mathrm{cone}\{x_1,\dots, x_M \} $ es convexo y compacto:

\begin{itemize}
	\item  Convexo: tenemos que comprobar que dados $ \ttt, \sss \in \mathrm{cone}\{x_1,\dots, x_M \} $ y $ \lambda \in \lbrack 0,1 \rbrack $ entonces $ \lambda\ttt + (1-\lambda)\sss \in \mathrm{cone}\{x_1,\dots, x_M \} $. En efecto:

	\begin{equation*}
	\begin{split}
		\lambda\ttt + (1-\lambda)\sss &=   \lambda \sum_{j=1}^{N}{\mu_j^t x_j } + (1-\lambda)\sum_{j=1}^{N}{\mu_j^s x_j } \\
		&= \sum_{j=1}^{N}{\lbrack \lambda \mu_j^t x_j  + (1-\lambda)\mu_j^s x_j \rbrack} \\
		&= \sum_{j=1}^{N}{\lbrack \lambda \mu_j^t  + (1-\lambda)\mu_j^s \rbrack x_j}.
		\end{split}
		\end{equation*}
Como $ \mu_j^t \text{ y } \mu_j^s $ son no negativos, entonces $ \lambda \mu_j^t  + (1-\lambda)\mu_j^s  $ también es una cantidad no negativa para $ j=1,\dots,M $. Así, podemos concluir que $ \lambda\ttt + (1-\lambda)\sss \in \mathrm{cone}\{x_1,\dots, x_M \} $ y por tanto es un subconjunto convexo.

\item Cerrado: sea $ \lbrace \ttt_n \rbrace_{ n \in \NN} $ una sucesión de $ \mathrm{cone}\{x_1,\dots, x_M \} $ y sea $ \ttt_0 \in \RR^N $ tal que $ \lbrace \ttt_n \rbrace_{ n \in \NN} \longrightarrow \ttt_0 $. Si notamos $ \ttt_n = \sum_{j=1}^{N}{\mu_j^n x_j }$, entonces:
\[
\lbrace \ttt_n \rbrace_{ n \in \NN} = \lbrace \sum_{j=1}^{N}{\mu_j^n x_j } \rbrace_{ n \in \NN} \longrightarrow \sum_{j=1}^{N}{\mu_j^0 x_j } = \ttt_0.
\]
Como para cada $ \ttt_n $ para $ n \in \NN $ cumple que $ \mu_j^n \geq 0$ con $ j=1,\dots,N $ podemos asegurar que $ \mu_j^0 \geq $ con $ j=1,\dots,N $. Hemos demostrado que $ \ttt_0 $ se expresa como combinación de $ \{x_1,\dots, x_M \} $ con coeficientes no negativos. Por lo tanto, $ \ttt_0 \in \mathrm{cone}\{x_1,\dots, x_M \}  $ y por consiguiente es cerrado. 
\end{itemize} 

Enunciamos ahora otro de los teoremas de la alternativa más conocidos.

\begin{lemaBox}[Lema de Farkas]
Sean $ \{x_1,\dots, x_M \} \subset \RR^N $ y $ b \in \RR^N $ con $ M,N \in \RR^N $. Entonces una, y solo una, de la siguientes afirmaciones se cumple:
\begin{itemize}
\item[i')] $ \exists \mu_1,\dots,\mu_M \geq 0 $ tal que $ b = \sum_{j=1}^{M} \mu_j x_j$.
\item[ii')]$ \exists z_0 \in \RR^N $ que cumple que:
\begin{enumerate}
	\item $ \max_{ j=1,\dots,M} \langle z_0, x_j \rangle \leq 0$ y
	\item $ \langle z_0, b\rangle > 0$.
\end{enumerate} 
\end{itemize}
\end{lemaBox}

\begin{proof}
Planteamos las siguientes alternativas, que obviamente son excluyentes excluyentes, y que implican las de la tesis del lema:
\begin{itemize}
	\item[a)] $b \in \mathrm{cone}\{x_1,\dots, x_M \} $. Estamos en el caso $ i') $  ya que:
	\[
	b \in \left\lbrace \sum_{j=1}^{N}{\mu_j x_j } : \text{ } \mu_1,...,\mu_N \geq 0 \right\rbrace .
	\]
	\item[b)] $ b \notin \mathrm{cone}\{x_1,\dots, x_M \} $. Por su parte, esta alternativa implica $ ii') $. En efecto: 
	
	\begin{itemize}
	\item[$ ii') \Longrightarrow b) $] Razonamos por contradicción. Suponemos que $ b \in \mathrm{cone}\{x_1,\dots, x_M \} $, entonces, podemos expresar $ b =  \sum_{j=1}^{N}{\mu_j x_j } $ con $ \mu_1,...,\mu_N \geq 0$. Como se da $ ii') $, en particular se cumple 1 y obtenemos
	\begin{equation*}
	\begin{split}
		\langle z_0, b \rangle & = \langle z_0, \sum_{j=1}^{N}{\mu_j x_j } \rangle \\
		&= \sum_{j=1}^{N}{\mu_j\langle z_0, x_j \rangle } \leq 0.
	\end{split}
	\end{equation*}
	
	Por otro lado, por 2 se tiene que  $ \langle z_0, b\rangle > 0$. Así, obtenemos que
	\[
	\langle z_0, b \rangle \leq 0 <  \langle z_0, b \rangle,
	\]
	lo cual es imposible.
	\item[$ b) \Longrightarrow ii') $] Si $ b \notin \mathrm{cone}\{x_1,\dots, x_M \} $ aplicamos el teorema de separación (\ref{separacion1}) a los conjuntos $ \{b\} $ que es compacto y convexo y a $ \mathrm{cone}\{x_1,\dots, x_M \} $ que es cerrado y convexo. Obtenemos que:
	\[
	\exists z_0 \in \RR^N: \text{ } \sup_{a \in \mathrm{cone}\{x_1,\dots, x_M \}} \langle z_0, a \rangle < \langle z_0, b \rangle.
	\]
	Por un lado, es obvio que $ 0 \in \mathrm{cone}\{x_1,\dots, x_M \} $ y por ello:
	\[
	0 = \langle z_0, 0\rangle \leq \sup_{a \in \mathrm{cone}\{x_1,\dots, x_M \}} \langle z_0, a \rangle < \langle z_0, b \rangle.
	\]
	Hemos obtenido por tanto que es cierto $ 2 $. Para probar $ 1 $, fijamos $ a_0 \in \mathrm{co}\{x_1,\dots, x_M \}$ (es obvio si $ \mathrm{co}\{x_1,\dots, x_M \} = \{0\}$). Entonces, dado $ \rho > 0 $,
	\[
	\rho\langle z_0, a_0 \rangle = \langle z_0, \rho a_0 \rangle \leq \sup_{a \in \mathrm{cone}\{x_1,\dots, x_M \}} \langle z_0, a_0 \rangle.
	\]
	Llegamos a que el conjunto $ \lbrace\rho\langle z_0, a_0 \rangle: \text{ } \rho > 0 \rbrace  $ está acotado y eso solo es posible si $ \langle z_0, a_0 \rangle \leq 0 $. La arbitrariedad de $ a_0 $ nos aporta que \[ \sup_{a \in \mathrm{cone}\{x_1,\dots, x_M \}} = 0 \] de donde \[
	\max_{ j=1,\dots,M} \langle z_0, x_j \rangle \leq \sup_{a \in \mathrm{cone}\{x_1,\dots, x_M \}} = 0  
	\]
	y, en particular, hemos probado 1.
	\end{itemize}
\end{itemize}
\end{proof}

\chapter{Finanzas}

Nos introducimos ahora en el mundo de las denominadas \textit{matemáticas financieras}. En las secciones anteriores hemos expuesto todas las herramientas matemáticas necesarias para el trabajo y en esta vamos a explicar los conceptos económicos. Una vez se haya completado este apartado estaremos dispuestos a enunciar y probar el enunciado culmen de este trabajo que nos servirá para la valoración de activos financieros en mercados finitos. \\

Nos preguntamos entonces: ``¿qué es un \textit{activo}?''. Un activo o título de valor se define como un recurso con valor que alguien posee con el fin de obtener un beneficio en el futuro. Podemos diferenciar entre activos seguros, como depósitos en el banco o bonos del estado, y activos con riesgo, como las acciones. Uno de los conceptos más importantes que tenemos es nuestro modelo del mercado financiero es el \textit{principio de no arbitraje}. Este principio intuitivamente nos dice que no podemos obtener beneficio si no corremos algún riesgo. Puede resultar un poco confuso ya que acabamos de diferencias entre activos seguros y con riesgo. Esto se debe a que un activo, aunque se llame seguro, no significa que tenga el  beneficio asegurado. Por ejemplo, si tenemos nuestro dinero depositado es posible que el banco quiebre y perdamos todos nuestros ahorros. Así, en la realidad estas oportunidades, llamadas de arbitraje, son muy raras y cuando se dan suponen una ganancia muy pequeña en comparación con la cantidad de dinero que se está manejando globalmente. \\

Cuando nos movemos en el ámbito financiero también debemos de tener en cuenta el \textit{valor del dinero}. Nuestro dinero se va devaluando con el paso del tiempo. Es preferible obtener una cantidad de dinero en este momento que en el futuro ya que no tendremos el mismo poder adquisitivo. Por eso, cuando alguien tiene una deuda debe devolver el dinero con cierto interés porque, de otro modo, sería injusto para la persona que presta el dinero. Además, dicho interés es en cierta medida una estimación ya que no se puede saber con seguridad el precio en un futuro. Lo mismo pasa con los activos con riesgo, solo sabemos el precio que tienen en este momento. Por tanto, es posible que el precio en el futuro sea mayor que el actual o menor. Matemáticamente, podemos representar su valor mediante una variable aleatoria que generalmente mide la ganancia en vez del precio aunque se puede pasar de una otra fácilmente. Podemos suponer una situación con gran número de posibles ganancias intentando abarcar la mayor cantidad de situaciones que nos podríamos encontrar. Sin embargo, el caso binomial en el que solo existen dos posibilidades es el más habitual ya que es lo suficientemente simple de manejar y además refleja bastantes situaciones del mercado financiero real. Este modelo también supone que en cada paso la ganancia tiene el mismo comportamiento.  Tenemos por tanto la variable aleatoria $ K(n):\Omega \longrightarrow (-1.\infty) $ definida como:
\[
K(n) = \begin{cases}
 u & \text{ con probabilidad } p\\
 d & \text{ con probabilidad } 1-p
\end{cases}
\]
cumpliendo $ -1 < d < u $ y $ 0 < p <1 $. La primera condición es importante ya que garantiza que todos los precios van a ser positivos. El espacio de probabilidad $ \Omega $ denota todos los posibles escenarios $ \omega \in \Omega $ en los que varía el precio. Como nos hemos restringido al caso binomial tenemos que $ \Omega = \{ \omega_1, \omega_2\} $. Deberíamos denotar como $ K(n,\omega) $ a la ganancia obtenida en el paso $ n $ si el mercado sigue el escenario $ \omega \in \Omega $.

% Set the overall layout of the tree
\tikzstyle{level 1}=[level distance=2.5cm, sibling distance=3cm]
\tikzstyle{level 2}=[level distance=2.5cm, sibling distance=2cm]

% Define styles for bags and leafs
\tikzstyle{bag} = [text width=4em, text centered]
\tikzstyle{end} = [circle, minimum width=3pt,fill, inner sep=0pt]

% The sloped option gives rotated edge labels. Personally
% I find sloped labels a bit difficult to read. Remove the sloped options
% to get horizontal labels. 
\begin{figure}[h!]
\centering
\begin{tikzpicture}[grow=right, sloped]
\node[bag] {1}
child {
	node[bag] {$ 1+d $}        
	child {
		node[label=right:
		{$ (1+d)^2 $}] {}
		edge from parent
		node[above] {$1-p$}
	}
	child {
		node[label=right:
		{$(1+d)(1+u)$}] {}
		edge from parent
		node[above] {$p$}
	}
	edge from parent 
	node[above] {$1-p$}
}
child {
	node[bag] {$ 1+u $}        
	child {
		node[label=right:
		{$(1+u)(1+d)$}] {}
		edge from parent
		node[above] {$1-p$}
	}
	child {
		node[label=right:
		{$(1+p)^2$}] {}
		edge from parent
		node[above] {$p$}
	}
	edge from parent         
	node[above] {$p$}
};
\end{tikzpicture}
\caption{Ganancias en un árbol binomial de dos pasos.}
\end{figure}
Por lo tanto, si denotamos al precio de una activo en el paso $ n \in \NN$ como $ S(n) $ tenemos que:
\[
S(n) = S(0)(1+u)^i(1+d)^{n-i} \text{ con probabilidad } { n \choose i}p^i(1-p)^{n-i},
\]
donde $ S(0) $ es el precio actual del activo. \\

Los activos que hasta ahora hemos presentado con denominados \textit{primarios} porque son independientes de otros títulos de valor. Por otro lado tenemos los activos \textit{derivados} que son aquellos cuyo valor cambia en función de otros activos denominados subyacentes que pueden ser primarios u otros derivados. Ejemplos de activos derivados son:
\begin{enumerate}
\item Contrato forward (a plazo): es un acuerdo entre dos partes para comprar o vender cierto activo con riesgo a un precio fijo en un momento determinado en el futuro. 
\item Contrato de futuros: es un tipo de contrato forward pero que está estandarizado y negociado en un mercado organizado.
\item Opciones: es un contrato mediante el cual el comprador de la opción adquiere el derecho pero no la obligación de comprar o vender un activo subyacente al vendedor de la misma. El precio al que se puede ejercer el derecho de compra o de venta del activo se denomina precio de ejercicio o también strike price. Existen dos tipos de opciones: 
\begin{enumerate}
	\item Europeas: solo pueden ser ejercidas en la fecha de vencimiento.
	\item Americanas:  se puede ejercer en cualquier momento hasta la fecha de vencimiento.
\end{enumerate}
A su vez, distinguimos entre:
\begin{itemize}
	\item Opciones de compra (\textit{call}): otorga al poseedor de la misma la posibilidad de comprar el activo.
	\item Opciones de venta (\textit{put}): da al poseedor de la misma la posibilidad de vender el activo.
\end{itemize}

En este trabajo nos centraremos en las opciones europeas. 
\end{enumerate} 
%% Incluir en la bibliografía las referencias no citadas
 \nocite{*}
\bibliographystyle{alpha-es}
\bibliography{bibliografia/bibliografia}

%\bibliographystyle{apalike-es}
%\bibliography{bibliografia/bibliografia}
%
%\input{capitulos/03_Planificacion}
%
%\input{capitulos/04_Analisis}
%
%\input{capitulos/05_Diseno}
%
%\input{capitulos/06_Implementacion}
%
%\input{capitulos/07_Pruebas}
%
%\input{capitulos/08_Conclusiones}
%
%%\chapter{Conclusiones y Trabajos Futuros}
%
%
%%\nocite{*}
%\bibliography{bibliografia/bibliografia}\addcontentsline{toc}{chapter}{Bibliografía}
%\bibliographystyle{miunsrturl}
%
%\appendix
%\input{apendices/manual_usuario/manual_usuario}
%%\input{apendices/paper/paper}
%\input{glosario/entradas_glosario}
% \addcontentsline{toc}{chapter}{Glosario}
% \printglossary

\end{document}
